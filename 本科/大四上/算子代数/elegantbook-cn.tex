\documentclass[lang=cn,a4paper,newtx,section]{elegantbook}

\title{算子代数}


\author{h and m}

\date{April 9, 2022}


\setcounter{tocdepth}{3}

\logo{logo-blue.jpg}
\cover{cover.jpg}

% 本文档命令
\usepackage{array}
\usepackage{tikz-cd}
\newcommand{\ccr}[1]{\makecell{{\color{#1}\rule{1cm}{1cm}}}}

% 修改标题页的橙色带
% \definecolor{customcolor}{RGB}{32,178,170}
% \colorlet{coverlinecolor}{customcolor}

\begin{document}

\maketitle
\frontmatter

\tableofcontents

\mainmatter

\chapter{代数基础}

\section{代数的基本概念}

\begin{definition}[F代数]
  $F$为一个域,$A$为$F$上的线性空间,若
  \begin{equation*}
    A \times A \xrightarrow{m} A
  \end{equation*}
  为一个二元运算(结合律),则称$(A,m)$为一个$F$代数。若满足$m$为双线性且满足结合律,则称为结合代数。
\end{definition}

\begin{example}[结合代数]
  典型的结合代数为矩阵代数$M_n(F)$(矩阵代数组成$F$域)。
\end{example}

\begin{example}[函数代数]
  $\Omega$集合,$F$域,$F^\Omega$定义加法、数乘:
  \begin{align*}
    f, g \in F^\Omega, \quad (f+g)(x) &:= f(x) + g(x) \\
    \lambda f(x) &:= \lambda f(x) \\
    f \cdot g(x) &:= f(x)g(x)
  \end{align*}
  为$F$代数。
\end{example}

\begin{definition}[子代数]
  若$A$为$F$代数,$B$为$A$的非空子集,若$B$对$+$、数乘、乘法封闭,则称$B$为一个子代数。
\end{definition}

\begin{proposition}
  若$A$为$F$代数,$B$为非空子集,则
  \begin{equation*}
    B\text{为子代数} \Longleftrightarrow B\text{对}+\text{、数乘、乘法封闭}
  \end{equation*}
\end{proposition}

\begin{proof}
  $(\Rightarrow)$ 显然。
  
  $(\Leftarrow)$ 若$B$乘法双线性还存在,只需验证$B$继承了$A$的双线性性质。对任意$x, y, z \in B$和$\lambda \in F$,由于$B$对加法和数乘封闭,有
  \begin{align*}
    (x+y)z &= xz + yz \in B \\
    x(y+z) &= xy + xz \in B \\
    (\lambda x)y &= \lambda(xy) \in B \\
    x(\lambda y) &= \lambda(xy) \in B
  \end{align*}
  这些性质从$A$继承而来,因此$B$为子代数。
\end{proof}

\begin{example}
  $\ell^\infty(\Omega)$为$\mathbb{C}^\Omega$的子代数。($\ell^\infty(N)$为全体有界序列)
\end{example}

\begin{definition}[理想]
  $A$为一个$F$代数,$I$为$A$的非空子集,满足:
  \begin{enumerate}
    \item $I$为$A$的线性子空间;
    \item $\forall x \in A$,有$xI = \{xy : y \in I\} \subset I$(左吸收性);
    \item $Ix = \{yx : y \in I\} \subset I$(右吸收性)。
  \end{enumerate}
  则称$I$为$A$的一个理想,并称$A$为吸收理想,则称为理想代数。
\end{definition}

\begin{remark}
  一个代数,只有零理想,则称为单代数。例如$M_n(F)$为单代数。
\end{remark}

\begin{definition}[含幺代数]
  若$A$为代数,且存在幺元(单位元$1$),则称$A$为一个含幺代数。
\end{definition}

\begin{definition}[保幺同态]
  若$A \xrightarrow{f} B$同态,$A$、$B$含幺,称$f$为保幺同态若$f(1_A) = 1_B$。
\end{definition}

\begin{proposition}
  $A \xrightarrow{f} B$为代数同态,则$\ker f$为$A$的理想。
\end{proposition}

\begin{proof}
  对任意$x \in A$,$y \in \ker f$,有
  \begin{equation*}
    f(xy) = f(x) \cdot f(y) = f(x) \cdot 0 = 0
  \end{equation*}
  因此$xy \in \ker f$。类似地,
  \begin{equation*}
    f(yx) = f(y) \cdot f(x) = 0 \cdot f(x) = 0
  \end{equation*}
  故$yx \in \ker f$。因此$\ker f$对左右乘法都封闭,即为理想。
  
  此外,由于$f$是线性映射,$\ker f$显然是$A$的线性子空间。综上,$\ker f$为$A$的理想。
\end{proof}

\begin{definition}[商代数]
  $A$为一个$F$代数,$I$为$A$的理想,则$A/I$为商空间。在$A/I$中定义乘法:
  \begin{equation*}
    \forall [x], [y] \in A/I, \quad [x] \cdot [y] = [xy]
  \end{equation*}
\end{definition}

\begin{proof}
  下证良定义:设$[x] = [x']$,$[y] = [y']$,则需证$[xy] = [x'y']$,即$[xy - x'y'] = 0$。
  
  考虑
  \begin{align*}
    xy - x'y' &= xy - xy' + xy' - x'y' \\
    &= x(y - y') + (x - x')y'
  \end{align*}
  
  由于$y - y' \in I$且$x - x' \in I$,而$I$为理想,因此
  \begin{equation*}
    x(y - y') \in I, \quad (x - x')y' \in I
  \end{equation*}
  
  故$xy - x'y' \in I$,即$[xy - x'y'] = 0$。因此乘法定义良定。
\end{proof}

\begin{theorem}[同态基本定理]
  已知$A \xrightarrow{f} B$为代数同态,则存在唯一的单同态$\tilde{f}: A/\ker f \to B$使得下图交换:
  \begin{equation*}
    \begin{tikzcd}
      A \arrow[r, "f"] \arrow[d, "\pi"'] & B \\
      A/\ker f \arrow[ur, "\exists! \tilde{f}"', dashed]
    \end{tikzcd}
  \end{equation*}
  其中$\pi: A \to A/\ker f$为自然投射,$\tilde{f}([x]) = f(x)$,且$\tilde{f}$为单同态。
\end{theorem}

\begin{proof}
  定义$\tilde{f}([x]) = f(x)$。首先验证良定义性:若$[x] = [y]$,则$x - y \in \ker f$,故$f(x - y) = 0$,即$f(x) = f(y)$。
  
  下证$\tilde{f}$保乘法:
  \begin{align*}
    \tilde{f}([x][y]) &= \tilde{f}([xy]) = f(xy) \\
    &= f(x)f(y) = \tilde{f}([x])\tilde{f}([y])
  \end{align*}
  
  下证$\tilde{f}$为单射:若$\tilde{f}([x]) = 0$,则$f(x) = 0$,即$x \in \ker f$,故$[x] = 0$。因此$\ker \tilde{f} = \{0\}$,$\tilde{f}$为单射。
  
  显然$\tilde{f} \circ \pi = f$,且由于$\pi$为满射,$\tilde{f}$由$f$唯一确定。
\end{proof}

\section{谱理论基础}

\begin{definition}[谱]
  $A$为含幺$F$代数,$a \in A$。定义
  \begin{equation*}
    \sigma(a) = \{\lambda \in F : \lambda 1_A - a \text{ 不可逆}\} \subset F
  \end{equation*}
  称$\sigma(a)$为$a$在$A$中的谱(允许$F$)。
\end{definition}

\begin{proposition}
  $A$为含幺代数,$a \in A$。则$a$可逆$\Longleftrightarrow 0 \notin \sigma(a)$。
\end{proposition}

\begin{proof}
  只需证$a$不可逆$\Longleftrightarrow 0 \in \sigma(a)$。
  \begin{align*}
    a\text{不可逆} &\Longleftrightarrow -a\text{不可逆} \\
    &\Longleftrightarrow 0 \cdot 1_A - a\text{不可逆} \\
    &\Longleftrightarrow 0 \in \sigma(a)
  \end{align*}
\end{proof}

\begin{example}
  $M_n(\mathbb{C})$为$\mathbb{C}$代数,$A \in M_n(\mathbb{C})$。则
  \begin{equation*}
    \sigma(A) = \{\lambda : \lambda I_n - A \text{ 不可逆}\}
  \end{equation*}
  为$A$的全体特征值。
  
  反设$\lambda$为$A$的特征值$\in \sigma(A)$。则存在$x \neq 0$使得
  \begin{equation*}
    Ax = \lambda \mathrm{id} \cdot x \Longrightarrow (\lambda \mathrm{id} - A)x = 0 \Longrightarrow \ker(\lambda \mathrm{id} - A) \neq 0
  \end{equation*}
  
  故$\lambda \mathrm{id} - A$非单$\Longrightarrow$不可逆。
\end{example}

\begin{definition}[可除代数]
  若$A$为含幺代数,满足每一个非零元可逆($A \neq 0$),则称$A$为一个可除代数。
\end{definition}

\begin{proposition}
  若$A$为一个可除$F$代数,$\forall a \in A$,$\sigma(a) \neq \emptyset$,则$A \cong F$。
\end{proposition}

\begin{proof}
  构造映射$\theta: F \to A$,$\lambda \mapsto \lambda 1_A$。显然$\theta$为单同态,下证$\theta$为满射。
  
  $\forall a \in A$,由于$\sigma(a) \neq \emptyset$,$\exists \lambda \in \sigma(a)$使得$\lambda 1_A - a$在$A$中不可逆。
  
  由于$A$为可除代数,故$\lambda 1_A - a = 0$,即$\theta(\lambda) = a$。因此$\theta$为满射,$A \cong F$。
\end{proof}

\section{幺元化}

\begin{definition}[幺元化]
  $A$为代数,$\hat{A}$为含幺$F$代数,$\theta: A \to \hat{A}$,若
  \begin{equation*}
    \begin{tikzcd}
      A \arrow[r, "\theta"] \arrow[dr, "f"'] & \hat{A} \arrow[d, "\exists! \hat{f}"', dashed] \\
      & B
    \end{tikzcd}
  \end{equation*}
  对任意含幺代数$B$及$f: A \to B$(保幺同态),使图交换,则称$(\hat{A}, \theta)$为$A$的一个幺元化。
\end{definition}

\begin{remark}[Banach代数幺元化]
  $\hat{A} \cong A \times F$($A \oplus F$)。
\end{remark}

\begin{proposition}
  若$B$为含幺$F$代数,$A$为$B$的子代数,则$A + F \cdot 1$为它含$A$与$1$的最小子代数(Rmk: $k$可以不同于$F$)。
\end{proposition}

\begin{proof}
  显然$A + F \cdot 1$为子代数。
  
  往取$C$为含$A$与$1$的子代数,则$C \supset A$,$C \supset 1$。
  因此$C \supset A + F \cdot 1$。
\end{proof}

\begin{definition}[生成子代数]
  $A$为代数,$\Omega \subset A$,$\forall b$,$\Omega \subset A$中若$b$子代数,称为由$\Omega$生成的子代数。
\end{definition}

\begin{proposition}
  $A + \Omega$生成的子代数存在唯一性。
\end{proposition}

\begin{proof}
  $C \triangleq \cap \{B : \Omega \subset B\text{且}B\text{为}A\text{的子代数}\}$。
\end{proof}

\begin{proposition}
  若$A$为$B$的子代数,$1$为$B$的幺元,则
  \begin{enumerate}
    \item $1 \in A \Longleftrightarrow A \cap F1 = \{F0\}$
    \item $1 \notin A \Longleftrightarrow F1 \subset A$
  \end{enumerate}
\end{proposition}

\begin{proposition}
  若$\hat{A} = A + F1$且$A \cap F1 = \{0\}$(i.e. $A \oplus F1$),则$A \hookrightarrow \hat{A}$,$(\hat{A}, \iota)$为一个幺元化。
\end{proposition}
\begin{definition}[幺元化]
  $A$为代数,$\hat{A}$为含幺$F$代数,若存在映射$\theta: A \to \hat{A}$,使得对任意含幺代数$B$和代数同态$f: A \to B$,存在唯一的代数同态$\hat{f}: \hat{A} \to B$满足下图交换:
  \begin{equation*}
    \begin{tikzcd}
      A \arrow[r, "\theta"] \arrow[dr, "f"'] & \hat{A} \arrow[d, "\exists! \hat{f}"', dashed] \\
      & B \text{(含幺)}
    \end{tikzcd}
  \end{equation*}
  则称$(\hat{A}, \theta)$为$A$的一个幺元化。
\end{definition}

\begin{proof}
  若$\hat{f}$存在,则$\hat{f}(a + \lambda 1) = f(a) + \lambda 1_B$,故$\hat{f}$唯一。
  
  下证存在。定义$\hat{f}: \hat{A} \to B$,由$\hat{A} = A \oplus F1$,
  \begin{equation*}
    a + \lambda 1 \longmapsto f(a) + \lambda 1_B
  \end{equation*}
  
  则$\hat{f}$为代数同态,故幺元化存在且唯一。
\end{proof}
\begin{theorem}[幺元化存在性]
  若$A$为$F$代数,则$A$存在幺元化。
\end{theorem}

\begin{proof}
  定义$\hat{A} = A \oplus F$。在$\hat{A}$中定义乘法:
  \begin{equation*}
    (a, \lambda) \cdot (b, \mu) \triangleq (ab + \mu a + \lambda b, \lambda \mu)
  \end{equation*}
  
  验证这定义了一个含幺代数结构,幺元为$(0, 1)$。
  
  定义$\theta: A \to \hat{A}$,$a \mapsto (a, 0)$,则$\theta$为代数单同态。
  
  对任意含幺代数$B$和代数同态$f: A \to B$,若存在$\hat{f}: \hat{A} \to B$使得下图交换:
  \begin{equation*}
    \begin{tikzcd}
      A \arrow[r, "\theta"] \arrow[dr, "f"'] & \hat{A} \arrow[d, "\exists! \hat{f}"', dashed] \\
      & B
    \end{tikzcd}
  \end{equation*}
  
  则$\hat{f}$存在。定义$\hat{f}: \hat{A} \to B$,
  \begin{equation*}
    a + \lambda 1 \mapsto f(a) + \lambda 1_B
  \end{equation*}
  
  若$\hat{f}$存在,则$\hat{f}(a + \lambda 1) = f(a) + \lambda 1_B$,故$\hat{f}$唯一。
  
  下证存在。定义$\hat{f}: \hat{A} \to B$,由$\hat{A} = A \oplus F1$,
  \begin{equation*}
    a + \lambda 1 \mapsto f(a) + \lambda 1_B
  \end{equation*}
  
  则$\hat{f}$为代数同态,故幺元化唯一。
\end{proof}

\begin{proposition}[幺元化唯一性]
  若$A$有两个幺元化$(B_1, \theta_1)$及$(B_2, \theta_2)$,则$B_1 \cong B_2$,代数同构。
\end{proposition}

\begin{proof}
  由泛性质,存在唯一的代数同态$\sigma: B_1 \to B_2$和$\tau: B_2 \to B_1$使得下图交换:
  \begin{equation*}
    \begin{tikzcd}
      B_1 \arrow[rr, "\exists! \sigma", bend left=30] \arrow[loop left, "Id_{B_1}"] & & B_2 \arrow[ll, "\exists! \tau", bend left=30] \arrow[loop right, "Id_{B_2}"] \\
      & A \arrow[ul, "\theta_1"] \arrow[ur, "\theta_2"'] &
    \end{tikzcd}
  \end{equation*}
  
  由于$(B_1, \theta_1)$和$(B_2, \theta_2)$都是幺元化,应用泛性质:
  \begin{itemize}
    \item 对$(B_1, \theta_1)$,存在唯一$\sigma: B_1 \to B_2$使得$\sigma \circ \theta_1 = \theta_2$
    \item 对$(B_2, \theta_2)$,存在唯一$\tau: B_2 \to B_1$使得$\tau \circ \theta_2 = \theta_1$
  \end{itemize}
  
  则$\tau \circ \sigma: B_1 \to B_1$满足$(\tau \circ \sigma) \circ \theta_1 = \theta_1$,而$Id_{B_1}$也满足此性质。由唯一性,$\tau \circ \sigma = Id_{B_1}$。
  
  同理,$\sigma \circ \tau = Id_{B_2}$。故$B_1 \cong B_2$。
\end{proof}

\begin{proposition}
  若$A \xrightarrow{\theta} \hat{A} = \theta(A) \oplus F1_{\hat{A}}$,$A \xrightarrow{\sigma} B$为含幺代数同态,则$B = \sigma(A) \oplus F1_B$。
\end{proposition}

\begin{proof}
  考虑下图:
  \begin{equation*}
    \begin{tikzcd}
      A \arrow[r, "\theta"] \arrow[dr, "\sigma"'] & \hat{A} = \theta(A) \oplus F1_{\hat{A}} \arrow[d, "\exists! \tau", dashed] \\
      & B
    \end{tikzcd}
  \end{equation*}
  
  由幺元化的泛性质,存在唯一的代数同态$\tau: \hat{A} \to B$使得$\tau \circ \theta = \sigma$。
  
  因此$B = \sigma(A) \oplus F1_B$。
\end{proof}

\begin{example}
  若$\Omega$为局部紧$T_2$空间,$\Omega$非紧$\Rightarrow \hat{\Omega}$为全幺元化。
  
  $C_0(\Omega)$表示$\Omega \to \mathbb{C}$的全体在无穷远处趋于$0$的连续函数($\forall \varepsilon > 0$,$\exists K \subset \Omega$紧,$\text{s.t.} \forall x \in \Omega \setminus K$,有$|f(x)| < \varepsilon$)。
\end{example}

\begin{proposition}
  $C_0(\Omega)$为一个$\mathbb{C}$代数(不一定含幺)。
  
  $C(\hat{\Omega})$表示$\hat{\Omega}$上的全体连续函数(含幺代数)。
  
  则$C(\hat{\Omega})$为$C_0(\Omega)$的幺元化。
  
  定义$\theta: C_0(\Omega) \to C(\hat{\Omega})$,$f \mapsto \hat{f}$,其中
  \begin{equation*}
    \hat{f}(x) = \begin{cases} f(x) & x \neq \infty \\ 0 & x = \infty \end{cases}
  \end{equation*}
\end{proposition}

\section{谱的性质}

\begin{proposition}
  若$A$为含幺代数,$a, b \in A$,则
  \begin{equation*}
    \sigma(ab) \cup \{0\} = \sigma(ba) \cup \{0\}
  \end{equation*}
\end{proposition}

\begin{proof}
  只要证:$\forall \lambda \in F$,若$\lambda \neq 0$,则$\lambda 1 - ab$可逆$\Longleftrightarrow \lambda 1 - ba$可逆。
\end{proof}

\begin{lemma}
  若$A$为环,$a, b \in A$,则$1 - ab$可逆$\Longleftrightarrow 1 - ba$可逆。
\end{lemma}

\begin{proof}
  显然:$\lambda - ab$可逆$\Longleftrightarrow 1 - \frac{1}{\lambda}ab$可逆。
\end{proof}

\begin{theorem}
  若$A \hookrightarrow \hat{A}$为幺元化,$A$含幺元$e$,则有$\forall a \in A$,$\sigma_{\hat{A}}(a) \cup \{0\} = \sigma_A(a)$
  
\end{theorem}

\begin{proof}
  ($\subset$) $\forall \lambda \in \sigma_A(a)$。
  
  (1)是$\lambda = 0$,$A$是$\hat{A}$的真理想。
  
  
  \textbf{Rmk}: $1$为$A$的代数理想$\Longleftrightarrow I$为$A$的环理想,因为$\lambda x = \lambda \cdot 1_{\hat{A}} \cdot x = (\lambda 1_{\hat{A}})x \in I_{\hat{A}}$。
  
  $A$是$\hat{A}$的真理想$\Rightarrow a \in A \Rightarrow e$在$\hat{A}$中可逆$e \Rightarrow a$在$\hat{A}$中不可逆
  
  $\Rightarrow -a = 0 \cdot 1 - a$在$\hat{A}$中不可逆$\Rightarrow 0 \in \sigma(a)_{\hat{A}}$
  
  由$\lambda \in \sigma_{\hat{A}}(a)$,得$\lambda e - a$在$A$中不可逆,要证$\lambda 1 - a$在$\hat{A}$中不可逆。
  
  考虑交换图:
  \begin{equation*}
    \begin{tikzcd}
      A \arrow[r, "\theta", hook] \arrow[dr, "\varphi"'] & \hat{A} \arrow[d, "\exists! \psi", dashed] \\
      & A
    \end{tikzcd}
  \end{equation*}
  其中$\psi: \hat{A} \to A$,$a + \lambda 1 \longmapsto \varphi(e) + \lambda \varphi(1) = a + \lambda e$。
  
  验证$\psi(-a + \lambda 1) = -a + \lambda e$。$\psi$保存$\Rightarrow$保可逆性,由$\lambda e - a$不可逆,知$\lambda 1 - a$不可逆。
  
  ($\supset$) 下证$\sigma_A(a) \subset \sigma_{\hat{A}}(a) \cup \{0\}$。
  
  若$0 \in \sigma_{\hat{A}}(a) \Rightarrow 0 \in$右边。
  
  若$\lambda \neq 0$,由$\lambda \in \sigma_{\hat{A}}(a)$,$\lambda 1 - a$在$\hat{A}$中不可逆。
  
  要证$\lambda \in \sigma_A(a)$,即$\lambda e - a$在$A$中不可逆。
  
  反证:若$\lambda e - a$在$A$中可逆,设$b \in A$使得$(\lambda e - a)b = b(\lambda e - a)$。
  
  要证$\lambda 1 - a$在$\hat{A}$中有逆元$b + \frac{1}{\lambda}(1 - e)$:
  \begin{equation*}
    (\lambda 1 - a)\left(b + \frac{1}{\lambda}(1 - e)\right) = (\lambda 1 - a)b + (\lambda 1 - a)\frac{1}{\lambda}(1 - e) = e + 1 - e - 0 = 1
  \end{equation*}
  
  \begin{equation*}
    \begin{tikzcd}
      \hat{A} \arrow[r, "\varphi"] & A \\
      \lambda 1 + a \arrow[r, mapsto] & \lambda e + a
    \end{tikzcd}
  \end{equation*}
  其中$\ker \varphi = F(1 - e)$。
  
  若存在$c$使得$(\lambda e - a)\varphi(c) = e = \varphi(c)(\lambda e - a)$,考虑映射:
  \begin{equation*}
    \begin{tikzcd}
      \hat{A} \arrow[r, "\varphi"] & I_0 \\
      x + \lambda 1 \arrow[r, mapsto] & x + \lambda e \\
      \lambda 1 - a \arrow[r, mapsto] & \lambda e - a \\
      c \arrow[r, mapsto] & \varphi(c)
    \end{tikzcd}
  \end{equation*}
  
  由$\varphi(e) = e$,$b \in A \Rightarrow b = \varphi(b)$,$\varphi(b - c) = 0 \Rightarrow b - c \in \ker\varphi$。
  
  $\ker\varphi = \{x + \lambda 1 : x + \lambda e = 0\} = \{x + \lambda 1 : x = -\lambda e\} = \{-\lambda e + \lambda 1, \lambda \in F\} = F(1 - e)$。
  
  由$b - c \in \ker\varphi$,设$c - b = \mu(1 - e)$,则$c = b + \mu(1 - e)$。
  
  由$(\lambda 1 - a)c = 1 = c(\lambda 1 - a) \Rightarrow \mu = \frac{1}{\lambda}$。
\end{proof}

\section{谱的例子}

$\Omega$为Top sp.,$C(\Omega)$表示$\Omega$上~全体$\mathbb{C}$值连续函数。

\begin{example}
    $C(\Omega)$为$\mathbb{C}$代数
\end{example}


\textbf{左图(乘法连续):}
\begin{equation*}
  \begin{tikzcd}
    x \arrow[r, mapsto] \arrow[d] & \Omega \arrow[r, "f \cdot g"] \arrow[d, "g"] & \mathbb{C} \arrow[d, "\text{乘法连续}"] \\
    (x, y) \arrow[r, mapsto] & \Omega \times \Omega \arrow[r, "f \times g"] & \mathbb{C} \times \mathbb{C}
  \end{tikzcd}
\end{equation*}
其中$(x, y) \longmapsto (f(x), f(y))$

\textbf{右图(标量乘法):}
\begin{equation*}
  \begin{tikzcd}
    x \arrow[r, mapsto] \arrow[d, "f"'] & \Omega \arrow[r, "\lambda f"] \arrow[d, "f"'] & \mathbb{C} \\
    f(x) & \mathbb{C} \arrow[ur, "\ell_\lambda"'] &
  \end{tikzcd}
\end{equation*}
\begin{example}
    $C_b(\Omega)$表示$\Omega$上全体有界连续函数(Def: $f: \Omega \to \mathbb{C}$有界$\Longleftrightarrow \ell^\infty(\Omega) = C_b(\Omega)$(若$\Omega$赋离散拓扑)

\end{example}
\begin{example}
    $f \in C(\Omega)$为谱,$\sigma(f) = f(\Omega)$。(像是谱)

\end{example}

\begin{proof}
  $\sigma(f) = \{\lambda \in \mathbb{C} : \lambda 1 - f$不可逆$\} = \{\lambda \in \mathbb{C} : \exists x, \lambda - f(x) = 0\}$
  
  $\Omega \xrightarrow{f} \mathbb{C}$可逆$\Longleftrightarrow \forall x \in \Omega$,$f(x) \neq 0$。
  
  $\frac{1}{f}(x) \triangleq  \frac{1}{f_x}$。$\Omega \xrightarrow{f} \mathbb{C} \setminus \{0\}$
  
  \begin{equation*}
    \begin{tikzcd}
      \Omega \arrow[r, "f"] \arrow[dr, "\frac{1}{f}"'] & \mathbb{C} \setminus \{0\} \arrow[d] \\
      & \mathbb{C}
    \end{tikzcd}
  \end{equation*}
\end{proof}

$f $在$C_b(\Omega)$中的谱,$\sigma(f) = \overline{f(\Omega)}$

\begin{remark}
  $C_b(\Omega)$中$f$可逆$\Longleftrightarrow \exists \delta > 0$,使$\forall x \in \Omega$有$|f(x)| > \delta$。
\end{remark}

\begin{proof}
  ($\Rightarrow$) 设$f \cdot g = 1 = g \cdot f$,$\Rightarrow g = \frac{1}{f}$。
  
  $\forall x \in \Omega$,$|g(x)| = \frac{1}{|f|} > \frac{1}{M} > 0$
  
  ($\Leftarrow$) 由已知$\frac{1}{f}$存在,连续。
  
  $\forall x$,$\left|\frac{1}{f(x)}\right| = \frac{1}{|f(x)|} < \varepsilon$,$\frac{1}{f} \in C_b(\Omega)$
  
  ($\Leftarrow$) 上述等价于$d(f(\Omega), 0) > 0$
  
  \end{proof}
\begin{proposition}
    $f$不可逆$\Longleftrightarrow d(f(\Omega), 0) = 0 \Longleftrightarrow 0 \in \overline{f(\Omega)}$

\end{proposition}
\begin{proof}
  $\lambda 1 - f$不可逆$\Longleftrightarrow 0 \in \overline{(\lambda 1 - f)(\Omega)}$
  
  $(\lambda 1 - f)(x) = \lambda 1 x - f(\lambda)$
  
  ($\Rightarrow$) $\overline{(\lambda 1 - f)(\Omega)} = \overline{\lambda - f(\Omega)} = \lambda - \overline{f(\Omega)}$
  
  $\Longleftrightarrow \lambda \in \overline{f(\Omega)}$
\end{proof}

\section{谱与特征}

\begin{definition}[谱集]
  $A$为$F$代数,$\Sigma A$表示$A$上所有唯一0同态$\tau$的$F$值函数集合。
  
  即$\Sigma A \triangleq \{\tau \in F^A : \tau$的唯一0同态$\}$,称为$A$的谱。
  
  ($A \xrightarrow{\tau} F$是0同态,称为$A$的一个特征)
\end{definition}

\begin{proposition}
  若$A$为含幺代数,$\tau$为特征$\Longleftrightarrow \tau$为保幺同态。
\end{proposition}

\begin{proof}
  ($\Leftarrow$) 保幺:$\tau(1) = 1$。
  
  ($\Rightarrow$) $A \xrightarrow{\tau} F$
  
  $1 \longmapsto \tau(1)$为$\tau(A)$中幺元,$\dim F = 1 \Rightarrow \tau$满射,$\tau(A) = F$。
\end{proof}

\subsection{Banach代数}

Banach代数$A$,$a \in A$。

\begin{theorem}
  $A$为含幺交换Banach代数,$a \in A$则
  \begin{equation*}
    \sigma(a) = \{\tau(a) : \tau \in \Sigma A\}
  \end{equation*}
\end{theorem}

\begin{definition}[谱性质]
  若$F$代数$A$满足,$\forall B$为$F$代数,$A \xrightarrow{f} B$满同态,有$\forall b \in B$,$\sigma_B(b) \neq \phi$,
  称$A$具有谱性质(易知$A \neq \{0\}$)。
  
  e.g. 交换含幺Banach代数具有谱性质。
\end{definition}

\begin{proposition}
  $A$具有谱性质$\Longleftrightarrow \forall M$为$A$的极大理想,$A/M$每个元素谱非空。
\end{proposition}

\begin{proposition}
  $A \xrightarrow{\varphi} B$保幺代数同态,$a \in A$则$\sigma_B(\varphi(a)) \subset \sigma_A(a)$。
\end{proposition}

\begin{proof}
  $\forall \lambda \in \sigma_B(\varphi(a)) \Rightarrow \lambda 1_B - \varphi(a)$在$B$中不可逆
  
  $\Rightarrow \lambda 1_A - a$中不可逆$\Rightarrow \lambda \in \sigma_A(a)$
\end{proof}

\begin{proof}
  ($\Leftarrow$) 因$A \xrightarrow{\pi} A/M$满同态$\Rightarrow A/M$每个元素谱非空。
  
  ($\Leftarrow$) 往取$A \xrightarrow{f} B \neq 0$,满同态,$\ker f$为$A$的真理想,故$\ker f$含于某个极大理想$M$。
  
  有交换图:
  \begin{equation*}
    \begin{tikzcd}
      A \arrow[r, "f"] \arrow[d, "\pi"'] & B \\
      A/M \arrow[ur, "\exists! \varphi"', dashed] &
    \end{tikzcd}
  \end{equation*}
  存在唯一同态(满)$\Rightarrow$保幺。
  
  由已知$A/M$中每个元素谱非空$\Rightarrow B$中每个元素谱非空。
\end{proof}

\begin{theorem}
  若$A$为交换含幺代数且具有谱性质,则
  \begin{equation*}
    \forall a \in A, \quad \sigma(a) = \{\tau(a) : \tau \in \Sigma A\}
  \end{equation*}
\end{theorem}

\begin{proof}
  ($\supset$) $\forall \tau \in \Sigma A$,须证$\tau(a) \in \sigma(a)$。
  
  i.e. $\tau(a)1 - a$在$A$中不可逆。
  
  $A \xrightarrow{\tau} F$保幺:
  
  $\tau(a)1 - a \longmapsto \tau(a) - \tau(a) = 0$,不可逆$\Rightarrow \tau(a)1 - a$在$A$中不可逆。
  
  ($\subset$) $\forall \lambda \in \sigma(a)$,要证$\lambda$形如某个$\tau(a)$。
  
  由已知$\lambda 1 - a$在$A$中不可逆。
\end{proof}

\begin{remark}
  若$R$为交换幺环,$\lambda$为不可逆元,则$\exists M$极大理想,s.t. $\lambda \in M$。
\end{remark}

\begin{proof}
  对于$\lambda$,$R\lambda$为含幺的最小理想,由$\lambda$不可逆
  
  $\Rightarrow R\lambda \subsetneq R$(包含,若$R\lambda = R$,$1 \in R\lambda$,$\exists y \in R$,$y\lambda = 1$)。
  
  仍存在极大理想$M \supset R\lambda \ni \lambda$,因
  
  由$A$的交换幺环,$\exists M$为极大理想。
  
  考虑$A \xrightarrow{\pi} A/M$商代数。
  
  $A/M$为域,可除代数,由$A$有谱性质。
  
  $A/M$每个元素谱非空,由已证$A/M \cong F$。
  
  交换图:
  \begin{equation*}
    \begin{tikzcd}
      A \arrow[r, "\pi"] \arrow[rr, bend right=40, "\tau = \theta \circ \pi"'] & A/M \arrow[r, "\theta"] & F
    \end{tikzcd}
  \end{equation*}
  
  $\tau(a)$满足:
  
  $\lambda 1 - a \longmapsto 0 \longmapsto 0$,$\tau(\lambda 1 - a) = 0$
  
  $\Rightarrow \tau(\lambda 1) = \tau(a) \Rightarrow \lambda = \tau(a)$
\end{proof}

\begin{proposition}
  $\Sigma A \cup \{0\} \xrightarrow{\Theta} \Sigma \hat{A}$为双射。
  
  $\tau \longmapsto \hat{\tau}$
  
  交换图:
  \begin{equation*}
    \begin{tikzcd}
      A \arrow[r] \arrow[dr, "\tau"'] & \hat{A} \arrow[d, "\exists! \hat{\tau}"] \\
      & F
    \end{tikzcd}
  \end{equation*}
\end{proposition}

\begin{proof}
  $\tau_1 \longmapsto \hat{\tau}_1$,若$\hat{\tau}_1 = \hat{\tau}_2 \Rightarrow \tau_1 = \hat{\tau}_1|_A = \hat{\tau}_2|_A = \tau_2$
  
  $\tau_2 \longmapsto \hat{\tau}_2 \Rightarrow$单射
\end{proof}

\begin{proposition}
  若$A$为含幺代数,$\Omega \subset A$,则
  \begin{equation*}
    \langle \Omega \cup \{1\} \rangle_{al} = \Omega\text{生成的含幺的含幺代数}
  \end{equation*}
\end{proposition}

\begin{proof}
  记$B = \langle \Omega \cup \{1\} \rangle_{al}$,$C$为$\Omega$生成的含幺的含幺代数。
  
  ($\supset$) 要证$C \supset B$。
  
  因$C$是含$\Omega$的含幺子代数,故$1_A \in C$,从而$\Omega \cup \{1_A\} \subset C$。
  
  由$B = \langle \Omega \cup \{1\} \rangle_{al}$是包含$\Omega \cup \{1\}$的最小子代数,得$B \subset C$。
  
  ($\subset$) 要证$B \supset C$。
  
  由$\Omega \subset \Omega \cup \{1\} \subset B$,故$B$是包含$\Omega$的子代数。
  
  又$1_A \in \Omega \cup \{1\} \subset B$,故$B$是含幺子代数。
  
  由$C$是包含$\Omega$的最小含幺子代数,得$C \subset B$。
  
  综上,$B = C$。
\end{proof}

\begin{proposition}
  若$A$为含幺代数(不一定交换)
  \begin{equation*}
    \Omega \subset A \Rightarrow \langle \Omega \rangle = A \cdot \Omega \cdot A = \left\{\sum a_i \cdot x_i \cdot b_i, \quad \substack{a_i \in A \\ x_i \in \Omega \\ b_i \in A}\right\}
  \end{equation*}
\end{proposition}

\begin{proposition}
  $A$为代数,$\Omega \subset A$,若$\forall x, y \in \Omega$,$xy = yx$,则$\langle \Omega \rangle_{al}$为交换子代数($\Omega$称为交换子集)
\end{proposition}

\begin{definition}[对换]
  $A$为代数,$\Omega \subset A$,$\Omega \neq \emptyset$(结合)
  \begin{equation*}
    \Omega' \triangleq \{a \in A : \forall x \in \Omega, ax = xa\}
  \end{equation*}
\end{definition}

\begin{proposition}
  $\Omega'$为子代数,$\forall a, b \in \Omega'$,$(a+b) \in \Omega'$,$ab \in \Omega'$,$\lambda a \in \Omega'$
\end{proposition}

\begin{proof}
  $\langle \Omega \rangle_{al}$交换$\Longleftrightarrow \langle \Omega \rangle_{al} \subset \langle \Omega \rangle_{al}'$
  
  $\Longleftrightarrow \Omega \subset \langle \Omega \rangle_{al}' \Longleftrightarrow \langle \Omega \rangle_{al} \subset \Omega'$
  
  $\Longleftrightarrow \Omega \subset \Omega' \Longleftrightarrow \Omega$交换
\end{proof}

\section{多项式代数}

\begin{definition}
  $F[x]$表示以$x$为变元的多项式代数。
\end{definition}

\begin{proposition}
  若$A$为含幺$\sim F$代数,$a \in A$,则存在唯一同态$\hat{a}$:
  \begin{equation*}
    \begin{aligned}
      F[x] &\xrightarrow{\hat{a}} A \quad f_{a \in t}, \quad \hat{a}(f) = f(a) \\
      x &\longmapsto a
    \end{aligned}
  \end{equation*}
\end{proposition}

\begin{theorem}[谱映射定理]
  若$A$为含幺$F$代数,$F$代数同态,$a \in A$,$f \in F[x]$,则
  \begin{equation*}
    \sigma(f(a)) = f(\sigma(a))
  \end{equation*}
  其中$f(\sigma(a)) \triangleq \{f(\lambda) : \lambda \in \sigma(a)\}$。
\end{theorem}

\begin{remark}
  若$S$为幺半群,$x, y \in S$,且$xy = yx$,则$xy$可逆$\Longleftrightarrow x$与$y$可逆。
\end{remark}

\begin{proof}
  ($\Leftarrow$) 显然
  
  ($\Rightarrow$) $xy \cdot a := e = a \cdot xy = a \cdot yx$,$x$有右逆,$x$有左逆$\Rightarrow x$可逆。
\end{proof}

\begin{proof}
  ($\subset$) $\forall \lambda \in \sigma(f(a)) \Rightarrow \lambda 1 - f(a)$不可逆。
  
  $\Rightarrow (\lambda - f) \cdot a$,设$(\lambda - f) = c(x - \lambda_1) \cdots (x - \lambda_n)$,$c \neq 2$
  
  $(\lambda - f) \cdot a = c(a - \lambda_1 1) \cdots (a - \lambda_n 1)$
  
  不可逆$\Rightarrow \exists \lambda_i$,s.t. $a - \lambda_i 1$不可逆。
  
  $\Rightarrow \lambda_i \in \sigma(a)$
  
  而$(\lambda - f)(\lambda_i) = \lambda - f(\lambda_i)$
  
  $= 0 \Rightarrow \lambda = f(\lambda_i) \Rightarrow \lambda \in f(\sigma(a))$
  
  ($\supset$) $\forall \lambda \in f(\sigma(a))$,要证$\lambda \in \sigma(f(a))$
  
  i.e. $\lambda 1 - f(a)$不可逆。
\end{proof}

\section{乘子代数}

$(A^{op}, +, \cdot,$ 数乘$)$,$A^{op} \triangleq A$,加法、数乘不变
  
$x \cdot y \triangleq yx$(反乘法)易知$A^{op}$为代数


$A, B$为$F$代数,直和代数$A \oplus B$,乘法
\begin{equation*}
  (a, b) \cdot (a', b') \triangleq (aa', bb')
\end{equation*}
满足结合律、分配律

\begin{definition}
  $\operatorname{End}_F(A) \oplus \operatorname{End}_F(A)^{op}$为一个$F$代数
  \begin{align*}
    A &\xrightarrow{\ell} \operatorname{End}(A) & A &\xrightarrow{r} \operatorname{End}(A)^{op} \quad \text{同态}\\
    a &\longmapsto \ell_a & a &\longmapsto r_a
  \end{align*}
\end{definition}

$A \xrightarrow{\varphi} C$同态,$A \xrightarrow{\psi} D$同态,定义$A \xrightarrow{(\varphi, \psi)} C \oplus D$为同态
\begin{align*}
  x &\longmapsto (\varphi(x), \psi(x)) \\
  x + y &\longmapsto (\varphi(x) + \varphi(y), \psi(x) + \psi(y)) \\
  xy &\longmapsto (\varphi(x) \cdot \varphi(y), \psi(x) \cdot \psi(y)) \\
  1 &\longmapsto (1_C, 1_D) \text{ 若保幺}
\end{align*}

\begin{definition}
   \begin{equation*}
    \begin{aligned}
      A &\xrightarrow{\Theta} \operatorname{End}_F(A) \oplus \operatorname{End}_F(A)^{op} \text{为代数同态}\\
      a &\longmapsto (\ell_a, r_a)
    \end{aligned}
  \end{equation*}
\end{definition}

\begin{proposition}
  $\forall a \in A$,有以下性质
  \begin{enumerate}
    \item $\ell_a(x)y = \ell_a(xy)$
    \item $x r_a(y) = r_a(xy)$
    \item $x \ell_a(y) = x ay = r_a(x) y$
  \end{enumerate}
\end{proposition}

\begin{definition}
  $\forall (L, R) \in \operatorname{End}_F(A) \oplus \operatorname{End}_F(A)^{op}$
  
  称$(L, R)$为$A$上的一个乘子,若满足,$\forall x, y \in A$,有
  \begin{enumerate}
    \item $L(x)y = L(xy)$
    \item $x R(y) = R(xy)$
    \item $x L(y) = R(x)y$
  \end{enumerate}
  
  $A$上全体乘子记为$M(A)$。
\end{definition}

\begin{proposition}
  $M(A)$为$\operatorname{End}_F(A) \oplus \operatorname{End}_F(A)^{op}$的含幺子代数,幺元为$(id_A, id_A)$
\end{proposition}

\begin{proof}
  $(L, R), (L', R') \in M(A)$
  
  $(L, R) + (L', R') \in M(A)$?
  
  $\forall x, y \in A$,$\circled{1}$ $(L + L')(x)y = (L(x) + L'(x))y = L(x)y + L'(x)y$
  
  $= L(xy) + L'(xy) = (L + L')(xy)$
  
  同理$R + R'$满足$\circled{2}$
  
  $x(L + L')(y) = x(L(y) + L'(y)) = xL(y) + xL'(y) = R(x)y + R'(x)y$
  
  $= (R + R')(x)y$ $\circled{3}$
  
  数乘易证。
  
  $(L, R) \cdot (L', R') = (L \circ L', R \circ R') \in M(A)$
  
  $(L \circ L')(x)y = L(L'(x))y = L(L'(x)y) = L(L'(xy))$
  
  $= (L \circ L')(xy)$ $\circled{1}$
  
  $x(L \circ L')(y) = xL(L'(y)) = R(x)L'(y) = R'(R(x))y$
  
  $= (R' \circ R)(x)y$ $\circled{3}$
  
  $x(R' \circ R)(y) = xR'(R(y)) = R'(xR(y))$
  
  $= R'(R(xy)) = R' \circ R(xy)$
\end{proof}

\begin{definition}[典范同态]
  \begin{equation*}
    \begin{aligned}
      A &\xrightarrow{\Theta} M(A) \\
      a &\longmapsto (\ell_a, r_a)
    \end{aligned}
  \end{equation*}
  
  $M(A)$称为$A$的乘子代数
\end{definition}

\begin{proposition}
  $A \xrightarrow{\Theta} M(A)$典范同态,则$\Theta(A) \triangleleft M(A)$理想。
\end{proposition}

\begin{proof}
  $\Theta(A)$线性$op$,下证、吸收率。
  
  $\forall a \in A$,$(\ell_a, r_a) \in \Theta(A)$,$\forall (L, R) \in M(A)$,要证
  
  $(L, R)(\ell_a, r_a)$及$(\ell_a, r_a)(L, R)$都在$\Theta(A)$中。
  
  $(L \ell_a, R r_a)$=$(\ell_b, r_b)$,$(l_a L, R r_a)$=$(\ell_c, r_c)$
  
  若$A$含幺1代入上式,$L\ell_a(1) =L(a)= \ell_b(1) = b$
  
  $R r_a(1) = R(a)= r_c(1) = c$
  
  $\Rightarrow$下证$(L \ell_a, r_a R) = (\ell_{L(a)}, r_{L(a)})$
  
  及$(l_a L, R r_a) = (\ell_{R(a)}, r_{R(a)})$
  
  $\forall x \in A$,有$(L \ell_a)(x) = L(ax) = L(a) x = \ell_{L(a)}(x)$
  
  $(r_a R)(x) = R(xa) = x R(a) = r_{L(a)}(x)$
  
  $(l_a L)(x) = a L(x) = R(a)x = \ell_{R(a)}(x)$
  
  $(R r_a)(x) = R(xa) = x R(a) = r_{R(a)}(x)$
\end{proof}

\begin{proposition}
  $A$含幺 $\Leftrightarrow$ $A \xrightarrow{\theta} M(A)$同构
\end{proposition}

\begin{proof}
  ($\Leftarrow$) 显然。
  
  ($\Rightarrow$) $A \xrightarrow{\theta} M(A)$单射。
  
  若$a \longmapsto (0, 0)$,则$\ell_a = 0$,$r_a = 0$$\Rightarrow \ell_a(1) = 0 \Rightarrow a = 0$
  
  $\ker \theta = \{0\} \Rightarrow \theta$单射
  
  $\theta(A)$是理想,含幺,$\Rightarrow \theta(A) = M(A)$
  
  故$\theta$为同构。
\end{proof}

\begin{theorem}
  任取含幺$F$代数$B$,若$A$是$B$的理想,则$\exists B \xrightarrow{\tau} M(A)$保幺同态,s.t. $\tau|_A = \theta$。
  
  \begin{equation*}
    \begin{tikzcd}
      A \arrow[r, hook] \arrow[dr, "\theta"'] & B \arrow[d, "\tau"] \\
      & M(A)
    \end{tikzcd}
  \end{equation*}
\end{theorem}

\begin{proof}
  定义
  \begin{equation*}
    \begin{aligned}
      B &\xrightarrow{\tau} M(A) \\
      b &\longmapsto (L_b, R_b)
    \end{aligned}
  \end{equation*}
  其中
  \begin{equation*}
    \begin{aligned}
      A &\xrightarrow{L_b} A, & A &\xrightarrow{R_b} A \\
      x &\longmapsto bx, & x &\longmapsto xb
    \end{aligned}
  \end{equation*}
  
  易知$L_b, R_b \in \operatorname{End}_F(A)$且$(L_b, R_b)$为乘子。
  
  e.g. $M(C(X)) = C(X)$\quad
    $M(K(H)) = B(H)$
\end{proof}

\begin{definition}
  称代数(或环)$A$非退化,若满足$\operatorname{ann}(A) = \{0\}$
\end{definition}

\begin{definition}
  $\Omega \subset A$,$\operatorname{ann}(\Omega) := \{a \in A : a\Omega = \{0\} = \Omega a\}$
  
  i.e. $\forall x \in A$,若$xA = Ax = 0 \Rightarrow x = 0$(含$1$必不退化,$a1 = a = 0$)
\end{definition}

\begin{proposition}
  若$A$非退化,则$A \xrightarrow{\theta} M(A)$为单射。
\end{proposition}

\begin{proof}
  $A \xrightarrow{\theta} M(A)$
  
  若$a \longmapsto (\ell_a, r_a) = (0, 0)$,$\ell_a = 0$,$r_a = 0$
  
  $\Rightarrow aA = Aa = 0 \Rightarrow a = 0$
  
  反之若$\theta$单,则$A$非退化。
  
  设$aA = Aa = 0 \Rightarrow \ell_a = r_a = 0 \Rightarrow (\ell_a, r_a) = (0, 0) = \theta(a)$
  
  $\theta$单$\Rightarrow a = 0$
\end{proof}

\begin{definition}
  $A$为代数,$I$为$A$的理想,称$I$为本性理想
  
  若满足:$\operatorname{ann}(I) = \{0\}$
  
  i.e. $\forall a \in A$,若$aI = \{0\} = Ia$则$a = 0$
\end{definition}

\begin{remark}
  $\circledast$ $I \triangleleft R$,$\forall J \in R$,若$I \cap J = 0$则$J = 0$($I$很大)(不等价)
\end{remark}

\begin{proposition}
  若$A$非退化,则$\theta(A)$为$M(A)$的本性理想。
\end{proposition}

\begin{proof}
  已证$\theta(A) \triangleleft M(A)$。下设$\forall (L, R) \in M(A)$且$\forall a \in A$有:
  
  $(L, R)(\ell_a, r_a) = (0, 0)$且$(\ell_a, r_a)(L, R) = (0, 0)$则可$(L, R) = (0, 0)$
  
  条件($\Leftrightarrow$) $\forall a \in A$,$\forall x \in A$,$(L \ell_a)(x) = 0$
  
  $(r_a R)(x) = 0$,$(R r_a)(x) = 0$,$(\ell_a L)(x) = 0$
  
  $\Rightarrow \forall a \in A$,$\forall x \in A$,$L(ax) = 0 \Rightarrow L(a) \cdot x = 0$
  
  $aL(x) = 0 \Rightarrow xL(a) = 0$
  
  $\Rightarrow L(c)A = AL(c) = 0 \Rightarrow \forall c$,$L(c) = 0 \Rightarrow L = 0$
  
  $R(x)a = 0$($\Leftrightarrow$),$R(a)x = 0$,$R(xa) = xR(a) = 0$
  
  $\Rightarrow R(c)A = AR(c) = 0 \Rightarrow \forall a$,$R(a) = 0 \Rightarrow R = 0$
\end{proof}

\begin{theorem}
  已知$A$非退化。
  
  \begin{equation*}
    \begin{tikzcd}
      A \arrow[r, hook] \arrow[dr, "\theta"'] & B \arrow[d, "\exists! \tau"] \\
      & M(A)
    \end{tikzcd}
  \end{equation*}
  
  任取$B$含幺代数,存在唯一的保幺同态$\tau$,使图交换。
\end{theorem}

\begin{proof}
  若$\tau_1, \tau_2$都满足条件,则要证$\forall b \in B$。
  
  有$\tau_1(b) = \tau_2(b)$,在$M(A)$中。$A$非退化,知。
  
  $\theta(A)$本性。从而只要证$\forall a \in A$,有。
  
  $\tau_1(b)\theta(a) = \tau_2(b)\theta(a)$
  
  $\theta(a)\tau_1(b) = \theta(a)\tau_2(b)$
  
  $\tau_1(b)\theta(a) = \tau_1(b)\tau_1(a) = \tau_1(ba) = \theta(ba)$
  
  $\tau_2(b)\theta(a) = \tau_2(b)\tau_2(a) = \tau_2(ba) = \theta(ba)$
  
  同理$\theta(a)\tau_1(b) = \theta(a)\tau_2(b) = \theta(ab)$
  
  进而。$A$在$B$中本性 $\Leftrightarrow$ $\tau$单。
  
  \begin{equation*}
    \begin{tikzcd}
      A \arrow[r, "\text{本性}"] \arrow[dr, "\theta"'] & B \arrow[d, "\exists! \tau"] \\
      & M(A)
    \end{tikzcd}
  \end{equation*}
  
  $M(A)$为以$A$为本性理想的最大含幺代数。
\end{proof}

\section{$*$代数}

\begin{definition}
  若$A$为一个$\mathbb{C}$代数
  
  \begin{equation*}
    \begin{aligned}
      A &\xrightarrow{*} A \quad \text{映射满是以下性质} \\
      x &\longmapsto *(x) \triangleq x^*
    \end{aligned}
  \end{equation*}
  
  \begin{enumerate}
    \item $\forall x \in A$有$x^{**} = x$($x$为时全)
    \item $*$为共轭线性。i.e. $\forall x, y \in A$,有$(x+y)^* = x^* + y^*$,$\forall \lambda \in \mathbb{C}$,$(\lambda x)^* = \bar{\lambda}x^*$
    \item $*$为反同态。i.e. $\forall x, y \in A$,有$(xy)^* = y^*x^*$
  \end{enumerate}
\end{definition}

\begin{definition}
  若$A$为复代数,$*$为$A$上的一个星运算
  
  称$(A, *)$为一个$*$代数。
\end{definition}

\begin{definition}
  $A, B$为$*$代数,$A \xrightarrow{f} B$映射。
  
  称$f$为一个$*$同态,若$f$为一个代数同态,且$\forall x \in A$,有$f(x^*) = f(x)^*$
\end{definition}

\begin{definition}
  $A$为一个$*$代数,$I \triangleleft A$。若满足、$\forall x \in I$,有$x^* \in I$。
  
  称$I$为一个$*$理想,($I^* = I$)
\end{definition}

\begin{proposition}
  若$A \xrightarrow{f} B$为$*$同态。则$\ker f$为$A$的$*$理想。
\end{proposition}

\begin{proof}
  已设$\ker f$理想。$\forall x \in \ker f$,$f(x) = f(x) = 0 \Rightarrow x^* \in \ker f$。
\end{proof}

\begin{proposition}
  若$I$为$A$的$*$理想。则$A/I$上可良定义$*$运算
  
  $\forall x \in A$,$[x]^* = [x^*]$并且$(A/I, *)$为$*$代数
\end{proposition}

\begin{proof}
  若$[x] = [y]$,$\Rightarrow x - y \in I$,$\Rightarrow (x - y)^* \in I \Rightarrow [x^*] = [y^*]$
  
  下证
  \begin{equation*}
    \begin{aligned}
      A/I &\xrightarrow{*} A/I \text{ 是一个$*$运算} \\
      [x] &\longmapsto [x]^*
    \end{aligned}
  \end{equation*}
  
  \begin{enumerate}
    \item $[x]^{**} = [x^*]^* = [x^{**}] = [x]$
    \item $([x] + [y])^* = [x + y]^* = [(x + y)^*] = [x^* + y^*] = [x]^* + [y]^*$
    \item $([x][y])^* = [xy]^* = [(xy)^*] = [y^*x^*] = [y]^*[x]^*$
  \end{enumerate}
  
  $I$为$A$的$*$理想,
  \begin{equation*}
    \begin{aligned}
      A &\xrightarrow{\pi} A/I \text{ 为$*$同态} \\
      x &\longmapsto [x]
    \end{aligned}
  \end{equation*}
  
  $\pi(x^*) = [x^*] = [x]^* = \pi(x)^*$,$\ker \pi = I$
\end{proof}

\begin{theorem}[同态基本定理]
  \begin{equation*}
    \begin{tikzcd}
      A \arrow[r, "f"] \arrow[d, "\pi"'] & B \quad *\text{同态} \\
      A/\ker f \arrow[ur, "\exists! \bar{f}"'] &
    \end{tikzcd}
  \end{equation*}
\end{theorem}

\begin{proof}
  $\bar{f}([x]^*) = \bar{f}([x^*]) = f(x^*) = f(x)^* = \bar{f}([x])^*$
\end{proof}

\begin{proposition}
  $A$含$1$,$\langle \Omega ,1\rangle_{*-al} = \Omega$生成的含$1$ $*$子代数。
\end{proposition}

\begin{proposition}
  若$\Omega^* \subset \Omega$,则$\langle \Omega \rangle = \langle \Omega \rangle_*$,$\langle \Omega \rangle_{al} = \langle \Omega \rangle_{*-al}$。
\end{proposition}

\begin{proof}
  $\circled{1}$ $\langle \Omega \rangle \subset \langle \Omega \rangle_*$。
  
  下证$\langle \Omega \rangle \supset \langle \Omega \rangle_*$,只要证$\langle \Omega \rangle$为一个$*$理想。从而
  $\langle \Omega \rangle \supset \langle \Omega \rangle_*$
  
  $\forall x \in \langle \Omega \rangle$,要证$x^* \in \langle \Omega \rangle$,定义$B \triangleq \{x \in A : x^* \in \langle \Omega \rangle\}$
  
  易知$\langle \Omega \rangle \subset B$。下证$B$为一个理想,从而$\langle \Omega \rangle \subset B$。
  
  i.e. $\forall x \in \langle \Omega \rangle$,$x^* \in \langle \Omega \rangle$。
  
  这是因为。$0 \in B$,$x, y \in B \Rightarrow x^*, y^* \in \langle \Omega \rangle$
  
  $\Rightarrow (x + y)^* = x^* + y^* \in \langle \Omega \rangle \Rightarrow x + y \in B$。
  
  $(\lambda x)^* = \bar{\lambda}x^* \in \langle \Omega \rangle \Rightarrow \lambda x \in B$。
  
  $\forall x \in B$,$\forall a \in A$,有$(ax)^* = x^*a^* \in \langle \Omega \rangle \Rightarrow ax \in B$。
  
  故$B$为一个理想,$\Rightarrow \langle \Omega \rangle \subset B$。
  
  $\circled{2}$ $\langle \Omega \rangle_{al} \subset \langle \Omega \rangle_{*-al}$
  
  下证$\langle \Omega \rangle_{al} \supset \langle \Omega \rangle_{*-al}$,只要证$\langle \Omega \rangle_{al}$为*代数。
  
  从而$\langle \Omega \rangle_{al} \supset \langle \Omega \rangle_{*-al}$。
  
  令$B = \{\bar{x} : x \in \langle \Omega \rangle_{al}\}$。下证$B$为代数
  
  类似证明
\end{proof}

\section{$*$代数幺元化}

\begin{definition}
  $A$为$*$代数,$A \xrightarrow{\theta} B$为$*$同态。
  
  $B$为含幺$*$代数,若满足
  
  \begin{equation*}
    \begin{tikzcd}
      A \arrow[r, "\theta"] \arrow[dr, "f"'] & B \arrow[d, "\exists! \tilde{f}"', "\text{保幺同态}"] \\
      & C \text{ 含幺$*$代数}
    \end{tikzcd}
  \end{equation*}
\end{definition}

\begin{proposition}
  $A$为$*$代数幺元化存在唯一
\end{proposition}

\begin{proof}
  $A \xrightarrow{\theta} A \oplus \mathbb{C}$代数幺元化。其中$(a, \lambda)(b, \mu) \triangleq (ab + \lambda b + \mu a, \lambda\mu)$
  
  在$A \oplus \mathbb{C}$中定义$*$运算。
  
  \begin{proposition}
    $1^* = 1$,因为$1^*x = 1^*x^{**} = (x^*1)^* = x^{**} = x \Rightarrow 1^* = 1$
    
    $\forall (a, \lambda) \in A \oplus \mathbb{C}$,$(a, \lambda)^* \triangleq (a^*, \bar{\lambda})$
    
    \begin{enumerate}
      \item $(a, \lambda)^{**} = (a^{**}, \overline{\bar{\lambda}}) = (a, \lambda)$
      \item $(r(a, \lambda))^* = (ra, r\lambda)^* = ((ra)^*, \overline{r\lambda}) = (ra^*, \bar{r}\bar{\lambda}) = \bar{r}(a, \lambda)^*$
      \item $((a, \lambda)(b, \mu))^* = (ab + \lambda b + \mu a, \lambda\mu)^* = (b^*a^* + \bar{\mu}a^* + \bar{\lambda}b^*, \bar{\lambda}\bar{\mu})$
      
      $(b, \mu)^*(a, \lambda)^* = (b^*, \bar{\mu})(a^*, \bar{\lambda}) = (b^*a^* + \bar{\mu}a^* + \bar{\lambda}b^*, \bar{\lambda}\bar{\mu})$
    \end{enumerate}
    
    $\theta(a^*) = (a^*, 0) = (a, 0)^* = \theta(a)^*$
  \end{proposition}
  
  \begin{equation*}
    \begin{tikzcd}
      A \arrow[r, "\theta"] \arrow[dr, "f"', "*\text{同态}"] & A \oplus \mathbb{C} \arrow[d, "\exists! \tilde{f}"', "\text{保幺同态}"] \\
      & B \text{ 含幺$*$代数}
    \end{tikzcd}
  \end{equation*}
  
  下证$\tilde{f}$保$*$。
  
  $\tilde{f}((a, \lambda)^*) = \tilde{f}((a^*, \bar{\lambda})) = f(a^*) + \bar{\lambda}1_B$
  
  $\tilde{f}((a, \lambda))^* = (f(a) + \lambda 1_B)^* = f(a)^* + \bar{\lambda}1_B$
\end{proof}

\begin{proposition}
  $A \xrightarrow{\theta} M(A)$,$A$为$*$代数。
  
  $a \longmapsto (\ell_a, r_a)$。
  
  则在$M(A)$上可定义$*$运算如下:$(L, R)^* = (R^*, L^*)$
  
  其中$L^*(x) \triangleq L(x^*)^*$,$R^*(x) = R(x^*)^*$,要证$*$为$M(A)$上的一个$*$运算。
\end{proposition}

\begin{example}
  $\mathbb{C}$为一个$*$代数,$*$运算为共轭运算。
  
  $\forall \lambda \in \mathbb{C}$,$\lambda^* \triangleq \bar{\lambda}$
  
  \begin{enumerate}
    \item $\bar{\bar{\lambda}} = \lambda$
    \item $\bar{a} + \bar{b} = \overline{a + b}=a^*+b^*$,$(\lambda a)^* = \bar{\lambda}a = \bar{\lambda}\bar{a} = \bar{\lambda}a^*$
    \item $\overline{ab} = \bar{a}\bar{b} = \bar{b}\bar{a}$
  \end{enumerate}
\end{example}

\begin{example}
  $M_n(\mathbb{C})$,$*$运算为$A^* = \bar{A}^T$
  
  \begin{enumerate}
    \item $A^{**} = A$
    \item $(A + B)^* = A^* + B^*$,$(\lambda A)^* = \bar{\lambda}A^*$
    \item $(AB)^* = \overline{AB}^T = \overline{AB}^T = B^T\bar{A}^T = B^*A^*$
  \end{enumerate}
\end{example}

\begin{definition}
  $H_1, H_2$为Hilbert空间,$\mathcal{B}(H_1, H_2)$表示$H_1$到$H_2$的全体有界线性算子。
\end{definition}

\begin{proposition}
  若$H_1 \xrightarrow{T} H_2$,$T \in \mathcal{B}(H_1, H_2)$则存在唯一
  \begin{equation*}
    \begin{aligned}
      H_2 &\xrightarrow{T^*} H_1,T^* \in \mathcal{B}(H_2, H_1)
    \end{aligned}
  \end{equation*}
  使得$\forall x \in H_1$,$\forall y \in H_2$,有$\langle Tx, y \rangle_{H_2} = \langle x, T^*y \rangle_{H_1}$
\end{proposition}

\begin{proof}
  若$H_2 \underset{B}{\overset{A}{\rightleftarrows}} H_1$ s.t. $\langle Tx, y \rangle = \langle x, Ay \rangle = \langle x, By \rangle$
  
  $\langle x, (A-B)y \rangle = 0$,$\forall y \Rightarrow A - B = 0$
\end{proof}

\begin{proposition}
  $T^{**} = T$
\end{proposition}

\begin{theorem}[Riesz表示定理]
  给定Hilbert空间$H$,有
  \begin{equation*}
    \begin{aligned}
      H &\xrightarrow{\theta} H^* \triangleq \mathcal{B}(H, \mathbb{C}) \\
      x &\longmapsto \langle x, x\rangle
    \end{aligned}
  \end{equation*}
  为保范共轭线性同构
\end{theorem}

\begin{proof}
  $\circled{1}$ $x = 0$,$\langle x, x \rangle = 0 \Rightarrow ||\langle x, x \rangle|| = 0 = ||x||$
  
  $\circled{2}$ $x \neq 0$,$||x|| \neq 0$,$\langle \frac{x}{||x||}, x \rangle = ||x||$,$\Rightarrow ||\langle x, x \rangle|| = ||x||$
  
  $\lambda x \mapsto \langle x, \lambda x \rangle = \bar{\lambda}\langle x, x \rangle = \bar{\lambda}\theta(\lambda)$
  
  共轭线性$\Rightarrow$单
  
  (存在)$\forall y \in H_2$,定义
  \begin{equation*}
    \begin{aligned}
      H_1 &\xrightarrow{\langle T*, y \rangle} \mathbb{C} \\
      x &\longmapsto \langle Tx, y \rangle
    \end{aligned}
  \end{equation*}
  
  $|\langle Tx, y \rangle| \leq ||Tx|| ||y|| \leq ||T|| ||x|| ||y||$
 $\Rightarrow ||\langle T x, y \rangle|| \leq ||T|| ||y||$。i.e. $\langle T x, y \rangle \in H_1^*$
  
  由Riesz表示定理,$\exists z \in H_1$,s.t.$\langle T x, y \rangle = \langle x, z \rangle$,记$z = T^*y$
  
  定义
  \begin{equation*}
    \begin{aligned}
      T^*: H_2 &\longrightarrow H_1 \\
      y &\longmapsto T^*y
    \end{aligned}
  \end{equation*}
  
  易知,$\forall x, y$,有$\langle Tx, y \rangle = \langle x, T^*y \rangle$
  
  下证$T^*$线性。$\forall x$,$\forall y_1, y_2$。
  
  $\langle T x, y_1 + y_2 \rangle = \langle x, T^*(y_1 + y_2) \rangle$
   $=$

  $\langle Tx, y_1 \rangle + \langle Tx, y_2 \rangle = \langle x, T^*y_1 \rangle + \langle x, T^*y_2 \rangle$
   $= \langle x, T^*y_1 + T^*y_2 \rangle$
   $\Rightarrow T^*(y_1 + y_2) = T^*y_1 + T^*y_2$(由正定)
  
  $\langle Tx, \lambda y \rangle = \langle x, T^*(\lambda y) \rangle$
  
  $=$
  $\bar{\lambda}\langle Tx, y \rangle = \bar{\lambda}\langle x, T^*y \rangle = \langle x, \lambda T^*y \rangle$
  $\Rightarrow T^*(\lambda y) = \lambda T^*y$
  
  由Riesz表示,$||T^*y|| = ||\langle T x, y \rangle||$
  
  对$y$,$||T^*y|| \leq ||T|| ||y|| \Rightarrow ||T^*|| \leq ||T||$
  
  $T^* \in \mathcal{B}(H_2, H_1)$
  
  $||T|| = ||T^{**}|| \leq ||T^*|| \Rightarrow ||T^*|| \leq ||T||$
\end{proof}

\begin{proposition}
  \begin{enumerate}
    \item $T^{**} = T$
    \item $(T + S)^* = T^* + S^*$,且$(\lambda T)^* = \bar{\lambda}T^*$
    \item $(T \circ S)^* = S^* \circ T^*$
  \end{enumerate}
  
  \begin{equation*}
    \begin{tikzcd}[column sep=large, row sep=large]
      H_1 \arrow[rr, "S", bend left=20] \arrow[dr, "TS"'] & & H_2 \arrow[ll, "S^*", bend left=20] \arrow[dl, "T"] \\
      & H_3 \arrow[ul, "S^*T^*", bend left=20] \arrow[ur, "T^*"', bend left=20] &
    \end{tikzcd}
  \end{equation*}
\end{proposition}

\begin{proposition}
  $\mathcal{B}(H)$为一个$\mathbb{C}-*$代数,为$\operatorname{End}_\mathbb{C}(H)$的含幺子代数
\end{proposition}

\begin{proposition}
  $\forall T \in \mathcal{B}(H)$有$||T^*T|| = ||TT^*|| = ||T||^2$
\end{proposition}

\begin{proposition}
  若$A$交换非退化代数,则$M(A)$交换。
\end{proposition}

\begin{proof}
  $\forall (L, R), (L', R') \in M(A)$,要证$(L, R)(L', R') = (L', R')(L, R)$
  i.e. $(LL', R'R) = (L'L, RR')$
  
  $\forall x, y \in A$,$L(L'(x))y = L'(L(x))y$($A$非退化)
  
  $L(L'(x))y = yL(L'(x)) = R(y)L'(x) = L'(xR(y)) = L'(xR(y))$
  $= L'(R(y)x) = L'(yL(x)) = L'(L(x))y$
\end{proof}

\begin{proposition}
  若$A$非退化,$(L, R), (L_1, R_2) \in M(A)$,则$R_1 = R_2$
\end{proposition}

\begin{proof}
  $\forall x, y \in A$,有$R_1(x)y = xL(y) = R_2(x)y$(非退化)$\Rightarrow R_1 = R_2$
\end{proof}

\begin{lemma}
  已知$I$为含幺代数$B$的本性理想,若$I$交换,则$B$交换。
\end{lemma}

\begin{proof}
  由$I$在$B$中$*$性理想,知$I$非退化代数。
  
  \begin{equation*}
    \begin{tikzcd}
      I \arrow[r, hook] \arrow[dr, "\theta"'] & B \arrow[d, "\exists! \tau"', "\text{保幺单同态}"] \\
      & M(I)
    \end{tikzcd}
  \end{equation*}
  
  ($\forall x \in I$,若$xI = 0 = Ix$,$I$本性$\Rightarrow x = 0$)
  
  有$M(I)$交换$\Rightarrow B$交换
  
  $\forall a, b \in B$,要证$ab = ba$,由$I$本性,只要证$(ab - ba)I = 0 = I(ab - ba)$
  
  $\forall x \in I$,有$(ab - ba)x = 0 =x(ab - ba)$
  
  只要证$\forall x, y \in I$,有$(ab - ba)xy = 0 = y(ab - ba)x$($I$本性)
  
  W.S. $x(ab - ba)y = 0 = yx(ab - ba)$
  
  $\circled{1}$ $abxy = baxy$,$abxy = a(bx)y = ay \cdot bx = bx \cdot ay = ba \cdot xy$
\end{proof}


  全体Hilbert空间之间的有界线性算子满足
  \begin{enumerate}
    \item $T^{**} = T$(对合)$H \underset{T^*}{\overset{T}{\rightleftarrows}} H^*$
    \item $(T + S)^* = T^* + S^*$(共轭线性)
    \item $(T \circ S)^* = S^* \circ T^*$
  \end{enumerate}


\begin{proof}
  $\circled{2}$ $\langle (T + S)x, y \rangle = \langle x, (T + S)^*y \rangle$
  
  $\langle Tx, y \rangle + \langle Sx, y \rangle = \langle x, T^*y + S^*y \rangle$
  
  $\circled{3}$ $\langle T \circ Sx, y \rangle = \langle x, (T \circ S)^*y \rangle$
  
  $\langle Sx, T^*y \rangle = \langle x, S^*T^*y \rangle$
\end{proof}

\begin{example}
  $H$为一个Hilbert空间,$\mathcal{B}(H)$关于点态加、点态数乘、复合$*$运算为一个$*$代数。
\end{example}

\begin{example}
  X赋范空间,$\mathcal{B}(X)$为代数$\Rightarrow$赋范代数。
\end{example}

\begin{definition}
  若$A$为一个$F$代数,$\|\cdot\|$为$A$的一个代数,若满足$\forall x, y \in A$有$\|xy\| \leq \|x\| \|y\|$,则称$(A, \|\cdot\|)$为一个赋范代数。
\end{definition}

\begin{definition}
  若$(A, \|\cdot\|)$为一个赋范代数且为Banach空间,称$(A, \|\cdot\|)$为一个Banach代数。
  
  (Banach空间:X,$(\mathcal{B}(X), \|\cdot\|)$是完备的,$\mathcal{B}(X)$为Banach空间)
\end{definition}

\begin{proposition}
  $H_1 \underset{T^*}{\overset{T}{\rightleftarrows}} H_2$,$\|T^*T\| = \|T\|^2$
\end{proposition}

\begin{proof}
  $\|T^*T\| \leq \|T^*\| \|T\| = \|T\|^2$
  
  下证$\|T\|^2 \leq \|T^*T\|$
  
  $\forall x \in H_1$,$\|x\| \leq 1$,$\|Tx\|^2 = \langle Tx, Tx \rangle = |\langle x, T^*Tx \rangle|$
  $\leq \|x\| \|T^*Tx\| \leq \|T^*Tx\|$
  
  $\Rightarrow \|T\| \leq \|T^*T\|^{1/2} \Rightarrow \|T\|^2 \leq \|T^*T\|$
\end{proof}

\begin{definition}
  若$(A, \|\cdot\|)$为赋范代数,$A$为$*$代数,满足:$\forall x \in A$,$\|x^*x\| = \|x\|^2$,称$\|\cdot\|$满足$C^*$等式。
  
  进一步,若$(A, \|\cdot\|)$完备,称$(A, \|\cdot\|)$为一个$C^*$代数。
  (i.e. $A$为Banach代数,有$*$,$C^*$等式)
\end{definition}

\begin{example}
  $H$为Hilbert空间,$(\mathcal{B}(H), \|\cdot\|, *)$为一个$C^*$代数。
  $A \leq \mathcal{B}(H)$闭$*$子代数亦为$C^*$代数。
\end{example}

\begin{itemize}
  \item $(A_\alpha)_\alpha$为一族$F$代数
  
  在$\prod A_\alpha$定义加、数乘、求法,使$\prod A_\alpha$亦为$F$代数。
  
  $(x_\alpha)_\alpha + (y_\alpha)_\alpha \triangleq (x_\alpha + y_\alpha)_\alpha$
  
  $\lambda(x_\alpha)_\alpha \triangleq (\lambda x_\alpha)_\alpha$
  
  $(x_\alpha)_\alpha \cdot (y_\alpha)_\alpha \triangleq (x_\alpha y_\alpha)_\alpha$
\end{itemize}

\begin{definition}
  $(A_\alpha)_\alpha$为一族$*$代数,$\prod A_\alpha$为直积,
  定义典范的$*$运算。$\forall (x_\alpha)_\alpha \in \prod A_\alpha$,$(x_\alpha)_\alpha^* \triangleq (x_\alpha^*)_\alpha$
  
  容易验证。
\end{definition}

\begin{definition}
  $\bigoplus\limits_\alpha^{\ell^\infty} A_\alpha \triangleq \{(x_\alpha)_\alpha \in \prod A_\alpha : \{\|x_\alpha\| : \alpha \in \Lambda\} \text{有界}\}$
\end{definition}

\begin{proposition}
  $\bigoplus\limits_\alpha^{\ell^\infty} A_\alpha$为$\prod A_\alpha$的子代数。
\end{proposition}

\begin{proof}
  $\forall \alpha$,$\|x_\alpha + y_\alpha\| \leq \|x_\alpha\| + \|y_\alpha\| \leq M_1 + M_2$
   \quad $\|x_\alpha \cdot y_\alpha\| \leq \|x_\alpha\| \|y_\alpha\| \leq M_1 M_2$
\end{proof}

\begin{remark}
  $\bigoplus\limits_\alpha A_\alpha \leq \bigoplus\limits_\alpha^{\ell^\infty} A_\alpha$
\end{remark}

\begin{definition}
  $(\bigoplus\limits_\alpha A_\alpha, \|\cdot\|_\infty)$为一个赋范代数。
  $\|\cdot\|_\infty := \sup\limits_\alpha \|x_\alpha\|$
\end{definition}

\begin{proof}
  $\|(x_\alpha)_\alpha\|_\infty = 0 \Rightarrow \forall \alpha, \|x_\alpha\| = 0 \Rightarrow x_\alpha = 0 \Rightarrow (x_\alpha)_\alpha = 0$
  
  $\|\lambda(x_\alpha)_\alpha\|_\infty = \sup\limits_\alpha \|\lambda x_\alpha\| = |\lambda| \sup\limits_\alpha \|x_\alpha\| = |\lambda| \|(x_\alpha)_\alpha\|_\infty$
  
  $\|(x_\alpha)_\alpha + (y_\alpha)_\alpha\|_\infty = \sup\limits_\alpha \|x_\alpha + y_\alpha\| \leq \sup\limits_\alpha (\|x_\alpha\| + \|y_\alpha\|) \leq \sup\limits_\alpha \|x_\alpha\| + \sup\limits_\alpha \|y_\alpha\|$
  $\leq \|(x_\alpha)_\alpha\|_\infty + \|(y_\alpha)_\alpha\|_\infty$
\end{proof}

\begin{example}
  \begin{equation*}
    \begin{aligned}
      \ell^\infty(\Omega) &= \bigoplus\limits_{\omega \in \Omega} \mathbb{C} \\
      f &\longleftrightarrow (x_\alpha)_{\alpha \in \Omega}
    \end{aligned}
  \end{equation*}
\end{example}

\begin{lemma}
  若$(A_\alpha)_\alpha$为一族Banach代数,则$\bigoplus\limits_\alpha^{\ell^\infty} A_\alpha$为Banach代数。(取Cauchy列)
\end{lemma}

\begin{proposition}
  若$A$为赋范代数,且为$*$代数,满足:
   $\forall x \in A$,有$\|x\|^2 \leq \|x^*x\|$,则满足$C^*$等式。
\end{proposition}

\begin{proof}
  $\forall x \in A$,有$\|x\|^2 \leq \|x^*x\| \leq \|x^*\| \|x\|$
  $\Rightarrow \|x\| \leq \|x^*\|$,$\forall x$,$\Rightarrow \|x^*\| \leq \|x^{**}\| = \|x\|$
  $\Rightarrow \|x\| = \|x^*\|$
  
  $\|x^*x\| \leq \|x^*\| \|x\| = \|x\|^2$
\end{proof}

\begin{proposition}
  若$(A_\alpha)_\alpha$为一族$C^*$代数,
   则$(\bigoplus\limits_\alpha^{\ell^\infty} A_\alpha, \|\cdot\|_\infty)$为一个$C^*$代数。
\end{proposition}

\begin{proof}
  $\|(x_\alpha)_\alpha\|_\infty^2 = (\sup\limits_\alpha \|x_\alpha\|)^2 = \sup\limits_\alpha \|x_\alpha\|^2 = \sup\limits_\alpha \|x_\alpha^* x_\alpha\| = \|((x_\alpha^*)_\alpha (x_\alpha)_\alpha)\|_\infty$
\end{proof}

\begin{example}
  $A, B$为$C^*$代数,$(A \oplus B, \|\cdot\|_\infty)$为$C^*$代数。
\end{example}

\begin{proposition}
  $A$为一非零含幺$C^*$代数,则$\|1\| = 1$
\end{proposition}

\begin{proof}
  $\|1\|^2 = \|1^*1\| = \|1 \cdot 1\| = \|1\| \Rightarrow \|1\| = 1$
\end{proof}

\section{$C^*$代数幺元化}

$A$为$C^*$代数,$A$是否存在幺元化

$C^*$代数$A \xrightarrow{\theta} \tilde{A}$含幺$C^*$代数

\begin{equation*}
  \begin{tikzcd}
    A \arrow[r, "\theta"] \arrow[dr, "\forall \varphi"'] & \tilde{A} \arrow[d, "\exists!"', "\text{保幺$*$同态} \tilde{\varphi}"] \\
    & B \text{ $\forall$ 含幺$C^*$代数}
  \end{tikzcd}
\end{equation*}

$A$为Banach代数(或赋范代数),$A$是否存在幺元化

$A \longrightarrow \tilde{A}$ 含幺Banach alg

\begin{equation*}
  \begin{tikzcd}
    A \arrow[r] \arrow[dr, "\varphi"'] & \tilde{A} \arrow[d, "\exists! \tilde{\varphi}"', "\text{保有界、保幺}"] \\
    & B \text{ $\forall B$ Banach alg}
  \end{tikzcd}
\end{equation*}

$A$为赋范代数($C$-代数)

定义$\tilde{A} \triangleq A \oplus \mathbb{C}$,$(a, \lambda)(b, \mu) = (ab + \lambda b + \mu a, \lambda\mu)$

$\|(a, \lambda)\|_1 \triangleq \|a\| + |\lambda|$,$A \oplus \mathbb{C}$为1-范数

下证$(\tilde{A}, \|\cdot\|_1)$为赋范代数。

$\|(a, \lambda)(b, \mu)\|_1 = \|(ab + \lambda b + \mu a, \lambda\mu)\| = \|ab + \lambda b + \mu a\| + |\lambda\mu|$
$\leq \|a\| \|b\| + |\lambda| \|b\| + |\mu| \|a\| + |\lambda| |\mu|$

$\|(a, \lambda)\|_1 \|(b, \mu)\|_1 = (\|a\| + |\lambda|)(\|b\| + |\mu|)$ 等于上面

\begin{proposition}
  $A \xrightarrow{\theta} \tilde{A}$($A$视为$\tilde{A}$的理想)
  
  \begin{equation*}
    \begin{tikzcd}
      A \arrow[r, "\theta"] \arrow[dr, "\forall \varphi" description, "\text{连续同态}"'] & \tilde{A} \arrow[d, "\exists! \tilde{\varphi}"', "\text{保幺同态}"] \\
      & B
    \end{tikzcd}
  \end{equation*}
  
  $\|\theta(a)\| = \|(a, 0)\| = \|a\|$,等距嵌入
\end{proposition}

\begin{proof}
  下证$\tilde{\varphi}$是连续的$\Leftrightarrow \tilde{\varphi}$有界
  
  $\tilde{\varphi}((a, \lambda)) = \varphi(a) + \lambda 1_B$
  \quad $\|\tilde{\varphi}((a, \lambda))\| = \|\varphi(a) + \lambda 1_B\| \leq \|\varphi(a)\| + |\lambda| \|1_B\| \leq \|\varphi\| \|a\| + |\lambda| \|1_B\|$
  
  记$M = \max\{\|\varphi\|, \|1_B\|\} \Rightarrow \leq M(\|a\| + |\lambda|) = M\|(a, \lambda)\|_1$
\end{proof}

进一步,若$A$为Banach代数,则$(\tilde{A}, \|\cdot\|_1)$为Banach代数。

\begin{remark}
      $(A \oplus \mathbb{C}, \|\cdot\|_1)$没有$C^*$等式。
\end{remark}
\begin{remark}
  Banach代数范畴下态射是连续映射。
\end{remark}

\begin{proposition}
  $C^*$代数$A$非退化。
\end{proposition}

\begin{proof}
  $\forall \lambda \in A$,若$xA = 0 = Ax$,要证$x= 0$。
  
  $\Rightarrow xx^* = 0$。$\Rightarrow \|x\|^2 = \|xx^*\| = 0 \Rightarrow \|x\| = 0 \Rightarrow x = 0$。
\end{proof}

\subsection{$A$为含幺$(e)$的$C^*$代数}



  
  $A \xrightarrow{\theta} \tilde{A} \cong A \oplus \mathbb{C}$。
  
  定义
  \begin{equation*}
    \begin{aligned}
      \tilde{A} &\xrightarrow{\sigma} A \oplus \mathbb{C} \\
      (a, \lambda) &\longmapsto (a + \lambda e, \lambda)
    \end{aligned}
  \end{equation*}
  
  直接验证$\sigma$为同构。
  
  定义$\tilde{A}$上$\|\cdot\|$,$\|(a, \lambda)\| \triangleq \max\{\|a + \lambda e\|, |\lambda|\}$
  从而$\|\cdot\|$在$\tilde{A}$中满足$C^*$等式
  
  思路:
  \begin{equation*}
    \begin{tikzcd}[column sep=large, row sep=large]
      a \arrow[d] & A \arrow[r, "\theta"] \arrow[dr, "\eta"] & \tilde{A} \arrow[d, "\exists! \sigma"', "\text{保幺$*$同态,图交换}"] \\
      {(a, 0)} & & A \oplus \mathbb{C} \text{ $C^*$代数直和,有幺$(e, 1)$}
    \end{tikzcd}
  \end{equation*}
  
  $\sigma((a, \lambda)) = \eta(a) + \lambda 1_{A \oplus \mathbb{C}} = (a, 0) + \lambda(e, 1) = (a + \lambda e, \lambda)$。

\begin{proof}
  下证$\sigma$为双射。
  
  $\sigma$单:$(a, \lambda) \longmapsto (a + \lambda e, \lambda) = 0 \Rightarrow \lambda = 0, a + \lambda e = 0 \Rightarrow a = 0$
  
  $\Rightarrow (a, \lambda) = 0$。$\sigma$单。
  
  $\sigma$满:$\sigma(A \times \{0\}) = A \times \{0\}$。$\sigma(\{0\} \times \mathbb{C}) \supset \{0\} \times \mathbb{C}$。
  
  $\Rightarrow \sigma(\tilde{A}) \supset A \times \{0\} \cup \{0\} \times \mathbb{C}$。
  
  $\Rightarrow \sigma(\tilde{A}) \supset A \oplus \mathbb{C}$。故$\sigma$满,
  
  $\Rightarrow \sigma$为$*$同构,把$A \oplus \mathbb{C}$的范数拉回,
  
  即为$\tilde{A}$满足$C^*$等式的范数。
\end{proof}

\begin{proposition}
  $F$为无限域,$V$为$F$-代数
  若$\varphi$为$V \longrightarrow V$,线性映射满足任取$A, B$,
  
  有$\varphi(AB) = \varphi(A)\varphi(B)$或$\varphi(AB) = \varphi(B)\varphi(A)$,则$\varphi$为$V \longrightarrow V$乘法同态或反同态。
\end{proposition}

\begin{proof}
  任取$B$,$V_B \triangleq \{A : \varphi(AB) = \varphi(A)\varphi(B)\}$
   \quad $V_{B_2} \triangleq \{A : \varphi(AB) = \varphi(B)\varphi(A)\}$
  
  下证$V_B$、$V_{B_2}$为线性$Sp$。
  
  $\forall x,y \in \mathbb{F}, \exists V_B$,$\varphi[(x + y)B] = \varphi(xB) + \varphi(yB) = \varphi(x)\varphi(B) + \varphi(y)\varphi(B)$
  $= \varphi(x + y)\varphi(B)$
   $\Rightarrow x + y \in V_B$。
  
  $A \in V_B$,$\varphi(AB) = \lambda\varphi(A)\varphi(B) = \varphi(\lambda A)B$。$\Rightarrow \lambda A \in V_B$。
  
  故$V_B$为线性空间,同理$V_{B_2}$为线性空间
  
  $V = V_B \cup V_{B_2}$,由$F$无限域,则有填不满定理
  $\Rightarrow V_B = V$或$V_{B_2} = V$。
  
  记$V_1 = \{B : V_B = V\}$,$V_2 = \{B : V_{B_2} = V\}$
  
  下证$V_1$,$V_2$为线性$Sp$。
  
  $\forall B \in V_1$,要证$\lambda B \in V_1$,$\forall \lambda \in \mathbb{F}$,$\forall A \in V$,$\varphi(A\lambda B) = \lambda\varphi(A)\varphi(B)$
   $= \varphi(A)\varphi(\lambda B) \Rightarrow \lambda B \in V_1$。
  
  $\forall \lambda, \gamma \in V_1$,$\forall A \in V$,$\varphi(A(\lambda + \gamma)) = \varphi(A\lambda) + \varphi(A\gamma)$
  $= \varphi(A)\varphi(\lambda) + \varphi(A)\varphi(\gamma) = \varphi(A)\varphi(\lambda + \gamma) \Rightarrow \lambda + \gamma \in V_1$
  
  故$V_1$为线性$Sp$,同理$V_2$为线性$Sp$。
  
  $V = V_1 \cup V_2$,由填不满定理,$V_1 = V$或$V_2 = V$
\end{proof}

\subsection{ $A$为$C^*$代数且不含幺}

$A \xrightarrow{\theta} M(A)$,$*$代数/含幺代数

$\Rightarrow \begin{cases}
  aA = 0 \Rightarrow a = 0 \\
  Aa = 0 \Rightarrow a = 0 \\
  a^*a = 0 \Rightarrow a = 0
\end{cases}$

($C^*$代数性退化)

接下来在$M(A)$上定义$*$

\begin{itemize}
  \item $A$为$C^*$代数,$\varphi \in \operatorname{End}_{\mathbb{C}}(A)$ 定义$\varphi^*: A \longrightarrow A$,$\varphi^*(x) \triangleq \varphi(x^*)^*$
  
  $\varphi^*$线性。$\varphi^*: x \longmapsto x^* \xrightarrow{\varphi} \varphi(x^*) \longmapsto \varphi(x^*)^*$。
\end{itemize}

\begin{proposition}
  任取$(L, R) \in M(A)$,$(R^*, L^*) \in M(A)$。
\end{proposition}

\begin{proof}
  $\forall x, y \in A$。
  
  \begin{enumerate}
    \item $R^*(xy) = R(x^*)^*y = (y^*R(x^*))^* = (Ry^*x^*)^* = R^*(xy)$
    
    \item $xL^*(y) = L^*(xy)$。(自行验证)
    
    \item $L^*(x)y = xR^*(y)$
    
    $L^{(x)}y = L(x^*)^*y = (y^*L(x^*))^* = (Ry^*)x^*)^*$
    $= xRy^*)^* = xR^*(y)$
  \end{enumerate}
\end{proof}

\begin{proposition}
  若$A$为一个$*$代数,则
  \begin{equation*}
    \begin{aligned}
      M(A) &\xrightarrow{*} M(A) \\
      (L, R) &\longmapsto (R^*, L^*)
    \end{aligned}\text{ 为一个星运算} 
  \end{equation*}
\end{proposition}

\begin{proof}
  $\varphi^* = * \cdot \varphi \cdot *$,$(\varphi^*)^* = * \cdot \varphi^* \cdot * = * \cdot * \cdot \varphi \cdot * \cdot * = \varphi$
 
  
  \begin{enumerate}
    \item $(L, R)^* = (R^*, L^*)^* = (L^{**}, R^{**}) = (L, R)$
    
    \item $((L_1, R_1) + (L_2, R_2))^* = ((R_1 + R_2)^*, (L_1 + L_2)^*)$。
    $\stackrel{?}{=} (L_1, R_1)^* + (L_2, R_2)^* = (R_1^*, L_1^*) + (R_2^*, L_2^*)$
    
    由$(\varphi + \psi)^* = x \cdot (\varphi + \psi) \cdot x = x \cdot \varphi \cdot x + x \cdot \psi \cdot x = \varphi^* + \psi^*$
    
    验证 $\circled{1}$  $\circled{2}$ 两式相等
    
    $(\lambda(L, R))^* = (\lambda L, \lambda R)^* = ((\lambda R)^*, (\lambda L)^*)$
     $\stackrel{?}{=} \overline{\lambda}(L, R)^* = (\overline{\lambda}R^*, \overline{\lambda}L^*)$
    
    \item $(\lambda\varphi)^*(x) = (\lambda\varphi)(x^*)^* = \overline{\lambda}\varphi(x^*)^* = \overline{\lambda}\varphi^*(x)$
    

  \end{enumerate}
  
  反同态$((L_1, R)(L_2, R_2))^* = (L_1L_2, R_2R_1)^*$
  $= ((R_2R_1)^*, (L_1L_2)^*)$。
  
  $(L_2, R_2)^*(L_1, R_1)^* = (R_2^*, L_2^*)(R_1^*, L_1^*) = (R_2^*R_1^*, L_1^*L_2^*)$
  
  $(\varphi \circ \psi)^* = * \cdot \psi \circ \varphi \circ * = * \cdot \varphi \circ * \cdot \psi \circ * = \varphi^* \circ \psi^*$
  
  (因为$*^{-1} = *$) 故上面两式相等
\end{proof}

\begin{proposition}
  $A \xrightarrow{\theta} M(A)$ 为一个$*$同态。
\end{proposition}

\begin{proof}
  要证$\theta(a)^* = \theta(a^*)$
  
  $\theta(a)^* = (\ell_a, r_a)^* = (r_a^*, \ell_a^*)$
  
  $\theta(a^*) = (\ell_{a^*}, r_{a^*})$
  
  而由于$r_a^*(x) = r_a(x^*)^* = (x^*a)^* = a^*x = \ell_{a^*}(x)$
  
  上述两式相等。
\end{proof}

\begin{itemize}
  \item $A \hookrightarrow B$ 含幺$*$代数。$A$为$B$的$*$理想。
  
  \begin{equation*}
    \begin{tikzcd}[column sep=large, row sep=large]
      A \arrow[r] \arrow[dr, "\theta"'] & B \arrow[d, "\exists \tau"', "\text{保幺$*$同态}"] \\
      & M(A)
    \end{tikzcd}
  \end{equation*}
  
  \begin{equation*}
    \begin{aligned}
      B &\xrightarrow{\tau} M(A) \\
      b &\longmapsto (L_b, R_b)
    \end{aligned}
  \end{equation*}
  
  \begin{proof}
    只要证$\tau$保$*$
  
  $\tau(b^*) = (L_{b^*}, R_{b^*})$
  
  $\tau(b)^* = (L_b, R_b)^* = (R_b^*, L_b^*)$ 两式相等
  \end{proof}
\end{itemize}

\begin{lemma}
  若$A$是$C^*$代数,$(L, R) \in M(A)$
  则:$L, R \in \mathcal{B}(A) \subset \operatorname{End}_{\mathbb{C}}(A)$
\end{lemma}

\begin{proof}
  由$A$为Banach Sp,$A \xrightarrow{L} A$,$A \xrightarrow{R} A$。
  
  由闭图像Thm. 要证$L$连续。只要证
  
  若$x_n \longrightarrow x$,且$L(x_n) \longrightarrow y$ 则 $L(x) \longrightarrow y$
  
  只要证:$\forall z \in A$,有$zL(x) = zy$
  
  $zL(x) = R(z)x$
\end{proof}

\begin{proposition}
  $A$为赋范代数,则$A \times A \xrightarrow{\cdot} A$ 连续
  \begin{equation*}
    (x, y) \longmapsto xy
  \end{equation*}
\end{proposition}

\begin{proof}
  $x_n \longrightarrow x$,$y_n \longrightarrow y \stackrel{?}{\Rightarrow} x_n y_n \longrightarrow xy$
  
  $\|x_n y_n - xy\| = \|x_n y_n - x_n y + x_n y - xy\|$
  $\leq \|x_n\| \|y_n - y\| + \|x_n - x\| \|y\| \longrightarrow 0$
\end{proof}


 
  回到主题
  ,由 $x_n \longrightarrow x$,$\Rightarrow R(z)x_n =zL(x_n)\longrightarrow R(z)x$
  
 
  
  由$L(x_n) \longrightarrow y \Rightarrow zL(x_n) \longrightarrow zy$
  
  故 $zy = R(z)x$ ($T_2$) 


\begin{proposition}
  若$A$为$C^*$代数,$(L, R) \in M(A)$
    则:$\|L\| = \|R\|$
\end{proposition}

\begin{proof}
  $\forall x \in A$ 且 $\|x\| \leq 1$,$\forall y \in A$,$\|y\| \leq 1$。
  
  $\|yL(x)\| = \|R(y)x\| \leq \|R(y)\| \|x\| \leq \|R(y)\| \leq \|R\| \|y\|$
  $\leq \|R\|$ 
  
  取$y = L(x)^*$,则$\forall x$,$\|x\| \leq 1$,$\|L(x)\|^2 \leq \|R\| \|L(x)\|$
  
  $\Rightarrow \forall x$,$\|x\| \leq 1$,有 $\|L(x)\| \leq \|R\| \Rightarrow \|L\| \leq \|R\|$
  
  同理 $\|R\| \leq \|L\|$
\end{proof}

\begin{remark}
  \begin{equation*}
    \begin{aligned}
      A &\xrightarrow{r} \mathcal{B}(A)  \\
      a &\longmapsto r_a
    \end{aligned}\text{,为等距嵌入}
  \end{equation*}
\end{remark}

\begin{definition}
  若$A$为$C^*$代数,已证 $M(A)$上有典范$*$运算。 且$\forall (L, R) \in M(A)$ $\|L\| = \|R\|$
  
  定义 $\|(L, R)\| = \|L\| = \|R\|$
\end{definition}

\begin{theorem}
  $M(A)$为一个含幺$C^*$代数。且
  \begin{equation*}
    \begin{aligned}
      A &\xrightarrow{\theta} M(A) \\
      a &\longmapsto (\ell_a, r_a)
    \end{aligned}
  \end{equation*}
  为等距$*$同构嵌入。
\end{theorem}

\begin{proof}
  $(M(A), \|\cdot\|) \hookrightarrow (\mathcal{B}(A) \oplus \mathcal{B}(A)^{op}, \|\cdot\|_{eq})$ 为Banach Sp.
  
  $\|(\varphi, \psi)\| = \max\{\|\varphi\|, \|\psi\|\} \Rightarrow (M(A), \|\cdot\|)$ 为赋范空间。
  
  下证$M(A)$为闭子空间
  
  设$(L_n, R_n) \in M(A)$ 且 $(L_n, R_n) \xrightarrow{\|\cdot\|_\infty} (L, R)$
  
  下证$(L, R) \in M(A)$ $\forall x, y \in A$
  
  $L_n(x)y = L_n(xy)$
  
  $\downarrow$ \quad \quad 同理 $xR(y) = R(xy)$
  
  $L(x)y = L(xy)$。
  
  $xL_n(y) = R_n(x)y$
  
  $\downarrow$ \quad \quad $\downarrow$
  
  $xL(y) = R(x)y$
\end{proof}

若$A$不含幺元,则 $A \varsubsetneqq M(A)$ 且 $\theta(A) \cap \mathbb{C} 1= \emptyset$。



$A \xrightarrow{\theta \cdot} \theta(A)(A(\text{挖补Thm})) \oplus \mathbb{C}1. \subset M(A)$

$(a + \lambda 1)^* = a^* + \overline{\lambda} 1 \in \theta(A) \oplus \mathbb{C}1$。

故$\theta(A) \oplus \mathbb{C}1$ $*$封闭

$M(A)$ Banach Sp. 下证 $A \oplus \mathbb{C}1$闭于$M(A)$

由$A$闭(Banach) $\mathbb{C}1$有限维$\Rightarrow A \oplus \mathbb{C}1$闭(证明下一条)

\begin{equation*}
  \begin{tikzcd}[column sep=huge, row sep=huge]
    A \arrow[r, hook] \arrow[dr, "\forall\psi" description, "*\text{同态}"'] & A \oplus \mathbb{C}1 \subset M(A) \arrow[d, "\exists! \varphi"', "\text{保幺$*$同态}"] \\
    & B \text{ $\forall$含幺$C^*$代数}
  \end{tikzcd}
\end{equation*}

\begin{proposition}
  若$X$为赋范Sp,$Y$为$X$的闭子空间,$Z$为$X$的有限维子空间,则 $Y + Z$闭子Sp
\end{proposition}

\begin{proof}

  \begin{equation*}
    \begin{aligned}
     \text{考虑商范数} X &\xrightarrow{\pi} X/Y \quad\text{$\pi$有界线性}\\
      Z &\longmapsto \pi(Z)\text{  为$X/Y$的有限维线性Sp}
    \end{aligned}
  \end{equation*}

  
  从而$\pi(Z)$闭于$X/Y$,而 $\pi^{-1}\pi(Z) = \ker\pi + Z$
   $= Y + Z$,闭于$X$。
\end{proof}
\vspace{1cm}
$A \xrightarrow{f} B$ \quad $A, B$为$C^*$代数,
\quad $f$ $*$同态,$\operatorname{Hom}(A, B)$

\begin{equation*}
  \begin{tikzcd}[column sep=huge, row sep=huge]
    A \arrow[r] \arrow[d, "f"', "*\text{同态}"] & \tilde{A} \arrow[d, "\exists! \tilde{f}"', "\text{保幺$*$同态}"] \\
    B \arrow[r] & \tilde{B}
  \end{tikzcd}
\end{equation*}

$\widetilde{\operatorname{id}_A} = \operatorname{id}_{\tilde{A}}$,$\widetilde{f \circ g} = \tilde{f} \circ \tilde{g}$,这是因为

\begin{equation*}
  \begin{tikzcd}[row sep=large, column sep=large]
    A \arrow[rr, "g"] \arrow[dd] \arrow[dr, "f \circ g"'] & & B \arrow[dd] \arrow[dl, "f"'] \\
    & C \arrow[dd] & \\
    \tilde{A} \arrow[rr, "\tilde{g}"'] \arrow[dr, "\tilde{f} \circ \tilde{g}"'] & & \tilde{B} \arrow[dl, "\tilde{f}"'] \\
    & \tilde{C} &
  \end{tikzcd}
\end{equation*}

顶上变换,所有的侧面变换$\Rightarrow$ 底部变换

故幺元化为函子

\begin{example}
  Banach $*$代数
  
  $(L^1(G), *)$ 为Banach $*$代数,不是$C^*$代数。
  
  
\end{example}

\begin{definition}
  $A$为Banach 代数,且有$*$运算,若满足:$\forall x \in A$,有$\|x^*\| = \|x\|$,称$A$为Banach $*$代数
\end{definition}

\begin{definition}
  $A$为含幺代数,$G(A)$表示$A$的全体可逆元。$G(A)$为乘法群
\end{definition}

\begin{lemma}
  若$A$为含幺Banach代数,则$G(A)$为开集
  
  e.g. $\operatorname{GL}_n(\mathbb{R})$开于$M_n(\mathbb{R})$
  
  $\det$连续,$\det^{-1}(\mathbb{R} \setminus \{0\})$开
\end{lemma}

\begin{proposition}
  $A$为含幺Banach代数,$I_A$为幺元。则$B(I_A, 1) \subset G(A)$
\end{proposition}

\begin{proof}
  $\forall a \in B(I_A, 1)$设$a = 1 - x$,其中$x \in B(0, 1)$
  
  下证$a$可逆,i.e. 证$ (1-x)^{-1} = 1 + x + x^2 + \cdots$
  
  在Banach sp(绝对收敛$\Rightarrow$收敛)
  
  先证$\sum_{n=0}^{\infty} \|x^n\|$收敛,$\forall A$ Banach sp,知$\sum x^n$收敛
  
  $\forall n$,$\|x^n\| \leq \|x\|^n$,$\sum_{n=0}^{\infty} \|x^n\| \leq \sum_{n=0}^{\infty} \|x\|^n < +\infty$ $\Leftarrow \|x\| < 1$
  
  下证$(\sum_{n=0}^{\infty} x^n)(1-x) = 1 = (1-x)(\sum_{n=0}^{\infty} x^n)$
  
  $(\sum_{n=0}^{\infty} x_n) y = (\lim_{N\to\infty} \sum_{n=0}^{N} x_n) y = $ (右乘连续$\Rightarrow$)$\lim_{N\to\infty} (\sum_{n=0}^{N} x_n y) = \sum_{n=0}^{\infty} x_n y$
  
  故$(\sum_{n=0}^{\infty} x^n)(1-x) = \sum_{n=0}^{\infty} (x^n - x^{n+1}) = 1$ ($\|x^n\| \to 0$)
\end{proof}

\begin{proof}
  $\forall a \in G(A)$,考虑
  \begin{equation*}
    \begin{aligned}
      A &\xrightarrow{l_a} A \text{为同胚}\\
      x &\longmapsto ax \\
      1 &\longmapsto a
    \end{aligned}
  \end{equation*}
  
  $1$为$G(A)$内点,$\Rightarrow l_a(1) = a$为$l_a(G(A)) = G(A) $的内点
  $\Rightarrow G(A)$开。
\end{proof}

\begin{proposition}
  若$a \in G(A)$,则$B(a, \frac{1}{\|a^{-1}\|}) \subset G(A)$
\end{proposition}

\begin{proof}
  $\forall b \in B(a, \frac{1}{\|a^{-1}\|})$设$b = a + x$,其中$\|x\| < \frac{1}{\|a^{-1}\|}$
  
  考虑$a^{-1}b = 1 + a^{-1}x$,要证$b$可逆,只要证$a^{-1}b$可逆
  
  由于$ B(1, 1) \subset G(A)$,只要验证$a^{-1}x < 1$
  
  $\|a^{-1}x\| \leq \|a^{-1}\| \|x\| < 1$
\end{proof}

\begin{proposition}
  若$A$为Banach代数,$M$为$A$的极大理想,则$M$闭。
\end{proposition}

\begin{proposition}
  若$A$为赋范代数,
  \begin{enumerate}
    \item $I$为理想,则$\overline{I}$为理想
    \item $B$为子代数,则$\overline{B}$仍为子代数
  \end{enumerate}
\end{proposition}

\begin{proof}
  \begin{enumerate}
    \item $\forall n \in \overline{I}$,$\forall a \in A$,$\exists I$中序列
    \begin{equation*}
      \begin{aligned}
        ax_n &\longrightarrow ax \quad (\ell_a\text{连续}) \\
        x_n a &\longrightarrow xa \quad (r_a\text{连续})
      \end{aligned}
    \end{equation*}
    
    \item $\forall x, y \in \overline{B}$,$(x_n) \to x$,$(y_n) \to y$
    
    $(x_n y_n) \to (xy) \to xy = xy \in \overline{B}$
  \end{enumerate}
\end{proof}

\begin{proof}
  由于$ M$为理想 $\Rightarrow \overline{M}$为理想
  
  而$M \cap G(A) = \phi$ (真理想)
  
  $\Rightarrow \text{由} G(A)$开 \quad$\overline{M} \cap G(A) = \phi$
  
  $ M \subset \overline{M} \subsetneqq  A \Rightarrow M = \overline{M}$
\end{proof}

\begin{proposition}
  若$A$为赋范代数,$I$为闭理想,则$(A/I, \|\cdot\|)$为一个赋范代数
\end{proposition}

\begin{proof}
  $A/I$为商赋范空间,只需验证$\forall \alpha, \beta \in A/I$,有$\|\alpha\beta\| \leq \|\alpha\| \|\beta\|$
  
  由商范数定义:$\|\alpha\| = \inf\{\|a\| : a \in \alpha\}$,$\|\beta\| = \inf\{\|b\| : b \in \beta\}$
  
  对于$\alpha\beta = [xy]$(其中$x \in \alpha, y \in \beta$),有
  \begin{align*}
    \|\alpha\beta\| &= \inf\{\|z\| : z \in \alpha\beta\} \\
    &= \inf\{\|xy\| : x \in \alpha, y \in \beta\} \\
    &\leq \inf\{\|x\| \|y\| : x \in \alpha, y \in \beta\}  \\
    &= \inf\{\|x\| : x \in \alpha\} \cdot \inf\{\|y\| : y \in \beta\} \quad (\text{集合大 inf小 }) \\
    &= \|\alpha\| \|\beta\|
  \end{align*}
  
  故$(A/I, \|\cdot\|)$满足范数的次乘性,为赋范代数。
\end{proof}

\begin{corollary}
  $A$ Banach sp,$I$闭理想,则$(A/I, \|\cdot\|)$为Banach代数
\end{corollary}

\begin{proposition}
  若$A$为含幺交换Banach代数,$M$为极大理想,则$A/M \cong \mathbb{C}$
\end{proposition}

\begin{proof}
  $A/M$为域(可除代数)若任意元素谱非空,则$A/M \cong \mathbb{C}$(拓扑同构)
\end{proof}

\begin{definition}[Fréchet导数]
  若$X$, $Y$为赋范空间,$\Omega$开于$X$,$\Omega \xrightarrow{f} Y$,$x \in \Omega$。称$f$在$x$处可微(Fréchet导数),若满足$\exists X \xrightarrow{A} Y$有界线性,使
  \begin{equation*}
    \lim_{h \to 0}\frac{f(x+h) - f(x) - Ah}{\|h\|} = 0
  \end{equation*}
  $A$称为$f$在$x$处的微分。可以证明以上的$A$具有唯一性,$A$记为$Df(x)$。
\end{definition}

\begin{remark}
  $\Omega \xrightarrow{g} Y \xrightarrow{f} X$,$\Omega$开集,
  \begin{equation*}
    D(f \circ g)(x_0) = Df(g(x_0))Dg(x_0)
  \end{equation*}
  其中$f \leftarrow A$,$g \leftarrow B$,$x \longmapsto g(x_0) \longmapsto f(g(x_0))$,$f \circ g \leftarrow A \cdot B$。
\end{remark}

\begin{proposition}[链式法则]
  $\Omega$为$X$中开集,$\Omega \xrightarrow{g} Y \xrightarrow{f} Z$。已知$g$在$x$处可微且$Dg(x) = B$,$f$在$g(x)$处可微且$Df(g(x)) = A$,则$f \circ g$在$x$处可微且$D(f \circ g)(x) = A \cdot B$。
\end{proposition}

\begin{proof}
  \begin{align*}
    &\lim_{h \to 0}\frac{(f \circ g)(x+h) - (f \circ g)(x) - A \cdot B(h)}{\|h\|} \\
    &= \lim_{h \to 0}\frac{f(g(x+h) - g(x) + g(x)) - f(g(x)) - A(Bh)}{\|h\|} \\
    &= \lim_{h \to 0}\frac{f(g(x) + Bh + o_g(h)) - f(g(x)) - A(Bh)}{\|h\|} \\
    &= \lim_{h \to 0}\frac{O_f(Bh + o_g(h)) + A(Bh + o_g(h)) - A(Bh)}{\|h\|} \\
    &= \lim_{h \to 0}\frac{O_f(Bh + o_g(h))}{\|h\|} + A\left(\frac{o_g(h)}{\|h\|}\right) = 0
  \end{align*}
  
  因为$A\left(\frac{o_g(h)}{\|h\|}\right) \longrightarrow A(0) = 0$。
  
  \begin{equation*}
    \frac{O_f(Bh + o_g(h))}{\|h\|} = \begin{cases}
      \displaystyle\frac{O_f(Bh + o_g(h))}{\|Bh + o_g(h)\|} \cdot \frac{\|Bh + o_g(h)\|}{\|h\|} \longrightarrow 0 \cdot \text{有界} & \text{若}Bh + o_g(h) \neq 0 \\
      0 & \text{otherwise}
    \end{cases}
  \end{equation*}
\end{proof}

\begin{example}
  $X \xrightarrow{f} Y$,线性,则$f$在每个点处可微,且$Df(x) = f$。
\end{example}

\begin{example}
  $ \begin{aligned}
      X &\xrightarrow{f} Y \\
      x &\longmapsto Ax + b
    \end{aligned}$若$A \in \mathcal{B}(X, Y)$,$b \in Y$,则称$f$为仿射,

  $Df(x) = A$。仿射的表达也是唯一的。
\end{example}
\begin{theorem}
  若$A$为含幺复Banach代数,$a \in A$,则$\sigma(a) \neq \emptyset$,且为$\mathbb{C}$中紧子集
\end{theorem}

\begin{proof}
  先证$\sigma(a)$为$\mathbb{C}$中紧集
  
  对$\forall a$,定义连续映射
  \begin{equation*}
    \begin{aligned}
      \mathbb{C} &\xrightarrow{\tau} A \\
      \lambda &\longmapsto \lambda 1 - a
    \end{aligned}
  \end{equation*}
  
  $\lambda \in \tau^{-1}(G(A)^c) \Leftrightarrow \lambda 1 - a \in G(A)^c \Leftrightarrow \lambda 1 - a$不可逆
  $\Leftrightarrow \lambda \in \sigma(a)$
  
  由$ G(A)^c$闭 $\Rightarrow \tau^{-1}(G(A)^c)$闭集
  
  下证$\sigma(a)$有界。
  

\begin{definition}[谱半径]
  $A$为含幺Banach代数,$a \in A$,定义
  \begin{equation*}
    r(a) \triangleq \sup\{|\lambda| : \lambda \in \sigma(a)\}
  \end{equation*}
  称$r(a)$为$a$的谱半径。
\end{definition}

\begin{proposition}
  $r(a) \leq \|a\|$。
\end{proposition}

\begin{proposition}
  $r(a) = \max\{|\lambda| : \lambda \in \sigma(a)\}$。
\end{proposition}

\begin{proof}
  上面已证$\sigma(a)$是紧集。
\end{proof}  

  若$\lambda \neq 0$,则$\lambda 1 - a$不可逆$\Longleftrightarrow 1 - \lambda^{-1}a$不可逆。
  
  已证$B(1, 1) \subset G(A)$,$\Rightarrow$若$\lambda \in \sigma(a)$则$\|\lambda^{-1}a\| \geq 1$,
  i.e. $|\lambda| \leq \|a\|$,故$\sigma(a)$有界。
  
  下证$\sigma(a) \neq \emptyset$。反证,若$\sigma(a) = \emptyset$,则定义
  \begin{equation*}
    \begin{aligned}
      \mathbb{C} &\xrightarrow{f} A \quad \text{良定义} \\
      \lambda &\longmapsto (\lambda 1 - a)^{-1}
    \end{aligned}
  \end{equation*}
  下证$f$可微。
  

\begin{lemma}
  $A$为含幺Banach代数,则
  \begin{equation*}
    \begin{aligned}
      G(A) &\xrightarrow{\nu} G(A) \quad \text{可微} \\
      x &\longmapsto x^{-1}
    \end{aligned}
  \end{equation*}
\end{lemma}

\begin{proof}
  已证若$x$可逆,$B(x, \frac{1}{\|x^{-1}\|}) \subset G(A)$。
  
  若$h \in B(0, \frac{1}{\|x^{-1}\|})$,
  \begin{align*}
    \nu(x+h) &= (x+h)^{-1} = (x(1 + x^{-1}h))^{-1} = (1 + x^{-1}h)^{-1} \cdot x^{-1} \\
    &= \sum_{n=0}^{\infty}(-x^{-1}h)^n x^{-1} = x^{-1} + (-x^{-1}h)x^{-1} + \sum_{n=2}^{\infty}(-x^{-1}h)^n x^{-1}
  \end{align*}
  
  $\Rightarrow \nu(x+h) - \nu(x) = -x^{-1}hx^{-1} + \sum_{n \geq 2}(x^{-1}h)^n x^{-1}$
  
  定义$T_x: A \longrightarrow A$,连续线性,
  \begin{equation*}
    a \longmapsto -x^{-1}ax^{-1}
  \end{equation*}
  
  下证$D\nu(x) = T_x$。这是因为
  \begin{equation*}
    \frac{\nu(x+h) - \nu(x) - T_x h}{\|h\|} = \sum_{n \geq 2}\frac{(-1)^n}{\|h\|}(x^{-1}h)^n x^{-1} \quad \circled{1}
  \end{equation*}
  
  由$n \geq 2$有
  \begin{equation*}
    \left\|\frac{(-1)^n}{\|h\|}(x^{-1}h)^n x^{-1}\right\| \leq \frac{1}{\|h\|}\|x^{-1}\|^{n+1}\|h\|^n = \|h\|^{n-1}\|x^{-1}\|^{n+1}
  \end{equation*}
  
  \begin{equation*}
    \|\circled{1}\| \leq \left(\sum_{n \geq 2}\|h\|^{n-1}\|x^{-1}\|^{n-1}\right)\|x^{-1}\|^2 \longrightarrow 0
  \end{equation*}
\end{proof}
  由Lemma,$\lambda \longmapsto \lambda 1 - a \longmapsto (\lambda 1 - a)^{-1}$可微。
  
  $\Rightarrow$对于$\forall \varphi \in A^*$,有$\mathbb{C} \xrightarrow{f} A \xrightarrow{\varphi} \mathbb{C}$,$\varphi$有界线性。
  
  下证$\varphi \circ f$为有界函数,只要证$f(\mathbb{C})$在$A$中有界。
  
  $\forall |\lambda| > \|a\| + 1$,$\Rightarrow \|\lambda^{-1}a\| < 1$,有
  \begin{equation*}
    f(\lambda) = (\lambda(1 - \lambda^{-1}a))^{-1} = \lambda^{-1}(1 - \lambda^{-1}a)^{-1} = \lambda^{-1}\sum_{n=0}^{\infty}(\lambda^{-1}a)^n
  \end{equation*}
  
  \begin{equation*}
    \|f(\lambda)\| \leq \frac{1}{|\lambda|}\sum_{n=0}^{\infty}\|(\lambda^{-1}a)^n\| \leq \sum_{n=0}^{\infty}|\lambda^{-1}|^{n+1}\|a\|^n = \frac{1}{|\lambda|} \cdot \frac{1}{1 - |\lambda^{-1}|\|a\|} = \frac{1}{|\lambda| - \|a\|} < 1
  \end{equation*}
  
  $\forall |\lambda| \leq \|a\| + 1$,为紧集,像是紧的,故有界。
  \begin{equation*}
    f(\mathbb{C}) = f(\bar{B}(0, \|a\|+1)) \cup f(\bar{B}(0, \|a\|+1)^c)
  \end{equation*}
  
  由刘维尔定理,$\varphi \circ f$全平面解析有界$\Rightarrow \varphi \circ f$为常值函数。
  
  由Hahn-Banach延拓定理,$\forall \varphi \in A^*$允许分离点$\Rightarrow f(\mathbb{C})$为常元。
  
  矛盾,$\forall \lambda_1 \neq \lambda_2$,有$\lambda_1 1 \neq \lambda_2 1 \Rightarrow \lambda_1 1 - a \neq \lambda_2 1 - a \Rightarrow (\lambda_1 1 - a)^{-1} \neq (\lambda_2 1 - a)^{-1}$。
  
  故$\sigma(a) \neq \emptyset$。
\end{proof}


\section{Gelfand理论}

\begin{remark}
  称交换含幺$F$代数$A$具有谱性质,若$\forall A \xrightarrow{f} B$满同态,有$B$中每个元素谱非空。
  
  $\Longleftrightarrow$ $A$的每个元素谱非空,且$\forall M$为$A$的极大理想,$A/M$的元素谱非空。
\end{remark}

\begin{theorem}
  交换含幺Banach代数$A$具有谱性质。
\end{theorem}

\begin{proof}
  由Gelfand定理,$A$的每个元素谱非空。已证$\forall M$为$A$极大理想,$A/M$亦为Banach代数,从而$A/M$的每个元素谱非空。
\end{proof}

\begin{theorem}
  若$A$为交换含幺$F$代数且具有谱性质,则$\forall a \in A$,有
  \begin{equation*}
    \sigma(a) = \{\tau(a) : \tau \in \Sigma A\} \triangleq \hat{a}(\Sigma A), \quad \hat{a}(\tau) \triangleq \tau(a)
  \end{equation*}
  其中$\Sigma A \triangleq \{\tau \in F^A : \tau\text{为保幺同态}\}$,称为$A$的谱。
\end{theorem}

\begin{corollary}[Gelfand]
  若$A$为交换含幺Banach代数,$a \in A$,则
  \begin{equation*}
    \sigma(a) = \{\tau(a) : \tau \in \Sigma A\}
  \end{equation*}
  其中$A \xrightarrow{\tau} \mathbb{C}$称为特征。
\end{corollary}

\begin{proposition}
  若$A$为含幺Banach代数,则$A \xrightarrow{\tau} \mathbb{C}$保幺同态为压缩的。
\end{proposition}

\begin{proof}
  $\forall a \in A$,须证$\tau(a) \in \sigma(a)$。
  
  这是因为$\tau(\tau(a)1 - a) = \tau(a)\tau(1) - \tau(a) = 0$。
  
  $\Rightarrow \tau(a)1 - a$不可逆。
  
  $\Rightarrow |\tau(a)| \leq r(a) \leq \|a\|$,i.e. $\tau$压缩,从而$\tau$连续,$\|\tau\| \leq 1$。
\end{proof}

\begin{proposition}
  由上一个性质知,若$A$为含幺Banach代数,则$\Sigma A \subset \bar{B}_{A^*}(0, 1)$。而$\bar{B}_{A^*}(0, 1)$在弱$*$拓扑下为紧集。只要证$\Sigma A$闭,则$\Sigma A$紧。
\end{proposition}

\begin{proof}
  任取$\{\tau_\alpha\}_{\alpha \in \Lambda} \longrightarrow \tau$为$\Sigma A$中的网,下证$\tau \in \Sigma A$。
  
  $1 = \tau_\alpha(1) \longrightarrow \tau(1) \Rightarrow \tau(1) = 1$。
  
  $\tau_\alpha(ab) = \tau_\alpha(a)\tau_\alpha(b) \longrightarrow \tau(ab) = \tau(a)\tau(b)$。
  
  故$\tau \in \Sigma A$。
\end{proof}

\begin{proposition}
  $A$为Banach代数,则$\Sigma A$赋弱$*$拓扑为紧$T_2$空间。
\end{proposition}

\begin{proposition}
  若$\Omega$为紧$T_2$空间,则$C(\Omega)$为交换含幺Banach代数。$C(\Omega)$为$\Omega$上全体$\mathbb{C}$值连续映射。
\end{proposition}

\begin{theorem}[Gelfand变换]
  $A$为交换含幺Banach代数,则
  \begin{equation*}
    \begin{aligned}
      A &\xrightarrow{\theta} C(\Sigma A) \quad \text{为保幺同态且压缩且$\|\hat{a}\| = r(a)$。} \\
      a &\longmapsto \hat{a}: \Sigma A \longrightarrow \mathbb{C} \\
      & \quad\quad\quad \tau \longmapsto \tau(a)
    \end{aligned}
  \end{equation*}
  
\end{theorem}

\begin{proof}
  (良定义)$\hat{a}$连续,因为$\forall \tau_\alpha \xrightarrow{w^*} \tau$,有
  \begin{equation*}
    \begin{tikzcd}
      \hat{a}(\tau_\alpha) \arrow[r] \arrow[d, equal] & \hat{a}(\tau) \arrow[d, equal] \\
      \tau_\alpha(a) \arrow[r] & \tau(a)
    \end{tikzcd}
  \end{equation*}
  
  验证同态性质:$\widehat{a+b} = \hat{a} + \hat{b}$,$\widehat{ab} = \hat{a}\hat{b}$,$\widehat{\lambda a} = \lambda\hat{a}$,$\hat{1} = \tilde{1}$。
  
  \begin{equation*}
    \|\hat{a}\| = \sup\{|\hat{a}(\tau)| : \tau \in \Sigma A\} = \sup\{|\tau(a)| : \tau \in \Sigma A\} = r(a)
  \end{equation*}
  (已证$\{\tau(a) : \tau \in \Sigma A\} = \sigma(a)$)
  
  $\Rightarrow \|\hat{a}\| = \|\theta(a)\| \leq \|a\|$。
\end{proof}

\begin{theorem}[Jacobson根]
  \begin{equation*}
    \ker\theta = \bigcap\{M : M\text{为}A\text{的极大理想}\}
  \end{equation*}
\end{theorem}

\begin{proof}
  $a \in \ker\theta \Longleftrightarrow \theta(a) = 0 \Longleftrightarrow \forall \tau \in \Sigma A$,$\tau(a) = 0$
  $\Longleftrightarrow a \in \bigcap_{\tau \in \Sigma A}\ker\tau$。
\end{proof}

\begin{proposition}
  \begin{equation*}
    \begin{aligned}
      \Sigma A &\longrightarrow \mathrm{Max}A \quad \text{为双射} \\
      \tau &\longmapsto \ker\tau
    \end{aligned}
  \end{equation*}
\end{proposition}

\begin{proof}
  由同态基本定理,有交换图:
  \begin{equation*}
    \begin{tikzcd}
      A \arrow[r, "\tau", two heads] \arrow[d, "\pi"'] & F \\
      A/\ker\tau \arrow[ur, "\cong"'] &
    \end{tikzcd}
  \end{equation*}
  故$\tau \longmapsto \ker\tau$。
  
  下证单射。设$\ker\tau_1 = \ker\tau_2$,有交换图:
  \begin{equation*}
    \begin{tikzcd}
      A \arrow[r, "\tau_1", two heads] \arrow[d, "\tau_2"'] & F \arrow[dl, "\exists! \sigma"] \\
      F &
    \end{tikzcd}
  \end{equation*}
  $\sigma$数乘。由保幺,$\sigma = \ell_1 = \mathrm{id}$,$\Rightarrow \tau_1 = \tau_2$。
\end{proof}

\begin{proposition}
  若$A$为Banach代数,$A \xrightarrow{\tau} \mathbb{C}$为同态,则$\tau$为压缩映射。
\end{proposition}

\begin{proof}
  设$A \hookrightarrow \hat{A}$为Banach代数幺元化,也为$A$作为纯代数的幺元化。有交换图:
  \begin{equation*}
    \begin{tikzcd}
      A \arrow[r, hook] \arrow[rd, "\tau"'] & \hat{A} \arrow[d, "\exists! \tilde{\tau}"] \\
      & \mathbb{C}
    \end{tikzcd}
  \end{equation*}
  由上次已证,$\tilde{\tau}$压缩$\Rightarrow \tau = \tilde{\tau}|_A$压缩。
\end{proof}

\begin{definition}[单点紧化]
  若$Y$为拓扑空间,$X$为其子空间,$Y = X \cup \{\infty\}$。若$Y$为紧$T_2$空间,则称$Y$为$X$的一个单点紧化。
  
  或一般的定义:$X \overset{\eta}{\hookrightarrow} Y = \eta(X) \cup \{\infty\}$,$\eta$为同胚嵌入,称$(Y, \eta)$为$X$的一个单点紧化。
\end{definition}

\begin{proposition}
  若$X$存在单点紧化,则$X$为局部紧$T_2$空间。
\end{proposition}

\begin{proposition}
  $X \hookrightarrow X \cup \{\infty\}$紧$T_2$空间,则$\forall \varphi \in C_0(X)$,$\exists! \tilde{\varphi} \in C(X \cup \{\infty\})$,使$\tilde{\varphi}|_X = \varphi$且$\tilde{\varphi}(\infty) = 0$。
  \begin{equation*}
    \begin{tikzcd}
      X \arrow[r, hook] \arrow[rd, "\varphi"'] & X \cup \{\infty\} \arrow[d, "\exists! \tilde{\varphi}"] \\
      & \mathbb{C}
    \end{tikzcd}
  \end{equation*}
\end{proposition}

\begin{remark}
  $C_0(X) \triangleq \{f \in \mathbb{C}^X : f\text{连续且}\forall \varepsilon > 0, \exists K\text{为}X\text{的紧子集使}\forall x \in X \setminus K\text{有}|f(x)| < \varepsilon\}$。
  
  易知,$C_0(X) \subset C_b(X)$。
\end{remark}

\begin{proof}
  唯一性显然,下证存在性。
  
  已知$\varphi \in C_0(X)$,定义$\tilde{\varphi}(x) \triangleq \begin{cases} \varphi(x), & x \in X \\ 0, & x = \infty \end{cases}$,使$\tilde{\varphi}|_X = \varphi$且$\tilde{\varphi}(\infty) = 0$。
  
  由于$X$为$X \cup \{\infty\}$的开子集且$\varphi|_X = \varphi$连续,则$\forall x \in X$,$\tilde{\varphi}$在$x$处连续。
  
  $\forall \varepsilon > 0$,由$\varphi \in C_0(X)$,$\exists K$为$X$中紧集,$K$闭集,$X \setminus K \cup \{\infty\}$为$\{\infty\}$的开邻域,$\forall x \in X \setminus K$,有$\varphi(x) \in B_{\mathbb{C}}(0, \varepsilon)$。
\end{proof}

\begin{remark}
  $C(X \cup \{\infty\}, \infty) \triangleq \{f \in C(X \cup \{\infty\}) : f(\infty) = 0\}$,为$C(X \cup \{\infty\})$的闭的极大理想。
\end{remark}

\begin{proof}
  $X$为紧空间,$\forall x \in X$,有
  \begin{equation*}
    \begin{aligned}
      C(X) &\xrightarrow{\hat{x}} \mathbb{C} \quad \text{为保幺同态} \\
      f &\longmapsto f(x)
    \end{aligned}
  \end{equation*}
  有交换图:
  \begin{equation*}
    \begin{tikzcd}
      C(X) \arrow[r, "\hat{x}"] \arrow[d] & \mathbb{C} \\
      C(X)/\ker\hat{x} \arrow[ur, "\text{为代数同构$\Rightarrow \ker\hat{x}$极大理想。}"'] &
    \end{tikzcd}
  \end{equation*}
  
  
  $C(X \cup \{\infty\}, \infty)$为交换Banach代数。
\end{proof}

\begin{lemma}
  \begin{equation*}
    \begin{aligned}
      C_0(X) &\xrightarrow{\theta} C(X \cup \{\infty\}, \infty) \quad \text{为等距代数同构} \\
      \varphi &\longmapsto \tilde{\varphi}
    \end{aligned}
  \end{equation*}
\end{lemma}

\begin{proof}
  若$\tilde{\varphi} = \tilde{\psi} \Rightarrow \varphi = \tilde{\varphi}|_X = \tilde{\psi}|_X = \psi \Rightarrow \theta$单。
  
  有交换图:
  \begin{equation*}
    \begin{tikzcd}
      X \arrow[r, hook] \arrow[rd, "f|_X"'] & X \cup \{\infty\} \arrow[d, "f", "\text{连续}"'] \\
      & \mathbb{C}
    \end{tikzcd}
  \end{equation*}
  由定义易知,$\theta$满。
  
  \begin{equation*}
    \begin{aligned}
      C(X \cup \{\infty\}, \infty) &\xrightarrow{\theta^{-1}} C(X) \quad \text{为代数同态} \\
      f &\longmapsto f|_X
    \end{aligned}
  \end{equation*}
  
  $\mathrm{Im}\,\theta^{-1} = C_0(X)$为$C(X)$的子代数。$\Rightarrow C_0(X)$为交换Banach代数。
\end{proof}

\begin{lemma}
  $A$为Banach代数,$A \hookrightarrow \hat{A}$为Banach代数幺元化。$\Sigma A$为全体非零同态。有交换图:
  \begin{equation*}
    \begin{tikzcd}
      A \arrow[r, hook] \arrow[rd, "\tau"'] & \hat{A} \arrow[d, "\exists! \tilde{\tau}"] \\
      & \mathbb{C}
    \end{tikzcd}
  \end{equation*}
  则
  \begin{equation*}
    \begin{aligned}
      \{0\} \cup \Sigma A &\xrightarrow{\theta} \Sigma\hat{A} \quad \text{为同胚} \\
      \tau &\longmapsto \tilde{\tau}
    \end{aligned}
  \end{equation*}
\end{lemma}

\begin{proof}
  $\tau_1 \longmapsto \tilde{\tau}_1$,$\tau_1 = \tilde{\tau}_1|_A = \tilde{\tau}_2|_A = \tau_2 \Rightarrow \theta$单。
  
  任取$\psi \in \Sigma\hat{A}$,$\psi|_A^{\text{显}} \in \{0\} \cup \Sigma A$,$\widetilde{\psi|_A} = \psi$(唯一性)$\Rightarrow \theta$满。
  
  已证$\theta$为双射,下证$\theta$为同胚。Rmk:$\{0\} \cup \Sigma A \subset A^*$,$T_2$。
  
  由已证$\Sigma\hat{A}$紧,$\{0\} \cup \Sigma A$为$T_2$。
  
  从而只要证$\theta^{-1}$为连续,从而同胚。
  \begin{equation*}
    \begin{aligned}
      \Sigma\hat{A} &\xrightarrow{\theta^{-1}} \{0\} \cup \Sigma A \\
      \psi &\longmapsto \psi|_A
    \end{aligned}
  \end{equation*}
  
  这是因为:若$\psi_\alpha \xrightarrow{w^*} \psi$(在$\hat{A}^*$中),要证$\psi_\alpha|_A \xrightarrow{w^*} \psi|_A$(在$A^*$中)。
  
  而$\forall a \in A$,有$\psi_\alpha|_A(a) = \psi_\alpha(a) \longrightarrow \psi(a) = \psi|_A(a)$。
  
  $\Rightarrow \psi_\alpha|_A \xrightarrow{w^*} \psi|_A$(在$A^*$中)。
\end{proof}

\begin{proposition}
  \begin{equation*}
    \begin{aligned}
      \Sigma A &\xrightarrow{\mu} \Sigma\hat{A} = \mu(\Sigma A) \cup \{\tilde{0}\} \quad \text{为单点紧化} \\
      \tau &\longmapsto \tilde{\tau}
    \end{aligned}
  \end{equation*}
  其中$\tilde{\tau}|_A = \tau$(从而$\Sigma A$为LCH空间)。
  
  诱导
  \begin{equation*}
    \begin{aligned}
      C_0(\Sigma A) &\longrightarrow C(\Sigma\hat{A}, \tilde{0}) \quad \text{等距同构} \\
      \varphi &\longmapsto \tilde{\varphi}
    \end{aligned}
  \end{equation*}
  有交换图:
  \begin{equation*}
    \begin{tikzcd}
      \Sigma A \arrow[r, "\mu"] \arrow[rd, "\varphi"'] & \Sigma\hat{A} \arrow[d, "\tilde{\varphi}"] \\
      & \mathbb{C}
    \end{tikzcd}
  \end{equation*}
  其中$\tilde{\varphi} \circ \mu = \varphi$且$\tilde{\varphi}(\tilde{0}) = 0$。
\end{proposition}

\begin{theorem}[Gelfand]
  $A$为交换Banach代数,则
  \begin{equation*}
    \begin{aligned}
      A &\xrightarrow{\theta} C_0(\Sigma A) \quad \text{为代数同态,且}\|\hat{a}\| = r(a) \\
      a &\longmapsto \hat{a}: \tau \longmapsto \tau(a)
    \end{aligned}
  \end{equation*}
\end{theorem}

\begin{proof}
  已证$\sigma_{\hat{A}}(a) \cup \{0\} = \tilde{\sigma}_{\hat{A}}(a) = \{\tilde{\tau}(a) : \tilde{\tau} \in \Sigma\hat{A}\} = \{\tau(a) : \tau \in \Sigma A \cup \{0\}\}$。
  
  $\|\hat{a}\| = \|\tilde{\hat{a}}\| = \|\hat{\hat{a}}\| = r(a)$。
  
  有交换图:
  \begin{equation*}
    \begin{tikzcd}
      \Sigma A \arrow[r, hook] \arrow[rd, "\hat{a}"'] & \Sigma\hat{A} \arrow[d, "\tilde{\hat{a}}"] \\
      & \mathbb{C}
    \end{tikzcd}
  \end{equation*}
  $\hat{a}(\tilde{0}) = \tilde{0}(a) = 0$($\tilde{\hat{a}} = \hat{a}$)。
  
  $\ker\theta = \{a : \hat{a} = 0\} = \{a \in A : \forall \tau \in \Sigma A, \tau(a) = 0\} = \bigcap_{\tau \in \Sigma A}\ker\tau$。
\end{proof}

\begin{proposition}
  $A \xrightarrow{\tau} \mathbb{C}$,$A$交换不含幺,有交换图:
  \begin{equation*}
    \begin{tikzcd}
      A \arrow[r, "\tau", two heads] \arrow[d, "\pi"'] & \mathbb{C} \\
      A/\ker\tau \arrow[ur, "\cong"'] &
    \end{tikzcd}
  \end{equation*}
  $\ker\tau$是$A$的极大理想。
\end{proposition}

\begin{definition}[正则理想]
  若$A$为代数,$I \trianglelefteq A$,满足$A/I$含幺,称$I$为一个正则理想。
\end{definition}

\begin{proposition}
  $A$的全体正则极大理想,记为$\mathrm{Max}(A)$($A$交换)。则
  \begin{equation*}
    \begin{aligned}
      \Sigma A &\xrightarrow{\theta} \mathrm{Max}(A) \quad \text{为双射} \\
      \tau &\longmapsto \ker\tau
    \end{aligned}
  \end{equation*}
\end{proposition}

\begin{proof}
  已证$\ker\tau$是正则极大。下证$\theta$单。
  
  若$\ker\tau_1 = \ker\tau_2$,有交换图:
  \begin{equation*}
    \begin{tikzcd}
      A \arrow[r, "\tau_1", two heads] \arrow[d, "\tau_2"'] & \mathbb{C} = A/\ker\tau \arrow[dl, "\exists! \psi"] \\
      \mathbb{C} &
    \end{tikzcd}
  \end{equation*}
  $\psi$代数同态,$\psi$数乘$\Rightarrow \psi = \ell_\lambda$,$\lambda \in \mathbb{C}$。
  
  $\ell_\lambda(1 \cdot 1) = \ell_\lambda(1)\ell_\lambda(1) \Rightarrow \lambda = \lambda^2$。由$\psi$满,$\lambda \neq 0$。
  
  $\Rightarrow \lambda = 1 \Rightarrow \tau_2 = \tau_1$。
  
  下证$\theta$满:
\end{proof}

\begin{proposition}
  若$A$为Banach代数,$M$为正则极大理想,则$M$为闭理想。
\end{proposition}

\begin{proof}
  $\forall M \in \mathrm{Max}(A)$,有$A \xrightarrow{\pi} A/M$为Banach代数,交换含幺单环,故$A/M \cong \mathbb{C}$。
  
  有交换图:
  \begin{equation*}
    \begin{tikzcd}[column sep=large]
      A \arrow[r, "\pi", two heads] & A/M \arrow[r, "\cong"] & \mathbb{C} \arrow[ll, bend right=50, "\exists \tau \neq 0"']
    \end{tikzcd}
  \end{equation*}
  故$\exists \tau$ s.t. $\ker\tau = M$。
\end{proof}

\begin{proposition}
  $I$为$A$的正则理想,$[e] \in A/I$为幺元,$e \in A$,则$e \notin \bar{I}$。
\end{proposition}

\begin{proof}
  $A \hookrightarrow \hat{A}$ Banach代数幺元化。反证,若$e \in \bar{I}$(在$\hat{A}$中)。
  
  在$\hat{A}$中$B(e, 1) \cap I \neq \varnothing$,取$b \in I$ s.t. $\|e - b\| < 1$。
  
  $\Rightarrow 1 - (e - b)$在$\hat{A}$中可逆,$(1 - (e - b))^{-1} = u$。
  
  $\Rightarrow$由$A$为$\hat{A}$的理想,$Au^{-1} \subset A \Rightarrow A \subset Au$。
  
  下证$Au \subset I$,这是因为$\forall a \in A$,$au = a(1 - e + b) = a - ae + ab \in I$($ae \in I$,$ab \in I$)。
  
  $A \subset Au \subset I \subset A$,与$I$真理想矛盾。
\end{proof}

\begin{proposition}
  若$I$为正则理想,$J \trianglelefteq A$,$I \subset J \Rightarrow J$正则。
\end{proposition}

\begin{proof}
  $[e] \in A/I$,$\forall a \in A$,$ae - a \in I \subset J$,$ea - a \in I \subset J$。
  
  $\Rightarrow J$正则。
\end{proof}

\begin{proposition}
  若$A$为Banach代数,$M$为正则极大理想,则$M$为闭理想。
\end{proposition}

\begin{proof}
  $M$已闭$\Rightarrow \bar{M}$真正则理想,由$M$极大,$M = \bar{M}$。
\end{proof}

\section{谱半径公式}

\begin{theorem}[Beurling-Gelfand谱半径公式]
  若$A$为含幺Banach代数,$a \in A$,则($A$无须交换)
  \begin{equation*}
    r(a) = \inf\{\|a^n\|^{\frac{1}{n}} : n \in \mathbb{N}^*\} = \lim_{n \to \infty}\|a^n\|^{\frac{1}{n}}
  \end{equation*}
\end{theorem}

\begin{proof}
  已证$\forall n \in \mathbb{N}^*$,$\sigma(a^n) = \sigma(a)^n = \{\lambda^n : \lambda \in \sigma(a)\}$。
  
  $\Rightarrow \forall \lambda \in \sigma(a)$,有$\lambda^n \in \sigma(a^n) \Rightarrow |\lambda^n| \leq r(a^n) \leq \|a^n\|$。
  
  $|\lambda| \leq \|a^n\|^{\frac{1}{n}}$,$\forall n$。
  
  $\Rightarrow r(a) \leq \inf\{\|a^n\|^{\frac{1}{n}} : n \in \mathbb{N}^*\} \leq \liminf_{n \to \infty}\|a^n\|^{\frac{1}{n}}$。
  
  下证$\limsup_{n \to \infty}\|a^n\|^{\frac{1}{n}} \leq r(a)$。
  
  记$\Omega = \bar{B}(0, r(a))^c$为$\mathbb{C}$中开集,$U \triangleq \bar{B}(0, \|a\|)^c$为$\Omega$的开子集。
  
  定义$\forall \varphi \in A^*$,
  \begin{equation*}
    \begin{aligned}
      \Omega &\xrightarrow{f} A \xrightarrow{\varphi} \mathbb{C} \\
      \lambda &\longmapsto (1 - \lambda^{-1}a)^{-1} \longmapsto \varphi((1 - \lambda^{-1}a)^{-1})
    \end{aligned}
  \end{equation*}
  
  $f$可微,$\varphi$可微$\Rightarrow \varphi \circ f$可微。
  
  \begin{equation*}
    \varphi((1 - \lambda^{-1}a)^{-1}) = \sum_{n=-\infty}^{\infty}C_n\lambda^n \quad \text{洛朗展开}
  \end{equation*}
  
  而
  \begin{equation*}
    \begin{aligned}
      U &\xrightarrow{f} A \xrightarrow{\varphi} \mathbb{C} \\
      \lambda &\longmapsto (1 - \lambda^{-1}a)^{-1}
    \end{aligned}
  \end{equation*}
  
  $\lambda \in U \Rightarrow |\lambda| > \|a\| \Rightarrow \|\lambda^{-1}a\| = \frac{1}{|\lambda|}\|a\| < 1$。
  
  故$(1 - \lambda^{-1}a)^{-1} = \sum_{n=0}^{\infty}(\lambda^{-1}a)^n = \sum_{n=0}^{\infty}\lambda^{-n}a^n \xrightarrow{\varphi} \sum_{n=0}^{\infty}\varphi(a^n)\lambda^{-n}$。
  
  由洛朗展开唯一,知$\varphi(a^n) = C_{-n}$。
  
  $\forall \lambda \in \Omega$有$f(\varphi(\lambda)) = \sum_{n=0}^{\infty}\varphi(a^n)\lambda^{-n} = \sum_{n=0}^{\infty}\varphi(\lambda^{-n}a^n)$收敛。
  
  $\Rightarrow \forall \lambda \in \Omega$,$\forall \varphi$,$\{\varphi(\lambda^{-n}a^n) : n \in \mathbb{N}^*\}$有界。
  
  $\Rightarrow \{\lambda^{-n}a^n : n \in \mathbb{N}^*\}$弱有界$\Rightarrow \{\|\lambda^{-n}a^n\| : n \in \mathbb{N}\}$有界
  
  (Thm. 弱有界$\Rightarrow$范数有界)。
  
  $\Rightarrow \exists M_\lambda > 0$,使$\forall n$,$\|\lambda^{-n}a^n\| \leq M_\lambda$,即$|\lambda|^{-n}\|a^n\| \leq M_\lambda$。
  
  $\Rightarrow \frac{1}{|\lambda|}\|a^n\|^{\frac{1}{n}} \leq M_\lambda^{\frac{1}{n}}$。
  
  $\Rightarrow \frac{1}{|\lambda|}\limsup_{n \to \infty}\|a^n\|^{\frac{1}{n}} \leq \limsup_{n \to \infty}M_\lambda^{\frac{1}{n}} = \lim_{n \to \infty}M_\lambda^{\frac{1}{n}} = 1$。
  
  $\Rightarrow \limsup_{n \to \infty}\|a^n\|^{\frac{1}{n}} \leq |\lambda|\leq r(a)$,$\forall \lambda \in \Omega$。
  

\end{proof}

\end{document}