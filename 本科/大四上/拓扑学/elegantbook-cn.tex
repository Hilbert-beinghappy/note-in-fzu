\documentclass[lang=cn,a4paper,newtx,figureplace=section,tableplace=section]{elegantbook}

\title{拓扑学}


\author{hjw}

\date{April 9, 2022}


\setcounter{tocdepth}{3}

\logo{logo-blue.jpg}
\cover{cover.jpg}

% 本文档命令
\usepackage{array}
\newcommand{\ccr}[1]{\makecell{{\color{#1}\rule{1cm}{1cm}}}}

% 修改标题页的橙色带
% \definecolor{customcolor}{RGB}{32,178,170}
% \colorlet{coverlinecolor}{customcolor}

\begin{document}

\maketitle
\frontmatter

\tableofcontents

\mainmatter

\chapter{拓扑空间基础}

\section{拓扑空间的定义}

我们首先引入拓扑和拓扑空间的概念
\begin{definition}[拓扑]\label{def:topology}
设 $X$ 为集合,拓扑 $\tau$ 为 $X$ 的子集族,满足
\begin{enumerate}
\item $\varnothing\in\tau$ 且 $X\in\tau$;
\item 若 $\{U_i\}_{i\in I}\subseteq\tau$ 则 $\bigcup_{i\in I}U_i\in\tau$;
\item 若 $U_1,\dots,U_n\in\tau$ 则 $\bigcap_{k=1}^n U_k\in\tau$
\end{enumerate}
\end{definition}

例如,对于实数集 $\mathbb{R}$,常取由所有开区间生成的标准拓扑;将所有子集都视为开集得到离散拓扑,这个是最大的拓扑;而仅有 $\varnothing$ 和 $X$ 为开集的则称为平凡拓扑。

\begin{example}[Zariski 拓扑]\label{def:zariski} 对域 $\mathbb{k}$ 上的仿射空间 $\mathbb{k}^n$,Zariski 拓扑定义为其闭集恰为多项式集合的零点集,即对于任意 $S\subseteq\mathbb{k}[x_1,\dots,x_n]$,令 \begin{equation} 
V(S)=\{x\in\mathbb{k}^n:\; f(x)=0\ \text{对所有 }f\in S\}, 
\end{equation} 并以所有此类 $V(S)$ 作为闭集产生拓扑。 
\end{example}

\begin{definition}[拓扑空间]\label{def:topological_space}
设 $X$ 为集合,若 $\tau\subseteq\mathcal{P}(X)$ 是一个拓扑,则称 $(X,\tau)$ 为拓扑空间。 
\end{definition}

\begin{definition}[邻域]\label{def:neighborhood} 对拓扑空间 $(X,\tau)$ 中的点 $x\in X$,若 $U\in\tau$ 且 $x\in U$,则称 $U$ 为点 $x$ 的一个领域,点 $x$ 的所有领域所成的集合记为 $\mathcal{N}(x)$,即 \begin{equation} \mathcal{N}(x)=\{\,U\in\tau:\; x\in U\,\}. \end{equation} \end{definition}

\begin{definition}[聚点]\label{def:accumulation_point} 设 $(X,\tau)$ 为拓扑空间,且 $A\subseteq X$,若对任意 $x\in X$ 满足对任意领域 $U\in\mathcal{N}(x)$,有 \begin{equation} (U\setminus\{x\})\cap A\neq\varnothing, \end{equation} 则称 $x$ 为集合 $A$ 的一个聚点。 
\end{definition}

\begin{definition}[闭包点]\label{def:closure_point} 设 $(X,\tau)$ 为拓扑空间,且 $A\subseteq X$,若对任意 $x\in X$ 满足对任意领域 $U\in\mathcal{N}(x)$,有 $U\cap A\neq\varnothing$,则称 $x$ 为集合 $A$ 的一个闭包点,集合 $A$ 的闭包记为 $\overline{A}$,并可等价地表述为 \begin{equation} \overline{A}=\{\,x\in X:\; U\cap A\neq\varnothing\ \text{对任意}\ U\in\mathcal{N}(x)\,\}. \end{equation}
\end{definition}

\begin{definition}[导集]\label{def:derived} 设 $(X,\tau)$ 为拓扑空间,且 $A\subseteq X$,集合 $A$ 的导集定义为所有属于 $A$ 的聚点所构成的集合,记为 $A'$,即 \begin{equation} A'=\{\,x\in X:\; (U\setminus\{x\})\cap A\neq\varnothing\ \text{对任意}\ U\in\mathcal{N}(x)\,\}. \end{equation} \end{definition}

\begin{definition}[内点]\label{def:interior_point} 设 $(X,\tau)$ 为拓扑空间,且 $A\subseteq X$,若对任意 $x\in X$ 存在一个邻域 $U\in\mathcal{N}(x)$ 使得 $U\subseteq A$,则称 $x$ 为集合 $A$ 的一个内点,集合 $A$ 的内记为 $A^\circ$,并可等价地表述为 \begin{equation} A^\circ=\{\,x\in X:\;\exists\,U\in\mathcal{N}(x)\ \text{使得}\ U\subseteq A\,\}=\bigcup\{\,U\in\tau:\ U\subseteq A\,\}. \end{equation} \end{definition}

\begin{remark}
回忆起度量空间定义的上面的东西,我们发现二者其实是等价的,即在度量空间 $(X,d)$ 上由开球生成的拓扑与公理化拓扑下的内、闭、导集概念相容。
\end{remark}

\section{开集与闭集}

\begin{definition}[开集]\label{def:open_set}设 $(X,\tau)$ 为拓扑空间,子集 $U\subseteq X$ 若对任意 $x\in U$ 存在邻域 $V\in\mathcal{N}(x)$ 使得 $V\subseteq U$,则称 $U$ 为 $X$ 的一个开集
\end{definition}

\begin{proposition}设 $(X,\tau)$ 为拓扑空间,且 $U\subseteq X$,则 $U$ 为开集当且仅当对任意 $x\in U$ 存在邻域 $V\in\mathcal{N}(x)$ 使得 $V\subseteq U$。 \end{proposition}
\begin{proof}
    左推右:
    若 $U$ 为开集,则对任意 $x\in U$ 按照开集的定义存在邻域 $V\in\mathcal{N}(x)$ 使得 $V\subseteq U$;
    
    右推左:若 $U$ 为空,则显然为开集;若 $U$ 非空,则对任意 $x\in U$ 存在邻域 $V_x\in\mathcal{N}(x)$ 使得 $V_x\subseteq U$,因此
\begin{equation}
U=\bigcup_{x\in U}V_x,
\end{equation}
故 $U$ 为开集。
\end{proof}

\begin{proposition}
若 $A$ 为闭集,则有
\begin{equation}
\overline{A}=A
\end{equation}
\end{proposition}
\begin{proof}
左推右:
    先证明$\overline{A}\subset A$,设 $x\in\overline{A}$,则按闭包的定义有
\begin{equation}
\forall U\ni x,\quad U\cap A\neq\emptyset,
\end{equation}
若反设 $x\notin A$,则由于 $A$ 为闭集,$X\setminus A$ 为开集,故存在开邻域 $V\ni x$ 且 $V\subset X\setminus A$,从而 $V\cap A=\emptyset$,与上式矛盾,因此 $x\in A$,于是 $\overline{A}\subset A$.

再证明
$A\subset\overline{A}$
:设 $x\in A$,则对于任意开邻域 $U\ni x$ 有 $x\in U\cap A$,从而 $U\cap A\neq\emptyset$,这说明 $x\in\overline{A}$,因此 $A\subset\overline{A}$。

右推左:若 $\overline{A}=A$,则对任意 $x\in X\setminus A$ 有 $x\notin\overline{A}$,因此存在开邻域 $U\ni x$ 使得
\begin{equation}
U\cap A=\emptyset,
\end{equation}
从而 $X\setminus A$ 为开集,故 $A$ 为闭集。
\end{proof}
\begin{proposition}
    $\overline{A}$为闭集
\end{proposition}
\begin{proof}
    设 $x\in X\setminus\overline{A}$,则由 $x\notin\overline{A}$ 可知存在开邻域 $U\ni x$ 使得 $U\cap A=\emptyset$,从而 $U\subset X\setminus\overline{A}$,说明 $X\setminus\overline{A}$ 为开集,故 $\overline{A}$ 为闭集。
\end{proof}

\begin{proposition}
    $\overline{A}$为包含$A$的最小闭集
\end{proposition}
\begin{proof}
    若 $F$ 为任一包含 $A$ 的闭集,则由 $A\subset F$ 及 $F$ 的闭性可推出 $\overline{A}\subset F$,因此 $\overline{A}$ 是包含 $A$ 的最小闭集。
\end{proof}
\begin{remark}

    $E$为含$A$的最小闭集当且仅当$E=\overline{A}$
\end{remark}
\begin{proof}

    右推左:已经证明过了

    左推右:设 $E$ 为含 $A$ 的最小闭集,则由于 $\overline{A}$ 是包含 $A$ 的闭集,由最小性得 $E\subset\overline{A}$,另一方面由 $\overline{A}$ 为所有包含 $A$ 的闭集的交集可得 $\overline{A}\subset E$,因此
\begin{equation}
E=\overline{A}.
\end{equation}
\end{proof}


\begin{proposition}
    若 $A$ 为开集,则有
\begin{equation}
A^\circ=A
\end{equation}
\end{proposition}

\begin{proof}
    左推右:由于 $A$ 自身为开集且显然 $A\subset A$,由 $A^\circ$ 作为所有包含于 $A$ 的开集的并的定义可知 $A\subset A^\circ$,而先前已证明 $A^\circ\subset A$,因此 $A^\circ=A$。

    右推左:因为 $A^\circ$ 为开集且
\begin{equation}
(A^\circ)^\circ = A^\circ
\end{equation}
故若 $A^\circ=A$ 则 $A$ 为开集,从而完成证明。
\end{proof}
\begin{proposition}
    
$A$的内点集(记$A^\circ$)是所有包含于$A$的最大开集
\end{proposition}
\begin{proof}
    记
\begin{equation}
A^\circ=\bigcup\{U\mid U\subset A,\; U\text{ 为开集}\},
\end{equation}
则作为开集并的并集 $A^\circ$ 本身为开集,且对任意开集 $U$ 若 $U\subset A$ 则有 $U\subset A^\circ$,因此 $A^\circ$ 是所有包含于 $A$ 的开集中的最大者。
\end{proof}

\begin{remark}
    $B$为$A^\circ$当且仅当$B$为包含于$A$的最大开集
\end{remark}
\begin{proof}
    类似
\end{proof}

\begin{proposition}
    $(X,\tau)$拓扑空间,$A\subset X$,有
    \begin{equation}
        A^\circ=A^{c-c}(\text{习题})
    \end{equation}
\end{proposition}
\begin{proposition}
    $A$为开集当且仅当$A^c$为闭集
\end{proposition}
\begin{proof}
    左推右:若 $A$ 为开集,则对任意 $A^c$ 的极限点 $x$,若反设 $x\notin A^c$ 即 $x\in A$,则由 $A$ 的开性可取开邻域 $U$ 使得 $x\in U\subset A$,从而 $U$ 与 $A^c$ 无交以违背 $x$ 为极限点的假设,因此 $x\in A^c$,即 $A^c$ 含其所有极限点,故 $A^c$ 为闭集。

    右推左:设 $A^c$ 为闭集且 $x\in A$,则 $x\notin A^c$,故 $x$ 不是 $A^c$ 的极限点,因而存在开邻域 $U$ 使得 $U\cap A^c=\varnothing$,从而 $U\subset A$,说明任意点 $x\in A$ 都有包含于 $A$ 的开邻域,即 $A$ 为开集。
\end{proof}

\chapter{连续映射}

本章讨论拓扑空间之间的连续映射及其性质。

\begin{definition}[连续性]\label{def:continuity}设拓扑空间 $(X,\mathcal{T}_X)$ 与 $(Y,\mathcal{T}_Y)$,映射 $f:X\to Y$ 在点 $x\in X$ 处连续,若对任意包含 $f(x)$ 的开集 $V\in\mathcal{T}_{f(x)}$,存在包含 $x$ 的开集 $U\in\mathcal{T}_X$ 使得 $f(U)\subset V$。\end{definition}
\begin{proposition}
映射 $f:X\to Y$ 在每一点连续当且仅当对于任意开集 $V\subset Y$,其原像 $f^{-1}(V)$ 在 $X$ 中为开集
\end{proposition}
\begin{proof}
    右推左:设任意包含 $f(x)$ 的开集 $V\in\mathcal{T}_{f(x)}$,由假设 $f^{-1}(V)$ 在 $X$ 中为开且包含 $x$,于是取 $U=f^{-1}(V)$ 可得包含 $x$ 的开邻域且 $f(U)\subset V$,故 $f$ 在 $x$ 处连续,因 $x$ 为任意点,遂得 $f$ 在每一点连续。
\end{proof}
\begin{proposition}映射 $f:X\to Y$ 在 $X$ 上连续当且仅当对于任意闭集 $C\subset Y$,其原像 $f^{-1}(C)$ 在 $X$ 中为闭集。\end{proposition}
\begin{proposition}
    $f:X\to Y$ 
    \begin{enumerate}
        \item $\forall U\subset Y$,有$f^{-1}(U^c)=f^{-1}(U)^c$
        \item $\forall A,B\subset Y$有,$f^{-1}(A\setminus B)=f^{-1}(A)\setminus f^{-1}(B)$
        \item 任意一簇子集有
\begin{equation}
f^{-1}\Big(\bigcup_{i\in I}A_i\Big)=\bigcup_{i\in I}f^{-1}(A_i),\qquad
f^{-1}\Big(\bigcap_{i\in I}A_i\Big)=\bigcap_{i\in I}f^{-1}(A_i).
\end{equation}
        \item 对于任意族 $\{U_i\}_{i\in I}\subset Y$,有
\begin{equation*}
f^{-1}\Big(\bigcup_{i\in I} U_i\Big)=\bigcup_{i\in I} f^{-1}(U_i).
\end{equation*}
    \item $f(B_1\cap B_2)\subset f(B_1)\cap f(B_2) $
    \end{enumerate}
\end{proposition}
\begin{proposition}
设拓扑空间 $(X,\mathcal{T}_X)$ 与 $(Y,\mathcal{T}_Y)$,映射 $f:X\to Y$,则以下四条等价
    \begin{enumerate} \item 对任意点 $x\in X$,$f$ 在 $x$ 连续; \item 对任意开集 $V\subset Y$,$f^{-1}(V)$ 在 $X$ 中为开集; \item 对任意闭集 $C\subset Y$,$f^{-1}(C)$ 在 $X$ 中为闭集; \item 若$\forall A\subset X$,有
\begin{equation}
f(\overline{A})\subset\overline{f(A)}
\end{equation}. 
    \item $\forall B\subset Y$有 \begin{equation} \overline{f^{-1}(B)}\subset f^{-1}(\overline{B}) \end{equation}.
\end{enumerate}
\end{proposition}
\begin{proof}
    1,2,3相互等价已经证明过了,下面证明3和4等价

    我们先证明3推4,要证$f(\overline{A})\subset\overline{f(A)}$,只要证 $\overline{A}\subset f^{-1}(\overline{f(A)})$,由于$\overline{f(A)}$为闭,对于任意 $a\in A$ 有 $f(a)\in f(A)\subset\overline{f(A)}$,故 $a\in f^{-1}(\overline{f(A)})$,即
\begin{equation}
A\subset f^{-1}(\overline{f(A)})
\end{equation}
而 $f^{-1}(\overline{f(A)})$ 为闭集,故包含 $\overline{A}$,于是 $f(\overline{A})\subset\overline{f(A)}$.

接下来是4推3,那我们任取一个闭集
\begin{equation}
E\subset Y,\quad E\ \text{闭},
\end{equation}
要证$f^{-1}(E)$闭于$X$,只要证$\overline{f^{-1}(E)}\subset f^{-1}(E) $,只要证$f(\overline{f^{-1}(E)})\subset E$,这是因为,由于3对任意集合 $A$ 成立,将 $A=f^{-1}(E)$ 代入得
\begin{equation}
f(\overline{f^{-1}(E)})\subset\overline{f(f^{-1}(E))}\subset\overline{E}=E,
\end{equation}
从而 $\overline{f^{-1}(E)}\subset f^{-1}(E)$,即 $f^{-1}(E)$ 为闭集。
\end{proof}

\chapter{网与收敛}

在一般拓扑空间中,序列的概念不足以刻画收敛性质,我们需要推广到网的概念。

\section{预序集与定向集}

\begin{definition}[预序集]\label{def:preorder}设 $(\Omega,\leq)$ 为集合 $\Omega$ 上带二元关系 $\leq$ 的结构,如果对任意 $x,y,z\in \Omega$ 都有 (i) $x\leq x$(自反性),以及 (ii) $x\leq y$ 且 $y\leq z$ 蕴含 $x\leq z$(传递性),则称 $(\Omega,\leq)$ 为预序集\end{definition}

\begin{example}
    比如说自然数就是一个预序集
\end{example}

\begin{definition}[定向集]\label{def:directed}设 $(\Omega,\leq)$ 为预序集,如果对任意有限子集 $\{x_1,\dots,x_n\}\subset \Omega$ 存在 $y\in D$ 使得 $x_i\leq y$ 对所有 $i=1,\dots,n$ 成立,则称 $(\Omega,\leq)$ 为定向集。\end{definition}
\begin{example}
    领域就是一个定向集
\end{example}

\section{网的定义与收敛}

\begin{definition}[网]\label{def:net}设 $(\Omega,\leq)$ 为定向集,$X$ 为集合,则称任意映射 $x:\Omega\to X$ 为以 $\Omega$ 为下标的网,记作 $(x_{\alpha})_{\alpha\in \Omega}$,并称每个 $\alpha\in \Omega$ 为该网的下标和位置\end{definition}

\begin{definition}[网的收敛]\label{def:net-convergence}设 $(X,\tau)$ 为拓扑空间,$(x_{\alpha})_{\alpha\in \Omega}$ 为以定向集 $(\Omega,\le)$ 为下标的网,则称 $(x_{\alpha})$ 收敛到 $x\in X$(记作 $x_{\alpha}\to x$)若对于任意包含 $x$ 的邻域 $U$ 存在 $\alpha_0\in\Omega$ 使得对所有 $\alpha\in\Omega$ 若 $\alpha\ge \alpha_0$ 则 $x_{\alpha}\in U$,即满足 \begin{equation} \forall U\in\mathcal{N}(x)\;\exists \alpha_0\in\Omega\;\forall \alpha\in\Omega\; (\alpha\ge \alpha_0\Rightarrow x_{\alpha}\in U). \end{equation} \end{definition}

\begin{proposition}
    存在一个以邻域族 $\mathcal{N}(x)$ 为定向集的网 $(x_U)_{U\in\mathcal{N}(x)}$,使得对每个 $U\in\mathcal{N}(x)$ 有 $x_U\in U$,并且该网收敛于 $x$,即
\begin{equation}
\forall U\in\mathcal{N}(x)\; x_U\in U,\quad (x_U)_{U\in\mathcal{N}(x)}\to x.
\end{equation}
\end{proposition}
\begin{proof}
    取定向集为邻域族 $\mathcal{N}(x)$ 并以反包含关系作为偏序(即对 $U,V\in\mathcal{N}(x)$ 令 $U\ge V$ 当且仅当 $U\subseteq V$),对每个 $U\in\mathcal{N}(x)$ 任取 $x_U\in U$,则对于任意邻域 $W\in\mathcal{N}(x)$ 取 $\alpha_0=W$ 有
\begin{equation}
\forall W\in\mathcal{N}(x)\;\exists \alpha_0=W\in\mathcal{N}(x)\;\forall \alpha\in\mathcal{N}(x)\;(\alpha\ge \alpha_0\Rightarrow x_\alpha\in W),
\end{equation}
从而 $(x_U)_{U\in\mathcal{N}(x)}\to x$.
\end{proof}

\begin{proposition}\label{prop:closure-net}设 $X$ 为拓扑空间,$A\subseteq X$,则
\begin{equation}
x\in\overline{A}\iff \exists\ \text{net }(x_\alpha)_{\alpha\in D}\subset A\ \text{ such that }\ (x_\alpha)\to x.
\end{equation}
\end{proposition}
\begin{proof}

     若存在网 $(x_\alpha)_{\alpha\in D}\subset A$ 且 $(x_\alpha)\to x$,则对任一邻域 $W\in\mathcal{N}(x)$ 存在 $\alpha_0$ 使得 $\alpha\ge\alpha_0\Rightarrow x_\alpha\in W$,从而 $W\cap A\neq\varnothing$,即 $x\in\overline{A}$;
     
     反之,若 $x\in\overline{A}$,则对每一邻域 $U\in\mathcal{N}(x)$ 选取一点 $x_U\in A\cap U$,即
\begin{equation}
\forall U\in\mathcal{N}(x)\quad x_U\in A\cap U,
\end{equation}
由此得到网 $(x_U)_{U\in\mathcal{N}(x)}\subset A$,且对于任意 $W\in\mathcal{N}(x)$,当 $U\subset W$ 时有 $x_U\in U\subset W$,因此
\begin{equation}
(x_U)_{U\in\mathcal{N}(x)}\to x,
\end{equation}
从而存在所需的网,证明完成。
\end{proof}

\begin{proposition}
    $f $在 $x$ 连续当且仅当对于任意收敛到 $x$ 的网 $(x_U)_{U\in\mathcal{N}(x)}$ 都有
\begin{equation}
(f(x_U))_{U\in\mathcal{N}(x)}\to f(x).
\end{equation}
\end{proposition}
\begin{proof}
右推左 

我们首先有\begin{equation}
f(\overline{A})\subset\overline{f(A)}
\end{equation}. 因此对于任意收敛到 $x$ 的网 $(x_U)_{U\in\mathcal{N}(x)}$,由于对于任一包含 $x$ 的集合 $A$ 有尾部最终包含于 $\overline{A}$ 并且 $f(\overline{A})\subset\overline{f(A)}$,其像网收敛于 $f(x)$,即
\begin{equation}
(f(x_U))_{U\in\mathcal{N}(x)}\to f(x).
\end{equation}

 左推右:设 $f$ 在 $x$ 处连续,由 $x\in\overline{A}$,知$U\cap A\neq\varnothing$则存在来自 $A$ 的网 $(x_U)_{U\in\mathcal{N}(x)}$ 收敛于 $x$,故由连续性其像网收敛于 $f(x)$,即
\begin{equation}
(f(x_U))_{U\in\mathcal{N}(x)}\to f(x).
\end{equation}

\end{proof}

\chapter{拓扑的生成}

本章讨论如何从给定的集合族生成拓扑。

\section{拓扑族的交}

\begin{proposition}
    $X$为集合,若$(\mathcal{T}_{\alpha})_{\alpha}$为$X$上的一族拓扑,则
\begin{equation}
\mathcal{T}:=\bigcap_{\alpha}\mathcal{T}_{\alpha}
\end{equation}
为 \(X\) 上的拓扑,且为包含所有 \(\mathcal{T}_{\alpha}\) 的最小拓扑,而 \(\bigcup_{\alpha}\mathcal{T}_{\alpha}\) 未必为拓扑。
\end{proposition}
\begin{proof}
具体地,设 $\{U_i\}_{i\in I}\subset\mathcal{T}$,则对任意 $\alpha$ 有 $U_i\in\mathcal{T}_{\alpha}$,且由于每个 $\mathcal{T}_{\alpha} $对任意并运算闭合,我们有
\begin{equation}
\bigcup_{i\in I}U_i\in\mathcal{T}_\alpha\quad(\forall\alpha),
\end{equation}

从而 $\bigcup_{i\in I}U_i\in\bigcap_{\alpha}\mathcal{T}_{\alpha}=\mathcal{T}$,类似地可证有限交运算的闭合性,故 $\mathcal{T}$ 为拓扑且显然为包含所有 $\mathcal{T}_{\alpha} $的最小拓扑,而 $\bigcup_{\alpha}\mathcal{T}_{\alpha}$ 未必为拓扑。
\end{proof}

\section{拓扑基与子基}

\begin{definition}[拓扑子基]\label{def:subbasis}$(X,\mathcal{T})$ 为拓扑空间,若由 $\mathcal S$ $\subset\mathcal{T}$, $\mathcal{T}$为 $\mathcal S$ 的最小拓扑,则称 $\mathcal S$ 为 $(X,\mathcal{T})$上的拓扑子基
\end{definition}
\begin{proposition}
    $f:X\to Y$ ,$X,Y$是拓扑的, $\mathcal S$是$Y$的子基,则下列两个命题等价:\begin{enumerate} \item $f$ 连续; \item 对任意 $S\in\mathcal S$,有 $f^{-1}(S)$开于$X$. 
    \end{enumerate}
\end{proposition}
\begin{proof}
    1$\to$2 显然

    2$\to$1 令
\begin{equation}
\mathcal A=\{\,U\subset Y\mid f^{-1}(U)\text{ 在 }X\text{ 中为开集}\,\}
\end{equation}
只要证明$\mathcal A\supset\mathcal{T}_{Y}$,从而对任意$V\in\mathcal{T}_{Y}$有$f^{-1}(V)$在$X$中开,即$f$连续。

下证$\mathcal A\supset\mathcal{T}_{Y}$,令 $\mathcal B$ 为生成 $\mathcal{T}_{Y}$ 的一组基,任取 $V\in\mathcal{T}_{Y}$,则存在指标集 $I$ 及基元 $\{B_i\}_{i\in I}\subset\mathcal B$ 使得 $V=\bigcup_{i\in I}B_i$,从而
\begin{equation}
f^{-1}(V)=f^{-1}\Big(\bigcup_{i\in I}B_i\Big)=\bigcup_{i\in I}f^{-1}(B_i)\in\mathcal{T}_{X},
\end{equation}

\end{proof}

\begin{proposition}
    若$\mathcal S$ 为拓扑空间 $(X,\mathcal{T})$上的子基,$(x_{\alpha})_{\alpha}$是$X$中的网,$x\in X$,则以下等价
    \begin{enumerate}
        \item $x_{\alpha}\to x$
        \item \begin{equation} \forall U\in\mathcal{N}(x)\text{且}U\in\mathcal S\;\exists \alpha_0\in\Omega,\alpha\ge \alpha_0\Rightarrow x_{\alpha}\in U. \end{equation}
    \end{enumerate}
\end{proposition}
\begin{proof}
    1$\to$2 显然

    2$\to$1  令
\begin{equation}
\mathcal A=\{\,E\in\mathcal{T}\mid \exists \alpha_0\in\Omega,\alpha\ge \alpha_0\Rightarrow x_{\alpha}\in U ,\}
\end{equation}

下证$\mathcal A=\mathcal T$

\begin{enumerate}
    \item \emph{$\mathcal S \subset \mathcal A$且$\mathcal A$包含由$\mathcal S$生成的拓扑:}  
    拓扑$\mathcal T$是由子基$\mathcal S$生成的,因此任意$V\in\mathcal T$包含一个有限交
    \[
    B = S_1 \cap S_2 \cap \cdots \cap S_n,
    \]
    其中每个$S_i\in\mathcal S$且$x\in S_i$。  
    由假设,每个$S_i$在某一$\alpha_i$后包含$x_\alpha$,取$\alpha_0=\max\alpha_i$,则$\forall\alpha\ge\alpha_0$,有$x_\alpha\in B$。因此$B\in\mathcal A$。由于$B\subset V$且$\mathcal A$对超集封闭,得$V\in\mathcal A$,即$\mathcal T\subset\mathcal A$。
    \item \emph{$\mathcal A$是拓扑:}  
    由定义可见,若$E_i \in \mathcal A$,则存在相应的$\alpha_i$使得网最终在$E_i$中。取$\alpha_0 = \max \alpha_i$可得网最终在有限交$\bigcap_i E_i$中;对任意族$\{E_j\}$,网最终在某个$E_j$中即在其并集内,因此$\mathcal A$对有限交和任意并封闭,从而$\mathcal A$是拓扑。
    \item \emph{$\mathcal A\subset\mathcal T$:}  
    由定义$\mathcal A\subset\mathcal T$成立,因为每个$E\in\mathcal A$本身就是拓扑中的开集。
    综上,$\mathcal A=\mathcal T$。
    由$\mathcal A=\mathcal T$可知,对任意邻域$U\in\mathcal N(x)$,都有$U\in\mathcal A$,即网最终进入$U$,这正是$x_\alpha\to x$的定义。
\end{enumerate}
    
\end{proof}
\begin{definition}
    $(X,\mathcal T)$是拓扑空间,$\mathcal B\subset \mathcal T$,若满足对任意$U\in\mathcal{T}$,存在$\mathcal{C}\subset\mathcal{B}$使得
\begin{equation}
U = \bigcup_{B \in \mathcal{C}} B\text{也记作}U=U\mathcal{C}
\end{equation}
成立,则称$\mathcal{B}$是$(X,\mathcal{T})$的拓扑基
\end{definition}
\begin{example}
    $(X,d)$为度量空间,定义$\mathcal{B}=\{B(x,\varepsilon):x\in X,\varepsilon>0\}$则$\mathcal{B}$为$(X,\mathcal{T}_d)$的一个基。其中$\mathcal{T}_d$表示由度量$d$所诱导的拓扑,即由所有开球$B(x,\varepsilon)$生成的拓扑。
\end{example}
\begin{proof}
设$U \in \mathcal{T}_d$,则按定义$U$为$d$-开集,即
\[
\forall x \in U, \ \exists \ \varepsilon_x > 0 \ \text{使得} \ B(x,\varepsilon_x) \subset U.
\]
由此我们有
\begin{equation}
U = \bigcup_{x\in U} B(x,\varepsilon_x).
\end{equation}
显然每个$B(x,\varepsilon_x) \in \mathcal{B}$,且该并为$U$的覆盖,故$\mathcal{B}$为$(X,\mathcal{T}_d)$的一个基。
\end{proof}
\begin{proposition}
    若$\mathcal{B}$是$(X,\mathcal{T})$的基,$\mathcal{B}$是$(X,\mathcal{T})$的子基
\end{proposition}
\begin{proof}
由假设$\mathcal{B} \subset \mathcal{T}$,且$\mathcal{T}$中任意开集均为$\mathcal{B}$中元素的并。设$\mathcal{T}'$是由$\mathcal{B}$生成的拓扑,则:
\begin{enumerate}
    \item 显然$\mathcal{B} \subset \mathcal{T}'$,且$\mathcal{T}' \subset \mathcal{T}$;
    \item 对任意$U \in \mathcal{T}$,由基的定义存在$\mathcal{C} \subset \mathcal{B}$使得
    \begin{equation}
        U = \bigcup_{B \in \mathcal{C}} B,
    \end{equation}
    而这些$B$均在$\mathcal{T}'$中,因此$U \in \mathcal{T}'$。
\end{enumerate}
由此得$\mathcal{T} \subset \mathcal{T}'$且$\mathcal{T}' \subset \mathcal{T}$,故$\mathcal{T} = \mathcal{T}'$,从而$\mathcal{B}$亦为$(X,\mathcal{T})$的一个子基。
\end{proof}

\begin{proposition}
    若$\mathcal{S}$为$(X,\mathcal{T})$的子基,则$\mathcal{S}_{f\cap}\cup\{x\}$为$X$的一个基.这里$\mathcal{S}_{f\cap}$为由 $\mathcal{S}$ 取有限交得到的所有集合
\end{proposition}
\begin{proof}
    1. 由 $\mathcal{S}$ 是 $(X,\mathcal{T})$ 的子基可知,$\mathcal{S}_{f\cap} \subset \mathcal{T}$ 且 $X \in \mathcal{T}$,从而 $\mathcal{S}_{f\cap} \cup \{X\} \subset \mathcal{T}$。

2. 设
\[
\mathcal{A} \triangleq \{ E \subseteq X \mid \exists \mathcal{B}' \subset \mathcal{S}_{f\cap} \cup \{X\} \ \text{使得}\ E = \bigcup_{B \in \mathcal{B}'} B \},
\]
则由子基定义知 $\mathcal{T} \subseteq \mathcal{A}$,反之由第1步知 $\mathcal{A} \subseteq \mathcal{T}$,故 $\mathcal{A} = \mathcal{T}$。

下证 $\mathcal{S}_{f\cap} \cup \{X\}$ 满足基的判别条件,只要证 $\mathcal{A}$ 为由 $\mathcal{S}$ 生成的子拓扑,具体如下:
\begin{enumerate}
    \item 由于 $\mathcal{S} \subseteq \mathcal{S}_{f\cap} \subseteq \mathcal{S}_{f\cap} \cup \{X\}$,对任意 $E \in \mathcal{S}$,有 $E = E \cap E$,且 $E = \bigcup\{E\}$,因此 $E \in \mathcal{A}$,即 $\mathcal{S} \subseteq \mathcal{A}$。
    \item 空集 $\varnothing$ 可以表示为 $\varnothing = \bigcup \varnothing \in \mathcal{A}$,而 $X = \bigcup\{X\} \in \mathcal{A}$;此外,$\mathcal{A}$ 对任意并封闭且对有限交封闭,故 $\mathcal{A}$ 为包含 $\mathcal{S}$ 的拓扑。
\end{enumerate}
\end{proof}

\chapter{分离公理}

我们现在引入分离公理的概念,用以刻画拓扑空间中点与点之间的"可分离性"。

\begin{definition}[$T_0$ 空间(可分离空间)]\label{def:T0}
设 $(X,\mathcal{T})$ 为拓扑空间,若对任意 $x, y \in X$ 且 $x \neq y$,
\begin{equation}
\exists U \in \mathcal{N}(x) \text{ 使得 } y \notin U \quad \text{或} \quad \exists V \in \mathcal{N}(y) \text{ 使得 } x \notin V,
\end{equation}
则称 $(X,\mathcal{T})$ 为 $T_0$ 空间(或可分离空间)。
\end{definition}

\begin{definition}[$T_1$ 空间]\label{def:T1}
设 $(X,\mathcal{T})$ 为拓扑空间,若对任意 $x, y \in X$ 且 $x \neq y$,
\begin{equation}
\exists U \in \mathcal{N}(x) \text{ 使得 } y \notin U,
\end{equation}
则称 $(X,\mathcal{T})$ 为 $T_1$ 空间。
\end{definition}

\begin{remark}
$T_1$ 空间等价于:对任意 $x, y \in X$ 且 $x \neq y$,
\begin{equation}
\exists U \in \mathcal{N}(x) \text{ 使得 } y \notin U \quad \text{且} \quad \exists V \in \mathcal{N}(y) \text{ 使得 } x \notin V.
\end{equation}
\end{remark}

\begin{proposition}\label{prop:T1-characterization}
设 $(X,\mathcal{T})$ 为拓扑空间,则以下条件等价:
\begin{enumerate}
    \item $X$ 为 $T_1$ 空间;
    \item $X$ 的每个单点集为闭集;
    \item $X$ 的每个有限子集为闭集。
\end{enumerate}
\end{proposition}

\begin{definition}[$T_2$ 空间 (Hausdorff)]\label{def:T2}
设 $(X,\mathcal{T})$ 为拓扑空间,若对任意 $x, y \in X$ 且 $x \neq y$,
\begin{equation}
\exists U \in \mathcal{N}(x), \, \exists V \in \mathcal{N}(y) \quad \text{使得} \quad U \cap V = \varnothing,
\end{equation}
则称 $(X,\mathcal{T})$ 为 $T_2$ 空间(或 Hausdorff 空间)。
\end{definition}

\begin{remark}
显然有如下包含关系:
\begin{equation}
T_2 \Rightarrow T_1 \Rightarrow T_0.
\end{equation}
即任何 Hausdorff 空间都是 $T_1$ 空间,任何 $T_1$ 空间都是 $T_0$ 空间。
\end{remark}

\begin{proposition}\label{prop:T2-unique-limit}
设 $(X,\mathcal{T})$ 为 $T_2$ 空间,则 $X$ 中任意网的极限若存在必唯一。即若网 $(x_\alpha)_{\alpha\in\Omega}$ 满足 $x_\alpha \to x$ 且 $x_\alpha \to y$,则 $x = y$。
\end{proposition}
\begin{proof}
反证法。假设 $x \neq y$,由于 $(X,\mathcal{T})$ 为 $T_2$ 空间,存在 $U \in \mathcal{N}(x)$ 和 $V \in \mathcal{N}(y)$ 使得
\begin{equation}
U \cap V = \varnothing.
\end{equation}

由于 $x_\alpha \to x$,存在 $\alpha_1 \in \Omega$ 使得对所有 $\alpha \geq \alpha_1$,有 $x_\alpha \in U$;

由于 $x_\alpha \to y$,存在 $\alpha_2 \in \Omega$ 使得对所有 $\alpha \geq \alpha_2$,有 $x_\alpha \in V$。

由定向集的性质,存在 $\alpha_0 \in \Omega$ 使得 $\alpha_0 \geq \alpha_1$ 且 $\alpha_0 \geq \alpha_2$,从而
\begin{equation}
x_{\alpha_0} \in U \quad \text{且} \quad x_{\alpha_0} \in V,
\end{equation}
即 $x_{\alpha_0} \in U \cap V$。但这与 $U \cap V = \varnothing$ 矛盾。

因此必有 $x = y$,即极限唯一。
\end{proof}

\begin{proposition}\label{prop:T2-diagonal}
设 $X$ 为拓扑空间,则以下条件等价:
\begin{enumerate}
    \item $X$ 为 $T_2$ 空间;
    \item $X$ 具有网收敛的唯一性。
\end{enumerate}
\end{proposition}
\begin{proof}
$(1) \Rightarrow (2)$:由命题 \ref{prop:T2-unique-limit} 已证。

$(2) \Leftarrow (1)$:反证法。假设 $X$ 不为 $T_2$ 空间,则存在 $x, y \in X$ 且 $x \neq y$,使得对任意 $U \in \mathcal{N}(x)$ 和 $V \in \mathcal{N}(y)$,都有
\begin{equation}
U \cap V \neq \varnothing.
\end{equation}

令
\begin{equation}
\Lambda = \{U \cap V : U \in \mathcal{N}(x), V \in \mathcal{N}(y), U \cap V \neq \varnothing\}.
\end{equation}

对于 $\Lambda$ 中任意两个元素 $A, B \in \Lambda$,设 $A = U_1 \cap V_1$,$B = U_2 \cap V_2$,其中 $U_1, U_2 \in \mathcal{N}(x)$,$V_1, V_2 \in \mathcal{N}(y)$。令 $C = (U_1 \cap U_2) \cap (V_1 \cap V_2)$,则 $C \in \Lambda$ 且 $C \subseteq A$ 且 $C \subseteq B$。因此 $\Lambda$ 在反包含序下构成定向集。

对每个 $A \in \Lambda$,选取 $z_A \in A$,得到网 $(z_A)_{A \in \Lambda}$。

对任意 $U_0 \in \mathcal{N}(x)$,取 $A_0 = U_0 \cap X \in \Lambda$(因为 $x \in U_0$ 且 $y \in X$),则当 $A \subseteq A_0$ 时,有 $z_A \in A \subseteq A_0 \subseteq U_0$,故 $z_A \to x$。

同理,对任意 $V_0 \in \mathcal{N}(y)$,取 $B_0 = X \cap V_0 \in \Lambda$,可得 $z_A \to y$。

因此该网同时收敛到 $x$ 和 $y$,但 $x \neq y$,这与网收敛的唯一性矛盾。

故 $X$ 必为 $T_2$ 空间。
\end{proof}

\section{滤子与滤基}

\begin{definition}[有限交性质]\label{def:finite-intersection}
设 $X$ 为集合,$\mathcal{F} \subseteq \mathcal{P}(X)$。

称 $\mathcal{F}$ 具有有限交性质,若对任意 $E \subseteq \mathcal{F}$ 且 $E$ 有限,存在 $x \in X$ 使得 $x \in \bigcap E$,即
\begin{equation}
\bigcap_{A \in E} A \neq \varnothing.
\end{equation}
\end{definition}

\begin{definition}[滤子基]\label{def:filter-base}
设 $X$ 为集合,$\mathcal{F} \subseteq \mathcal{P}(X)$。

若 $\mathcal{F}$ 满足:
\begin{enumerate}
    \item $\varnothing \notin \mathcal{F}$;
    \item 对任意 $A, B \in \mathcal{F}$,存在 $C \in \mathcal{F}$ 使得 $C \subseteq A \cap B$,
\end{enumerate}
则称 $\mathcal{F}$ 为 $X$ 上的一个滤子基。
\end{definition}

\begin{definition}[滤子]\label{def:filter}
设 $X$ 为集合,$\mathcal{F} \subseteq \mathcal{P}(X)$。

若 $\mathcal{F}$ 满足:
\begin{enumerate}
    \item $\varnothing \notin \mathcal{F}$;
    \item 对任意 $A, B \in \mathcal{F}$,有 $A \cap B \in \mathcal{F}$;
    \item (向上封闭)对任意 $A \in \mathcal{F}$ 和 $B \subseteq X$,若 $B \supseteq A$,则 $B \in \mathcal{F}$,
\end{enumerate}
则称 $\mathcal{F}$ 为 $X$ 上的一个滤子。
\end{definition}

\begin{definition}[邻域基]\label{def:neighborhood-base}
设 $(X,\tau)$ 为拓扑空间,$x \in X$,$\mathcal{B} \subseteq \mathcal{N}(x)$。

称 $\mathcal{B}$ 为点 $x$ 的一个邻域基,若对任意 $U \in \mathcal{N}(x)$,存在 $B \in \mathcal{B}$ 使得
\begin{equation}
B \subseteq U.
\end{equation}
\end{definition}

\begin{proposition}\label{prop:neighborhood-base-filter-base}
设 $(X,\tau)$ 为拓扑空间,$x \in X$,$\mathcal{B} \subseteq \mathcal{N}(x)$。则 $\mathcal{B}$ 是点 $x$ 的邻域基当且仅当 $\mathcal{B}$ 是一个滤子基。
\end{proposition}

\begin{definition}[滤子收敛]\label{def:filter-convergence}
设 $(X,\tau)$ 为拓扑空间,$\mathcal{F} \subseteq \mathcal{P}(X)$,$x \in X$。

称 $\mathcal{F}$ 收敛到 $x$,记作 $\mathcal{F} \to x$,若对任意 $U \in \mathcal{N}(x)$,存在 $F \in \mathcal{F}$ 使得
\begin{equation}
F \subseteq U.
\end{equation}
\end{definition}

\begin{proposition}\label{prop:finite-intersection-unique-limit}
设 $X$ 为 $T_2$ 空间,$\mathcal{F}$ 为 $X$ 中具有有限交性质的集族。则 $\mathcal{F}$ 的收敛点若存在必唯一,即若 $\mathcal{F} \to x$ 且 $\mathcal{F} \to y$,则 $x = y$。
\end{proposition}
\begin{proof}
反证法。假设 $x \neq y$,由于 $X$ 为 $T_2$ 空间,存在 $U \in \mathcal{N}(x)$ 和 $V \in \mathcal{N}(y)$ 使得
\begin{equation}
U \cap V = \varnothing.
\end{equation}

由于 $\mathcal{F} \to x$,存在 $A \in \mathcal{F}$ 使得 $A \subseteq U$;

由于 $\mathcal{F} \to y$,存在 $B \in \mathcal{F}$ 使得 $B \subseteq V$。

因此 $A \cap B \subseteq U \cap V = \varnothing$,即 $A \cap B = \varnothing$。

但这与 $\mathcal{F}$ 具有有限交性质矛盾(因为 $A, B \in \mathcal{F}$ 是有限子族,其交应非空)。

故必有 $x = y$,收敛点唯一。
\end{proof}

\begin{proposition}\label{prop:T2-equivalences}
设 $X$ 为拓扑空间,则下列命题等价:
\begin{enumerate}
    \item $X$ 为 $T_2$ 空间;
    \item $X$ 中网收敛的极限唯一性;
    \item $X$ 中满足有限交性质的集族收敛唯一性;
    \item $X$ 中滤子基收敛唯一性;
    \item $X$ 中滤子收敛唯一性。
\end{enumerate}
\end{proposition}
\begin{proof}
我们按照 $(1)\Leftrightarrow(2)$,$(1)\Rightarrow(3)\Rightarrow(4)\Leftrightarrow(5)$,$(4)\Rightarrow(2)$ 的顺序证明。

$(1)\Leftrightarrow(2)$:由命题 \ref{prop:T2-diagonal} 已证。

$(1)\Rightarrow(3)$:由命题 \ref{prop:finite-intersection-unique-limit} 已证。

$(3)\Rightarrow(4)$:显然,因为滤子基是具有有限交性质的集族。

$(4)\Leftrightarrow(5)$:设 $\mathcal{B}$ 为滤子基,则 $\mathcal{B}_{\uparrow}$ 为滤子。若 $\mathcal{B} \to x$ 且 $\mathcal{B} \to y$,则 $\mathcal{B}_{\uparrow} \to x$ 且 $\mathcal{B}_{\uparrow} \to y$。由滤子收敛唯一性得 $x = y$。反之,滤子本身也是滤子基,故滤子收敛唯一性蕴含滤子基收敛唯一性。

$(4)\Rightarrow(2)$:设 $(x_\alpha)_{\alpha\in\Omega}$ 为 $X$ 中的网,且 $x_\alpha \to x$ 且 $x_\alpha \to y$。

令 $\mathcal{B} = \{\{x_\beta : \beta \geq \alpha\} : \alpha \in \Omega\}$,即 $E_\alpha = \{x_\beta : \beta \geq \alpha\}$。

可以验证 $\mathcal{B}$ 是一个滤子基:
\begin{itemize}
    \item $\varnothing \notin \mathcal{B}$,因为每个 $E_\alpha$ 至少包含 $x_\alpha$;
    \item 对任意 $E_\alpha, E_\beta \in \mathcal{B}$,由定向集性质存在 $\gamma \geq \alpha, \beta$,则 $E_\gamma \subseteq E_\alpha \cap E_\beta$。
\end{itemize}

若 $x_\alpha \to x$,则对任意 $U \in \mathcal{N}(x)$,存在 $\alpha_0$ 使得对所有 $\alpha \geq \alpha_0$ 有 $x_\alpha \in U$,即 $E_{\alpha_0} \subseteq U$,故 $\mathcal{B} \to x$。

同理 $\mathcal{B} \to y$。由滤子基收敛唯一性,$x = y$。

因此所有命题等价。
\end{proof}

\begin{definition}\label{def:topology-generated-by-base}
设 $X$ 为集合,$\mathcal{B} \subseteq \mathcal{P}(X)$。定义
\begin{equation}
\mathcal{B}_{\uparrow} = \{\,E \subseteq X : \exists B \in \mathcal{B} \text{ 使得 } B \subseteq E\,\}.
\end{equation}
\end{definition}

\begin{proposition}\label{prop:filter-base-generates-filter}
设 $\mathcal{B}$ 为 $X$ 上的一个滤子基,则 $\mathcal{B}_{\uparrow}$ 为 $X$ 上的一个滤子。
\end{proposition}
\begin{proof}
需要验证 $\mathcal{B}_{\uparrow}$ 满足滤子的三个条件。

(1) 若 $\varnothing \in \mathcal{B}_{\uparrow}$,则存在 $B \in \mathcal{B}$ 使得 $B \subseteq \varnothing$,即 $B = \varnothing$,这与 $\varnothing \notin \mathcal{B}$ 矛盾。故 $\varnothing \notin \mathcal{B}_{\uparrow}$。

(2) 对任意 $E, F \in \mathcal{B}_{\uparrow}$,存在 $B_1 \in \mathcal{B}$ 使得 $B_1 \subseteq E$,存在 $B_2 \in \mathcal{B}$ 使得 $B_2 \subseteq F$。

由于 $\mathcal{B}$ 为滤子基,存在 $B \in \mathcal{B}$ 使得 $B \subseteq B_1 \cap B_2$。

因此 $B \subseteq B_1 \cap B_2 \subseteq E \cap F$,即 $E \cap F \in \mathcal{B}_{\uparrow}$。

(3) (向上封闭)若 $E \in \mathcal{B}_{\uparrow}$ 且 $F \supseteq E$,则存在 $B \in \mathcal{B}$ 使得 $B \subseteq E \subseteq F$,故 $F \in \mathcal{B}_{\uparrow}$。

因此 $\mathcal{B}_{\uparrow}$ 为 $X$ 上的一个滤子。
\end{proof}

\begin{remark}
此时称 $\mathcal{B}_{\uparrow}$ 为包含 $\mathcal{B}$ 的最小滤子,或者说 $\mathcal{B}$ 生成滤子 $\mathcal{B}_{\uparrow}$。
\end{remark}

\section{正则空间与正规空间}

\begin{definition}[正则空间]\label{def:regular-space}
设 $X$ 为拓扑空间,称 $X$ 为正则空间,若对任意 $x \in X$ 和任意闭集 $E \subseteq X$,若 $x \notin E$,则存在开集 $U, V$ 使得
\begin{equation}
x \in U, \quad E \subseteq V \quad \text{且} \quad U \cap V = \varnothing.
\end{equation}
\end{definition}

\begin{proposition}\label{prop:regular-T1-implies-T2}
正则 $+ T_1 \Rightarrow T_2$,即:若 $X$ 为正则空间且为 $T_1$ 空间,则 $X$ 为 $T_2$ 空间。
\end{proposition}
\begin{proof}
设 $X$ 为正则且 $T_1$ 的空间,$x, y \in X$ 且 $x \neq y$。

由于 $X$ 为 $T_1$ 空间,$\{y\}$ 为闭集。因为 $x \notin \{y\}$,由正则性,存在开集 $U, V$ 使得
\begin{equation}
x \in U, \quad \{y\} \subseteq V, \quad \text{且} \quad U \cap V = \varnothing.
\end{equation}

因此 $x \in U$,$y \in V$,且 $U \cap V = \varnothing$,故 $X$ 为 $T_2$ 空间。
\end{proof}

\begin{definition}[$T_3$ 空间]\label{def:T3-space}
设 $X$ 为拓扑空间,称 $X$ 为 $T_3$ 空间,若 $X$ 既是正则空间又是 $T_1$ 空间。
\end{definition}

\begin{remark}
由命题 \ref{prop:regular-T1-implies-T2},$T_3 \Rightarrow T_2$。因此分离公理的层次为:$T_3 \Rightarrow T_2 \Rightarrow T_1 \Rightarrow T_0$。
\end{remark}

\begin{proposition}\label{prop:regular-characterization}
$X$ 为正则空间当且仅当对任意 $x \in X$,对任意 $V \in \mathcal{N}(x)$,存在 $U \in \mathcal{N}(x)$ 使得 $x \in U \subseteq \overline{U} \subseteq V$。
\end{proposition}
\begin{proof}
$(\Rightarrow)$:设 $X$ 为正则空间,$x \in X$,$V \in \mathcal{N}(x)$。

不妨设 $V$ 为开集(否则取 $V$ 的开邻域)。令 $E = V^c$,则 $E$ 为闭集且 $x \notin E$。

由正则性,存在开集 $U, W$ 使得 $x \in U$,$E \subseteq W$,且 $U \cap W = \varnothing$。

因为 $U \cap W = \varnothing$,所以 $U \subseteq W^c$。因此 $\overline{U} \subseteq \overline{W^c}$。

又因为 $W$ 为开集,所以 $W^c$ 为闭集,故 $\overline{W^c} = W^c$。

因为 $E \subseteq W$,所以 $W^c \subseteq E^c = V$。

综上,$x \in U \subseteq \overline{U} \subseteq W^c \subseteq V$。

$(\Leftarrow)$:设条件成立,即对任意 $x \in X$ 和 $V \in \mathcal{N}(x)$,存在 $U \in \mathcal{N}(x)$ 使得 $x \in U \subseteq \overline{U} \subseteq V$。

设 $x \in X$,$E$ 为闭集且 $x \notin E$。则 $V = E^c$ 为开集且 $x \in V$。

由条件,存在开邻域 $U$ 使得 $x \in U \subseteq \overline{U} \subseteq V = E^c$。

令 $W = (\overline{U})^c$,则 $W$ 为开集。

因为 $\overline{U} \subseteq E^c$,所以 $E \subseteq (\overline{U})^c = W$。

又因为 $U \cap W = U \cap (\overline{U})^c = \varnothing$。

因此 $x \in U$,$E \subseteq W$,且 $U \cap W = \varnothing$,故 $X$ 为正则空间。
\end{proof}

\begin{definition}[正规空间]\label{def:normal-space}
设 $X$ 为拓扑空间,称 $X$ 为正规空间,若对任意 $A, B$ 为 $X$ 中不相交的闭集,即 $A \cap B = \varnothing$,存在 $U, V$ 为 $X$ 中不相交的开集,使得
\begin{equation}
A \subseteq U \quad \text{且} \quad B \subseteq V.
\end{equation}
\end{definition}

\begin{proposition}\label{prop:normal-characterization}
$X$ 为正规空间当且仅当对任意闭集 $A \subseteq X$,对任意开集 $V \supseteq A$,存在开集 $U$ 使得 $A \subseteq U \subseteq \overline{U} \subseteq V$。
\end{proposition}
\begin{proof}
$(\Rightarrow)$:设 $X$ 为正规空间,$A$ 为闭集,$V$ 为开集且 $A \subseteq V$。

令 $B = V^c$,则 $B$ 为闭集,且 $A \cap B = \varnothing$。

由正规性,存在开集 $U, W$ 使得 $A \subseteq U$,$B \subseteq W$,且 $U \cap W = \varnothing$。

因为 $U \cap W = \varnothing$,所以 $U \subseteq W^c$。因此 $\overline{U} \subseteq \overline{W^c}$。

又因为 $W$ 为开集,所以 $W^c$ 为闭集,故 $\overline{W^c} = W^c$。

因为 $B \subseteq W$,所以 $W^c \subseteq B^c = V$。

综上,$A \subseteq U \subseteq \overline{U} \subseteq W^c \subseteq V$。

$(\Leftarrow)$:设条件成立,即对任意闭集 $A$ 和开集 $V \supseteq A$,存在开集 $U$ 使得 $A \subseteq U \subseteq \overline{U} \subseteq V$。

设 $A, B$ 为不相交的闭集,即 $A \cap B = \varnothing$。则 $V = B^c$ 为开集且 $A \subseteq V$。

由条件,存在开集 $U$ 使得 $A \subseteq U \subseteq \overline{U} \subseteq V = B^c$。

令 $W = (\overline{U})^c$,则 $W$ 为开集。

因为 $\overline{U} \subseteq B^c$,所以 $B \subseteq (\overline{U})^c = W$。

又因为 $U \cap W = U \cap (\overline{U})^c = \varnothing$。

因此 $A \subseteq U$,$B \subseteq W$,且 $U \cap W = \varnothing$,故 $X$ 为正规空间。
\end{proof}

\begin{definition}[$T_4$ 空间]\label{def:T4-space}
设 $X$ 为拓扑空间,称 $X$ 为 $T_4$ 空间,若 $X$ 既是正规空间又是 $T_1$ 空间。
\end{definition}

\begin{proposition}\label{prop:normal-T1-implies-T3}
正规 $+ T_1 \Rightarrow T_3$,即:若 $X$ 为正规空间且为 $T_1$ 空间,则 $X$ 为正则空间。
\end{proposition}
\begin{proof}
设 $X$ 为正规且 $T_1$ 的空间,$x \in X$,$E$ 为闭集且 $x \notin E$。

由于 $X$ 为 $T_1$ 空间,$\{x\}$ 为闭集。因为 $\{x\} \cap E = \varnothing$,由正规性,存在开集 $U, V$ 使得
\begin{equation}
\{x\} \subseteq U, \quad E \subseteq V, \quad \text{且} \quad U \cap V = \varnothing.
\end{equation}

因此 $x \in U$,$E \subseteq V$,且 $U \cap V = \varnothing$,故 $X$ 为正则空间。

由定义,$X$ 也是 $T_1$ 空间,因此 $X$ 为 $T_3$ 空间。
\end{proof}

\chapter{紧致性}

\section{紧空间}
\begin{definition}[紧空间]\label{def:compact-space}
设 $X$ 为拓扑空间,称 $X$ 为紧空间,若对任意 $\mathcal{U}$ 为 $X$ 的开覆盖(即 $\mathcal{U}$ 为开集族且 $\bigcup \mathcal{U} = X$),存在 $\mathcal{V} \subseteq \mathcal{U}$ 使得 $\mathcal{V}$ 为 $X$ 的覆盖且 $\mathcal{V}$ 为有限集(即存在有限子覆盖)。
\end{definition}

\begin{definition}[Lindelöff 空间]\label{def:lindeloff-space}
设 $X$ 为拓扑空间,称 $X$ 为 Lindelöff 空间,若对任意开覆盖,存在可数子覆盖。
\end{definition}

\begin{remark}
Lindelöff 空间有时也称为完全可数紧空间。其定义等价于:任意开覆盖 $\{U_\alpha\}_{\alpha \in A}$ 都存在可数子集 $\{U_{\alpha_n}\}_{n \in \mathbb{N}}$ 使得 $\bigcup_{n \in \mathbb{N}} U_{\alpha_n} = X$。
\end{remark}

\begin{definition}[可数紧空间]\label{def:countably-compact-space}
设 $(X,\tau)$ 为拓扑空间,称 $(X,\tau)$ 为可数紧空间,若对任意可数开覆盖,存在有限子覆盖(即 $\text{CA}_2$)。
\end{definition}

\begin{definition}[序列紧空间]\label{def:sequentially-compact-space}
设 $X$ 为拓扑空间,称 $X$ 为序列紧空间,若 $X$ 中的每个序列 $(x_n)_{n \in \mathbb{N}}$ 都有收敛子序列。
\end{definition}

\begin{proposition}\label{prop:CA2-implies-lindeloff}
$\text{CA}_2 \Rightarrow$ Lindelöff 空间,即可数紧空间必为 Lindelöff 空间。
\end{proposition}
\begin{proof}
设 $X$ 为可数紧空间,$\mathcal{U}$ 为 $X$ 的任意开覆盖。

由已知,$X$ 为 $\text{CA}_2$ 空间,即 $X$ 为第二可数空间,存在可数拓扑基 $\mathcal{B}$。

由命题 \ref{prop:base-property},存在 $\overline{\mathcal{B}} \subseteq \mathcal{B}$ 使得 $\overline{\mathcal{B}}$ 覆盖 $X$ 且 $\overline{\mathcal{B}}$ 加细 $\mathcal{U}$。

因为 $\mathcal{B}$ 可数,所以 $\overline{\mathcal{B}}$ 也可数。

由引理 \ref{lem:refinement-cover},由于 $\overline{\mathcal{B}}$ 可数且加细 $\mathcal{U}$,存在 $\mathcal{U}$ 的可数子族覆盖 $X$。

因此 $X$ 为 Lindelöff 空间。
\end{proof}

\begin{proposition}\label{prop:CA2-implies-separable}
$\text{CA}_2 \Rightarrow$ 可分,即第二可数空间必为可分空间。
\end{proposition}
\begin{proof}
设 $X$ 为第二可数空间,$\mathcal{B}$ 为 $X$ 的可数拓扑基,且 $\phi \neq \mathcal{B}$(即 $\mathcal{B}$ 非空)。

对每个 $B \in \mathcal{B}$,取 $\alpha_B \in B$。

令 $D = \{\alpha_B : B \in \mathcal{B}\}$。

则 $D$ 为可数集(可数集 $\mathcal{B}$ 的像)。

下证 $\overline{D} = X$。

对任意 $x \in X$ 和 $x$ 的任意开邻域 $U$,由于 $\mathcal{B}$ 为拓扑基,存在 $B \in \mathcal{B}$ 使得 $x \in B \subseteq U$。

因为 $\alpha_B \in B \subseteq U$,所以 $\alpha_B \in U \cap D$,即 $U \cap D \neq \varnothing$。

因此 $x \in \overline{D}$。

由 $x$ 的任意性,$\overline{D} = X$。

故 $X$ 为可分空间。
\end{proof}

\chapter{完全正则空间与 Urysohn 引理}

\section{完全正则空间}

\begin{definition}[完全正则空间]\label{def:completely-regular-space}
设 $X$ 为拓扑空间,称 $X$ 为完全正则空间,若对任意 $x \in X$ 和闭集 $F \subseteq X$ 且 $x \notin F$,存在连续函数 $f: X \to [0,1]$ 使得 $f(x) = 0$ 且 $f(F) \subseteq \{1\}$。
\end{definition}

\begin{proposition}\label{prop:completely-regular-implies-regular}
完全正则的 $\Rightarrow$ 正则,即:若 $X$ 为完全正则空间,则 $X$ 为正则空间。
\end{proposition}
\begin{proof}
设 $x \in X$,$F$ 为闭集且 $x \notin F$。

由完全正则性,存在连续函数 $f: X \to [0,1]$ 使得 $f(x) = 0$ 且 $f(F) \subseteq \{1\}$。

令
\begin{equation}
U = f^{-1}\left(\left[0, \frac{1}{2}\right)\right), \quad V = f^{-1}\left(\left(\frac{1}{2}, 1\right]\right).
\end{equation}

因为 $f$ 连续,$\left[0, \frac{1}{2}\right)$ 和 $\left(\frac{1}{2}, 1\right]$ 在 $[0,1]$ 中为开集,故 $U, V$ 为开集。

因为 $f(x) = 0 \in \left[0, \frac{1}{2}\right)$,所以 $x \in U$。

因为 $f(F) \subseteq \{1\} \subseteq \left(\frac{1}{2}, 1\right]$,所以 $F \subseteq V$。

因为 $\left[0, \frac{1}{2}\right) \cap \left(\frac{1}{2}, 1\right] = \varnothing$,所以 $U \cap V = \varnothing$。

因此 $X$ 为正则空间。
\end{proof}

\begin{definition}[$T_{3\frac{1}{2}}$ 空间]\label{def:T3.5-space}
设 $X$ 为拓扑空间,称 $X$ 为 $T_{3\frac{1}{2}}$ 空间,若 $X$ 既是完全正则空间又是 $T_1$ 空间。
\end{definition}

\begin{definition}
设 $X$ 为拓扑空间,称 $X$ 为完全正规空间,若对任意 $A, B$ 为不相交闭集,存在连续函数 $f: X \to [0,1]$ 使得 $f(A) \subseteq \{0\}$ 且 $f(B) \subseteq \{1\}$,则称 $X$ 满足 Urysohn 可分离性。
\end{definition}

\begin{proposition}\label{prop:urysohn-implies-normal}
若 $X$ 满足 Urysohn 可分离性,则 $X$ 正规。
\end{proposition}
\begin{proof}
设 $A, B$ 为不相交闭集。

由 Urysohn 可分离性,存在连续函数 $f: X \to [0,1]$ 使得 $f(A) \subseteq \{0\}$ 且 $f(B) \subseteq \{1\}$。

令
\begin{equation}
U = f^{-1}\left(\left[0, \frac{1}{2}\right)\right), \quad V = f^{-1}\left(\left(\frac{1}{2}, 1\right]\right).
\end{equation}

因为 $f$ 连续,$\left[0, \frac{1}{2}\right)$ 和 $\left(\frac{1}{2}, 1\right]$ 在 $[0,1]$ 中为开集,故 $U, V$ 为开集。

因为 $f(A) \subseteq \{0\} \subseteq \left[0, \frac{1}{2}\right)$,所以 $A \subseteq U$。

因为 $f(B) \subseteq \{1\} \subseteq \left(\frac{1}{2}, 1\right]$,所以 $B \subseteq V$。

因为 $\left[0, \frac{1}{2}\right) \cap \left(\frac{1}{2}, 1\right] = \varnothing$,所以 $U \cap V = \varnothing$。

因此 $X$ 为完全正则空间。
\end{proof}

\section{Urysohn 引理}

\begin{lemma}[Urysohn 引理的构造]\label{lem:urysohn-construction}
设 $\Lambda \to [0,1]$ 为有序对等,且 $\lambda \in \Lambda$。设 $X$ 为拓扑空间,$(U_\alpha)_{\alpha \in \Lambda}$ 为 $X$ 中一族开集满足: 对任意 $\alpha, \beta \in \Lambda$,若 $\alpha < \beta$,则 $U_\alpha \subseteq \overline{U_\alpha} \subseteq U_\beta$,且 $U_1 = X$。
则定义 $f: X \to [0,1]$  $\forall x$,
\begin{equation}
f(x) = \inf\{\alpha \in \Lambda : x \in U_\alpha\}
\end{equation}
为连续函数。
\end{lemma}
\begin{proof}
设 $c \in [0,1]$,需证明 $f^{-1}([0,c))$ 与 $f^{-1}((c,1])$ 均为 $X$ 中开集。

$\textbf{证明 $f^{-1}([0,c))$ 为开集:}$

\begin{align}
\{x : f(x) < c\} &= \{x : \exists \alpha \in \Lambda \text{ 且 } \alpha < c \text{ 使得 } x \in U_\alpha\} \\
&= \bigcup_{\substack{\alpha \in \Lambda \\ \alpha < c}} U_\alpha
\end{align}

这是开集的并,故为开集。

\textbf{证明 $f^{-1}((c,1])$ 为开集:}

对于 $\inf > c$,即 $f(x) > c$,则对任意 $\alpha < c$ 均有 $x \notin U_\alpha$。

\begin{align}
\{x : f(x) > c\} &= \bigcup_{\substack{\alpha \in \Lambda \\ c < \alpha}} \overline{U_\alpha}^c
\end{align}

通过如下推理:若 $f(x) > c$,则 $\inf\{\alpha : x \in U_\alpha\} > c$,因此 $\exists \alpha \in \Lambda$ 且 $c < \alpha < f(x)$ 使得 $x \notin U_\alpha$。由于 $\overline{U_\alpha} \subseteq U_\beta$ 对任意 $\beta > \alpha$,可知 $x \notin \overline{U_\alpha}$,即 $x \in \overline{U_\alpha}^c$。

反之,若 $x \in \overline{U_\alpha}^c$ 对某 $\alpha > c$,则 $x \notin \overline{U_\alpha}$,由 $U_\gamma \subseteq \overline{U_\gamma} \subseteq U_\alpha$ 对 $\gamma < \alpha$,可知 $\forall \gamma < \alpha$ 有 $x \notin U_\gamma$,因此 $\inf\{\beta : x \in U_\beta\} \geq \alpha > c$,即 $f(x) > c$。

因此 $\{x : f(x) > c\}$ 为开集的并,故为开集。

综上,$f$ 为连续函数。
\end{proof}

\begin{theorem}[Urysohn 引理]\label{thm:urysohn-lemma}
设 $X$ 为正规空间,$A, B$ 为 $X$ 中不相交闭集。则存在连续函数 $f: X \to [0,1]$ 使得 $f(A) \subseteq \{0\}$ 且 $f(B) \subseteq \{1\}$。
\end{theorem}
\begin{proof}
设 $A, B$ 为不相交闭集,即 $A \cap B = \varnothing$。

由正规性,存在开集 $U_{\frac{1}{2}}$ 使得 
\begin{equation}
A \subseteq U_{\frac{1}{2}} \subseteq \overline{U_{\frac{1}{2}}} \subseteq B^c.
\end{equation}

令 $\Lambda = \left\{\frac{m}{2^n} : n, m \in \mathbb{N}, 0 \leq m \leq 2^n\right\}$ 为 $[0,1]$ 中的二进有理数。

通过归纳法,对所有 $\alpha \in \Lambda \cap [0,1)$,构造开集 $U_\alpha$ 使得:对任意 $\alpha, \beta \in \Lambda$ 且 $\alpha < \beta$,有
\begin{equation}
A \subseteq U_\alpha \subseteq \overline{U_\alpha} \subseteq U_\beta.
\end{equation}

具体构造如下:假设对所有分母为 $2^n$ 的 $\alpha \in \Lambda$ 已构造 $U_\alpha$。对分母为 $2^{n+1}$ 的 $\alpha = \frac{2k+1}{2^{n+1}}$,由正规性,存在开集 $U_\alpha$ 使得
\begin{equation}
\overline{U_{\frac{k}{2^n}}} \subseteq U_\alpha \subseteq \overline{U_\alpha} \subseteq U_{\frac{k+1}{2^n}}.
\end{equation}

令 $U_1 = X$。则 $(U_\alpha)_{\alpha \in \Lambda}$ 满足引理 \ref{lem:urysohn-construction} 的条件。

定义 $f: X \to [0,1]$ 为
\begin{equation}
f(x) = \inf\{\alpha \in \Lambda : x \in U_\alpha\}.
\end{equation}

由引理 \ref{lem:urysohn-construction},$f$ 为连续函数。

对 $x \in B$,因为 $\overline{U_{\frac{1}{2}}} \subseteq B^c$,所以 $x \notin \overline{U_{\frac{1}{2}}}$,因此对所有 $\alpha \in \Lambda \cap [0,1)$,有 $x \notin U_\alpha$(由嵌套性质)。故 $f(x) = 1$。

对 $x \in A$,因为 $A \subseteq U_\alpha$ 对所有 $\alpha \in \Lambda \cap (0,1]$,所以 $f(x) = 0$。

因此 $f(A) \subseteq \{0\}$ 且 $f(B) \subseteq \{1\}$。
\end{proof}

\begin{corollary}\label{cor:T4-implies-T3.5}
$T_4 \Rightarrow T_{3\frac{1}{2}}$,即:若 $X$ 为 $T_4$ 空间,则 $X$ 为 $T_{3\frac{1}{2}}$ 空间。
\end{corollary}
\begin{proof}
由 $T_4$ 空间的定义,$X$ 为正规空间且为 $T_1$ 空间。

设 $x \in X$,$B$ 为闭集且 $x \notin B$。

由 $T_1$ 性质,单点集 $\{x\}$ 为闭集。

因为 $\{x\} \cap B = \varnothing$,由 Urysohn 引理(定理 \ref{thm:urysohn-lemma}),存在连续函数 $f: X \to [0,1]$ 使得 $f(x) = 0$ 且 $f(B) \subseteq \{1\}$。

因此 $X$ 为完全正则空间。

结合 $X$ 为 $T_1$ 空间,故 $X$ 为 $T_{3\frac{1}{2}}$ 空间。
\end{proof}

\begin{corollary}\label{cor:regular-normal-implies-completely-regular}
正则 $+$ 正规 $\Rightarrow$ 完全正则,即:若 $X$ 既是正则空间又是正规空间,则 $X$ 为完全正则空间。
\end{corollary}
\begin{proof}
设 $x \in X$,$B$ 为闭集且 $x \notin B$。

由正则性,存在开集 $U, V$ 使得 $x \in U$,$B \subseteq V$,且 $U \cap V = \varnothing$。

因此 $x \in U \subseteq V^c$,即 $x \in U \subseteq \overline{U} \subseteq V^c = B^c$(这里 $\overline{U} \subseteq V^c$ 因为 $U \cap V = \varnothing$)。

由于 $\{x\}$ 和 $B$ 为不相交闭集($\{x\} \subseteq \overline{U}$ 且 $\overline{U} \subseteq B^c$),由 Urysohn 引理(定理 \ref{thm:urysohn-lemma})和正规性,存在连续函数 $f: X \to [0,1]$ 使得 $f(\{x\}) = \{0\}$ 且 $f(B) \subseteq \{1\}$。

因此 $X$ 为完全正则空间。
\end{proof}

\chapter{可分性与可数性公理}

\section{可分空间与稠密性}

\begin{definition}[稠密]\label{def:dense}
设 $X$ 为拓扑空间,$D \subseteq X$。称 $D$ 在 $X$ 中稠密,若 $\overline{D} = X$。
\end{definition}

\begin{definition}[可分空间]\label{def:separable-space}
设 $X$ 为拓扑空间,称 $X$ 为可分空间,若 $X$ 中存在可数稠密子集,即存在可数集 $D \subseteq X$ 使得 $\overline{D} = X$。
\end{definition}

\begin{example}
以下是可分空间的例子:
\begin{enumerate}
\item $\mathbb{R}^n$ 是可分的,令 $D = \{(x_1, \ldots, x_n) : x_1, \ldots, x_n \in \mathbb{Q}\}$,则 $D$ 可数且 $\overline{D} = \mathbb{R}^n$。
\item 可分空间的可分子空间也是可分的。
\end{enumerate}
\end{example}

\section{Lindelöff 空间与正规性}

\begin{proposition}\label{prop:regular-lindeloff-implies-normal}
正则 $+$ Lindelöff $\Rightarrow$ 正规,即:若 $X$ 既是正则空间又是 Lindelöff 空间,则 $X$ 为正规空间。
\end{proposition}
\begin{proof}
设 $A, B$ 为不相交闭集。

对每个 $a \in A$,由正则性,存在开邻域 $U_a$ 使得 $a \in U_a$ 且 $\overline{U_a} \cap B = \varnothing$。

对每个 $b \in B$,由正则性,存在开邻域 $V_b$ 使得 $b \in V_b$ 且 $\overline{V_b} \cap A = \varnothing$。

由于 $A, B$ 为闭集且 $X$ 为 Lindelöff 空间,由命题 \ref{prop:lindeloff-characterization},存在可数序列 $(U_n)_{n=1}^{\infty}$ 为 $A$ 的开覆盖,满足对所有 $n$,有 $\overline{U_n} \cap B = \varnothing$。

类似地,存在可数序列 $(V_n)_{n=1}^{\infty}$ 为 $B$ 的开覆盖,满足对所有 $n$,有 $\overline{V_n} \cap A = \varnothing$。

定义新的开集序列:
\begin{align}
U_n^* &= U_n \setminus \left(\overline{V_1} \cup \overline{V_2} \cup \cdots \cup \overline{V_n}\right), \quad n = 1, 2, 3, \ldots \\
V_n^* &= V_n \setminus \left(\overline{U_1} \cup \overline{U_2} \cup \cdots \cup \overline{U_n}\right), \quad n = 1, 2, 3, \ldots
\end{align}

则 $U_n^*$ 和 $V_n^*$ 均为开集(开集减去有限个闭集的并)。

令
\begin{equation}
U^* = \bigcup_{n=1}^{\infty} U_n^*, \quad V^* = \bigcup_{n=1}^{\infty} V_n^*
\end{equation}

则 $U^*, V^*$ 为开集。

\textbf{证明 $A \subseteq U^*$:}

对任意 $a \in A$,存在 $n$ 使得 $a \in U_n$。由于 $\overline{V_k} \cap A = \varnothing$ 对所有 $k$,所以 $a \notin \overline{V_k}$ 对所有 $k \leq n$。因此 $a \in U_n^* \subseteq U^*$。

\textbf{证明 $B \subseteq V^*$:}

类似地,对任意 $b \in B$,存在 $n$ 使得 $b \in V_n$,且 $b \notin \overline{U_k}$ 对所有 $k \leq n$,因此 $b \in V_n^* \subseteq V^*$。

\textbf{证明 $U^* \cap V^* = \varnothing$:}

对任意 $n, m \in \mathbb{N}$,不失一般性设 $n \leq m$。则
\begin{equation}
U_n^* \cap V_m^* = \left(U_n \setminus \bigcup_{k=1}^{n} \overline{V_k}\right) \cap \left(V_m \setminus \bigcup_{k=1}^{m} \overline{U_k}\right) \subseteq U_n \cap (\overline{U_n})^c = \varnothing
\end{equation}

因为 $n \leq m$,所以 $\overline{U_n} \subseteq \bigcup_{k=1}^{m} \overline{U_k}$,从而 $V_m^* \subseteq (\overline{U_n})^c$。

因此 $U^* \cap V^* = \varnothing$。

故 $X$ 为正规空间。
\end{proof}

\begin{proposition}\label{prop:lindeloff-characterization}
若 $X$ 为 Lindelöff 空间,$A$ 为 $X$ 中闭集,则对 $A$ 的任意开覆盖 $\mathcal{U}$(即由 $X$ 中开集组成且覆盖 $A$),存在 $\mathcal{U}$ 的可数子覆盖覆盖 $A$。
\end{proposition}
\begin{proof}
设 $X$ 为 Lindelöff 空间,$A$ 为闭集,$\mathcal{U}$ 为 $A$ 的开覆盖,即 $\bigcup \mathcal{U} \supseteq A$。

考虑 $X$ 的开覆盖 $\mathcal{U} \cup \{A^c\}$。

由 Lindelöff 性质,存在可数子覆盖覆盖 $X$。

若 $A^c$ 在该子覆盖中,则去掉 $A^c$ 后,剩余的可数个 $\mathcal{U}$ 中的开集仍覆盖 $A$。

若 $A^c$ 不在该子覆盖中,则该子覆盖本身就是 $\mathcal{U}$ 的可数子覆盖,覆盖 $X$,因此也覆盖 $A$。

综上,存在 $\mathcal{U}$ 的可数子覆盖覆盖 $A$。
\end{proof}

\section{第二可数空间}

\begin{definition}[第二可数空间]\label{def:second-countable}
设 $X$ 为拓扑空间,称 $X$ 为第二可数空间(即 $\text{CA}_2$ 空间),若 $X$ 存在可数拓扑基。
\end{definition}

\begin{definition}[加细]\label{def:refinement}
设 $\mathcal{A}, \mathcal{B} \subseteq \mathcal{P}(X)$。称 $\mathcal{B}$ 为 $\mathcal{A}$ 的加细,若对任意 $A \in \mathcal{A}$,存在 $B \in \mathcal{B}$ 使得 $A \subseteq B$。
\end{definition}

\begin{proposition}\label{prop:base-property}
若 $\mathcal{B}$ 为 $X$ 的拓扑基,$\mathcal{U}$ 为 $X$ 的开覆盖,即 $\bigcup \mathcal{U} = X$,则存在 $\overline{\mathcal{B}} \subseteq \mathcal{B}$ 使得 $\bigcup \overline{\mathcal{B}} = X$,且 $\overline{\mathcal{B}}$ 加细 $\mathcal{U}$。
\end{proposition}

\begin{lemma}\label{lem:refinement-cover}
设 $\mathcal{A}, \mathcal{B}$ 为 $X$ 的覆盖(不要求是拓扑基),且 $\mathcal{A}$ 加细 $\mathcal{B}$。若 $\mathcal{A}$ 可数,则存在可数的 $\mathcal{B}' \subseteq \mathcal{B}$ 使得 $\mathcal{B}'$ 覆盖 $X$。
\end{lemma}
\begin{proof}
设 $\mathcal{A}, \mathcal{B}$ 均为 $X$ 的覆盖,$\mathcal{A}$ 加细 $\mathcal{B}$,且 $\mathcal{A}$ 可数。

由加细的定义,对每个 $A \in \mathcal{A}$,存在 $B_A \in \mathcal{B}$ 使得 $A \subseteq B_A$。

令 $\mathcal{B}' = \{B_A : A \in \mathcal{A}\}$。

则 $\mathcal{B}' \subseteq \mathcal{B}$。

因为 $\mathcal{A}$ 可数,所以 $\mathcal{B}' = \{B_A : A \in \mathcal{A}\}$ 也是可数的(可数集的像)。

且
\begin{equation}
\bigcup \mathcal{B}' = \bigcup_{A \in \mathcal{A}} B_A \supseteq \bigcup_{A \in \mathcal{A}} A = X
\end{equation}

因此 $\mathcal{B}'$ 是 $\mathcal{B}$ 的可数子族且覆盖 $X$。
\end{proof}

\section{可度量空间}

\begin{definition}[可度量空间]
    设 $(X, \mathcal{T})$ 为拓扑空间,称 $X$ 为可度量空间。若满足:$\exists \quad d: X \times X \to [0, +\infty)$(伪)度量,使得:
    \begin{equation}
        \mathcal{T}_d=\mathcal{T}
    \end{equation}
\end{definition}

\begin{property}
若$X$为一个可度量空间,则以下等价
\begin{enumerate}
    \item $X$ 为 $CA_2$的。
    \item $X$ 为可分空间。
    \item $X$ 为 Lindelöff 空间。
\end{enumerate}
\end{property}

\begin{proof}
1$\Rightarrow$2:设 $\mathcal{B}=\{B_n\}_{n\in\mathbb{N}}$ 为 $X$ 的可数拓扑基(定义 \ref{def:second-countable})。对每个非空 $B_n$ 取一点 $x_n\in B_n$,令 $D=\{x_n:\,B_n\neq\varnothing\}$,则 $D$ 可数。任取非空开集 $U$,存在基元 $B_n\subset U$,故 $x_n\in D\cap U\neq\varnothing$,从而 $\overline{D}=X$(定义 \ref{def:separable-space}),即 $X$ 可分。

1$\Rightarrow$3:令 $\mathcal{U}$ 为 $X$ 的任一开覆盖。由命题 \ref{prop:base-property},存在 $\overline{\mathcal{B}}\subset\mathcal{B}$ 使得 $\bigcup\overline{\mathcal{B}}=X$ 且 $\overline{\mathcal{B}}$ 加细 $\mathcal{U}$。因 $\mathcal{B}$ 可数,$\overline{\mathcal{B}}$ 亦可数。由引理 \ref{lem:refinement-cover},存在 $\mathcal{U}$ 的可数子族覆盖 $X$,故 $X$ 为 Lindelöff 空间。
\end{proof}

\begin{proposition}[基的等价刻画]\label{prop:base-equivalence}
设 $(X,\mathcal T)$ 为拓扑空间,$\mathcal B\subset \mathcal T$。则以下等价:
\begin{enumerate}
  \item $\mathcal B$ 是 $(X,\mathcal T)$ 的拓扑基;
  \item 对任意 $U\in\mathcal T$ 与任意 $x\in U$,存在 $B\in\mathcal B$ 使得 $x\in B\subset U$。
\end{enumerate}
\end{proposition}
\begin{proof}
(1)$\Rightarrow$(2):设 $U\in\mathcal T$。由基的定义,存在 $\mathcal C\subset \mathcal B$ 使得 $U=\bigcup_{C\in\mathcal C}C$。取 $x\in U$,则存在 $C\in\mathcal C$ 使 $x\in C$,且 $C\subset U$,令 $B=C$ 即得。

(2)$\Rightarrow$(1):设 $U\in\mathcal T$,令 $\mathcal C=\{B\in\mathcal B:\,B\subset U\}$。一方面 $\bigcup\mathcal C\subset U$;另一方面,任意 $x\in U$,由(2)存在 $B\in\mathcal B$ 使 $x\in B\subset U$,故 $x\in\bigcup\mathcal C$。于是 $U\subset \bigcup\mathcal C$,从而 $U=\bigcup\mathcal C$。
\end{proof}

\begin{proof}
(补充:2$\Rightarrow$1 与 3$\Rightarrow$1,度量空间情形)

2$\Rightarrow$1:设 $(X,d)$ 可分,取可数稠密集 $D$。对 $x\in D$ 与 $n\in\mathbb N^*$,记 $B_{x,n}=B(x,1/n)$,并令
\[
\mathcal B=\{B_{x,n}:\ x\in D,\ n\in\mathbb N^*\}.
\]
则 $\mathcal B$ 为可数族。任取开集 $U$ 与 $y\in U$,取 $\varepsilon>0$ 使 $B(y,\varepsilon)\subset U$。取 $n$ 使 $2/n<\varepsilon$,由 $D$ 稠密可取 $x\in D\cap B(y,1/n)$。则 $y\in B(x,1/n)$ 且对任意 $z\in B(x,1/n)$ 有 $d(z,y)\le d(z,x)+d(x,y)<\tfrac1n+\tfrac1n=\tfrac{2}{n}<\varepsilon$,故 $B(x,1/n)\subset B(y,\varepsilon)\subset U$。由命题 \ref{prop:base-equivalence} 知 $\mathcal B$ 为可数基,故 $X$ 为 $\mathrm{CA}_2$。

3$\Rightarrow$1:设 $(X,d)$ 为 Lindelöff。对每个 $n\in\mathbb N^*$,考虑半径 $1/n$ 的开球族
\[
  \mathcal U_n=\{B(x,1/n):\ x\in X\},
  \]
  它覆盖 $X$。由 Lindelöff 性质,存在可数子族 $\mathcal B_n\subset \mathcal U_n$ 覆盖 $X$。令 $\mathcal B=\bigcup_{n\ge1}\mathcal B_n$,则 $\mathcal B$ 可数。任取开集 $U$ 与 $y\in U$,取 $\varepsilon>0$ 使 $B(y,\varepsilon)\subset U$,并取 $n$ 使 $2/n<\varepsilon$。因 $\mathcal B_n$ 覆盖 $X$,存在 $B(x,1/n)\in\mathcal B_n$ 使 $y\in B(x,1/n)$。同理得 $B(x,1/n)\subset B(y,\varepsilon)\subset U$。  故由命题 \ref{prop:base-equivalence},$\mathcal B$ 为可数基,$X$ 为 $\mathrm{CA}_2$。
\end{proof}
  \begin{proposition}[紧的基刻画]\label{prop:compact-base-characterization}
  设 $(X,\mathcal T)$ 为拓扑空间,$\mathcal B$ 为 $X$ 的一组拓扑基。则以下等价:
  \begin{enumerate}
    \item $X$ 为紧空间;
    \item 对任意子族 $\mathcal U\subset \mathcal B$,若 $\bigcup \mathcal U = X$,则存在有限子族 $\mathcal U_0\subset \mathcal U$ 使得 $\bigcup \mathcal U_0 = X$。
  \end{enumerate}
  \end{proposition}

  \begin{proof}
  (1)$\Rightarrow$(2):显然。由基元素构成的覆盖亦为开覆盖,紧性蕴含存在有限子覆盖。

  (2)$\Rightarrow$(1):任取开覆盖 $\mathcal V$。由命题 \ref{prop:base-property},存在 $\overline{\mathcal B} \subset \mathcal B$ 使得 $\bigcup \overline{\mathcal B} = X$ 且 $\overline{\mathcal B}$ 加细 $\mathcal V$。由 (2) 可取有限子族 $\overline{\mathcal B}_0 \subset \overline{\mathcal B}$ 覆盖 $X$。对每个 $B\in \overline{\mathcal B}_0$,取相应的 $V_B\in \mathcal V$ 使得 $B\subset V_B$,则 $\{V_B: B\in \overline{\mathcal B}_0\}$ 为 $\mathcal V$ 的有限子覆盖,故 $X$ 紧。
  \end{proof}
 
 \begin{proposition}[紧的等价刻画:闭集的有限交性质]\label{prop:compact-FIP}
 设 $(X,\mathcal T)$ 为拓扑空间。以下命题等价:
 \begin{enumerate}
   \item $X$ 为紧空间;
   \item 对任意闭集族 $\mathcal F\subseteq\mathcal P(X)$,若 $\mathcal F$ 具有有限交性质(即任意有限子族的交非空),则
   \[
   \bigcap_{F\in\mathcal F} F \neq \varnothing.
   \]
 \end{enumerate}
 \end{proposition}
 \begin{proof}
 (1)$\Rightarrow$(2):设 $\mathcal F$ 为闭集族且具有有限交性质。若反设 $\bigcap_{F\in\mathcal F}F=\varnothing$,则其补开的族 $\mathcal U:=\{F^c:\,F\in\mathcal F\}$ 为 $X$ 的开覆盖。由紧性,存在有限子覆盖 $\{F_i^c\}_{i=1}^n$,则
 \[
 X=\bigcup_{i=1}^n F_i^c \quad\Longleftrightarrow\quad \bigcap_{i=1}^n F_i=\varnothing,
 \]
 这与 $\mathcal F$ 的有限交性质矛盾,故应有 $\bigcap_{F\in\mathcal F}F\neq\varnothing$。
 
 (2)$\Rightarrow$(1):设 $\mathcal U$ 为 $X$ 的开覆盖。令闭集族 $\mathcal F=\{X\setminus U:\,U\in\mathcal U\}$,则 $\mathcal F$ 具有有限交性质且
 若反设 $\mathcal U$ 无有限子覆盖,则对任意有限子集 $\{U_1,\dots,U_n\}\subset\mathcal U$,有
 \[
 U_1\cup\cdots\cup U_n\ne X\quad\Longleftrightarrow\quad (X\setminus U_1)\cap\cdots\cap(X\setminus U_n)\ne\varnothing,
 \]
 因而 $\mathcal F$ 具有有限交性质。由(2)得
 \[
 \bigcap_{F\in\mathcal F}F\ne\varnothing\quad\Longleftrightarrow\quad X\setminus\bigcup\mathcal U\ne\varnothing,
 \]
 与 $\mathcal U$ 为覆盖矛盾。故 $\mathcal U$ 必有有限子覆盖,$X$ 紧。
 \end{proof}
 
  \begin{definition}[滤子与滤子基的接触点]\label{def:filter-adherent}
 设 $(X,\mathcal T)$ 为拓扑空间。
\begin{enumerate}
    \item 若 $\mathcal F$ 为 $X$ 上的滤子,称 $x\in X$ 为 $\mathcal F$ 的接触点,若 $\forall A\in\mathcal F$ 均有 $x\in \overline{A}$。等价地 $x\in \bigcap_{A\in\mathcal F}\overline{A}$。
    \item 若 $\mathcal B$ 为 $X$ 上的滤子基,称 $x\in X$ 为 $\mathcal B$ 的接触点,若 $\forall B\in\mathcal B$ 均有 $x\in \overline{B}$。等价地 $x\in \bigcap_{B\in\mathcal B}\overline{B}$。
  \end{enumerate}
  \end{definition}
  
  \begin{proposition}[紧的等价刻画:滤子/滤子基的接触点]\label{prop:compact-filter-cluster}
  设 $(X,\mathcal T)$ 为拓扑空间,则以下等价:
  \begin{enumerate}
    \item $X$ 为紧空间;
    \item $X$ 中任意滤子基都有接触点;
    \item $X$ 中任意滤子都有接触点。
  \end{enumerate}
  \end{proposition}
  \begin{proof}
  (2)$\Leftrightarrow$(3):若 $\mathcal B$ 为滤子基,记其生成的滤子为 $\mathcal B_{\uparrow}$。则对任意 $x$,
  \[
  x\in \bigcap_{B\in\mathcal B}\overline{B}
  \ \Longleftrightarrow\ 
  \forall F\in \mathcal B_{\uparrow}\ \exists B\in\mathcal B\ (B\subset F\ \&\ x\in\overline{B}\subset \overline{F})
  \ \Longleftrightarrow\ 
  x\in \bigcap_{F\in\mathcal B_{\uparrow}}\overline{F}.
  \]
  故两者的接触点集合一致。
  
  (1)$\Rightarrow$(2):设 $X$ 紧且 $\mathcal B$ 为滤子基。闭集族 $\{\overline{B}:B\in\mathcal B\}$ 具有限交性质;由命题 \ref{prop:compact-FIP} 得 $\bigcap_{B\in\mathcal B}\overline{B}\ne\varnothing$,取其中一点即为接触点。
  
  (2)$\Rightarrow$(1):反证。若 $X$ 不紧,取无有限子覆盖的开覆盖 $\mathcal U$,令闭集族 $\mathcal F=\{X\setminus U:\ U\in\mathcal U\}$,则 $\mathcal F$ 具有有限交性质且
  \[
  \bigcap_{F\in\mathcal F}F=\varnothing.
  \]
  定义由 $\mathcal F$ 的有限交组成的族
  \[
  \mathcal B:=\Big\{\bigcap_{i=1}^n F_i:\ F_i\in\mathcal F,\ n\in\mathbb N^*\Big\},
  \]
  则 $\mathcal B$ 为滤子基,且每个 $B\in\mathcal B$ 为闭集。由(2)存在接触点 $x\in \bigcap_{B\in\mathcal B}\overline{B}=\bigcap_{B\in\mathcal B}B=\bigcap_{F\in\mathcal F}F=\varnothing$,矛盾。故 $X$ 紧。
  \end{proof}
  

  \begin{definition}[超滤子]\label{def:ultrafilter}
  设 $X$ 为非空集合,$\mathcal U\subseteq\mathcal P(X)$ 为 $X$ 上的一个滤子。称 $\mathcal U$ 为 $X$ 上的一个超滤子(极大滤子),若满足:对任意 $\mathcal A\subseteq\mathcal P(X)$,只要 $\mathcal U\subsetneqq\mathcal A$,则 $\mathcal A$ 不是 $X$ 上的滤子。
  \end{definition}
  
  \begin{proposition}[超滤子引理]\label{prop:ultrafilter-extension}
  任意 $X$ 上的滤子 $\mathcal F$ 均可扩张为一个超滤子 $\mathcal U\supseteq\mathcal F$。
  \end{proposition}
  \begin{proof}
  令
  \[
  \mathcal C=\{\mathcal G:\ \mathcal G \text{ 是 $X$ 上的滤子且 }\mathcal F\subseteq\mathcal G\},
  \]
  以包含关系为序。任一链 $\{\mathcal G_i\}$ 的并 $\mathcal G_*=\bigcup_i\mathcal G_i$ 仍为滤子:$\varnothing\notin\mathcal G_*$;对有限交与向上封闭性逐一验证可得,故每条链有上界。由 Zorn 引理,存在极大元 $\mathcal U$,即所求超滤子。
  \end{proof}

  \begin{proposition}[超滤子的等价性质]\label{prop:ultrafilter-properties}
  设 $\mathcal U$ 为 $X$ 上的滤子。则以下命题等价:
  \begin{enumerate}
    \item $\mathcal U$ 是 $X$ 上的超滤子(极大滤子);
    \item 对任意 $A\subseteq X$,有 $A\in\mathcal U$ 或 $A^c\in\mathcal U$;
    \item 对任意 $A,B\subseteq X$,若 $A\cup B\in\mathcal U$,则 $A\in\mathcal U$ 或 $B\in\mathcal U$;
    \item 对任意 $U\subseteq X$,若对所有 $F\in\mathcal U$ 有 $U\cap F\neq\varnothing$,则 $U\in\mathcal U$。
  \end{enumerate}
  \end{proposition}

  \begin{proof}
  (1)$\Rightarrow$(2):对任意 $A\subseteq X$,若 $A\notin\mathcal U$,欲证 $A^c\in\mathcal U$。由(1)超滤子定义,$\mathcal U\cup\{A\}$ 不能生成滤子(否则 $\mathcal U$ 不极大),故不具有限交性质,即存在 $F\in\mathcal U$ 使
  \[
  A\cap F=\varnothing\quad\Longrightarrow\quad F\subseteq A^c.
  \]
  由滤子向上封闭性得 $A^c\in\mathcal U$。故对任意 $A$,必有 $A\in\mathcal U$ 或 $A^c\in\mathcal U$。
  
  (2)$\Rightarrow$(3):设 $A,B\subseteq X$ 且 $A\cup B\in\mathcal U$。由(2)知 $A\in\mathcal U$ 或 $B\in\mathcal U$。反设 $A\notin\mathcal U$ 且 $B\notin\mathcal U$,则由(2)得 $A^c,B^c\in\mathcal U$,从而
  \[
  A^c\cap B^c=(A\cup B)^c\in\mathcal U.
  \]
  与 $A\cup B\in\mathcal U$ 及滤子不含 $\varnothing$ 矛盾。
  
  (3)$\Rightarrow$(4):设 $U\subseteq X$ 满足对任意 $F\in\mathcal U$ 均有 $U\cap F\ne\varnothing$。因 $\mathcal U$ 为滤子,$X\in\mathcal U$,而 $X=U\cup U^c$,由(3)得 $U\in\mathcal U$ 或 $U^c\in\mathcal U$。若 $U^c\in\mathcal U$,则 $U\cap U^c=\varnothing$,与假设矛盾,故 $U\in\mathcal U$。
  
  (4)$\Rightarrow$(1):反证。若 $\mathcal U$ 不是超滤子,则存在滤子 $\mathcal U'$ 满足 $\mathcal U\subsetneqq\mathcal U'$。取 $U\in\mathcal U'\setminus\mathcal U$。对任意 $F\in\mathcal U\subset\mathcal U'$,因 $\mathcal U'$ 为滤子,有 $U\cap F\ne\varnothing$(否则 $\varnothing\in\mathcal U'$ 矛盾)。由(4)得 $U\in\mathcal U$,矛盾。故 $\mathcal U$ 为超滤子。
  \end{proof}

  \begin{definition}[滤子的收敛]\label{def:filter-convergence-2}
  设 $(X,\mathcal T)$ 为拓扑空间。称滤子 $\mathcal F$ 在 $x\in X$ 收敛,记为 $\mathcal F\to x$,若对任意 $x$ 的邻域 $U$,有 $U\in\mathcal F$。
  \end{definition}
  
  \begin{proposition}[超滤子的接触点与收敛]\label{prop:ultrafilter-adherent-limit}
  设 $(X,\mathcal T)$ 为拓扑空间,$\mathcal U$ 为 $X$ 上的超滤子,$x\in X$。则以下等价:
  \begin{enumerate}
    \item $x$ 为 $\mathcal U$ 的接触点;
    \item $\mathcal U\to x$。
  \end{enumerate}
  \end{proposition}
  \begin{proof}
  (1)$\Rightarrow$(2):设 $x$ 为 $\mathcal U$ 的接触点。取 $x$ 的任意邻域 $U$。由接触点定义,$\forall F\in\mathcal U$ 有 $U\cap F\ne\varnothing$。若 $U\notin\mathcal U$,因 $\mathcal U$ 为超滤子,必有 $U^c\in\mathcal U$,与 $U\cap U^c=\varnothing$ 矛盾,故 $U\in\mathcal U$。于是 $\mathcal U\to x$。
  
  (2)$\Rightarrow$(1):若 $\mathcal U\to x$,则 $\forall$ 邻域 $U\in\mathcal U$。任取 $F\in\mathcal U$,由滤子对有限交封闭,$U\cap F\in\mathcal U$,故 $U\cap F\ne\varnothing$。于是 $x\in\overline{F}$。因 $F\in\mathcal U$ 任意,得 $x\in\bigcap_{F\in\mathcal U}\overline{F}$,即 $x$ 为 $\mathcal U$ 的接触点。
  \end{proof}
  \begin{proposition}[紧的等价刻画:极大滤子的接触点]\label{prop:compact-filter-cluster1}
  设 $(X,\mathcal T)$ 为拓扑空间,则以下等价:
  \begin{enumerate}
    \item $X$ 为紧空间;
    \item $X$ 中任意滤子基都有接触点;
    \item $X$ 中任意滤子都有接触点。
    \item $X$ 中任意极大滤子都有接触点。
    \item $X$ 中任意极大滤子收敛
  \end{enumerate}
  \end{proposition}
  \begin{proof}
  (3)$\Rightarrow$(4):极大滤子本身是滤子,显然成立。
  
  (4)$\Leftrightarrow$(5):由命题 \ref{prop:ultrafilter-adherent-limit},对任意极大滤子 $\mathcal U$ 与点 $x$,$x$ 为 $\mathcal U$ 的接触点当且仅当 $\mathcal U\to x$。于是“任意极大滤子有接触点”当且仅当“任意极大滤子收敛”。
  
  (4)$\Rightarrow$(3):任取滤子 $\mathcal F$。由超滤子引理 \ref{prop:ultrafilter-extension},存在极大滤子 $\mathcal U\supseteq\mathcal F$。由(4) 取 $x$ 使 $x$ 为 $\mathcal U$ 的接触点。因 $\mathcal F\subseteq\mathcal U$,对任意 $A\in\mathcal F$ 亦有 $x\in\overline{A}$,故 $x$ 为 $\mathcal F$ 的接触点。
  
  \end{proof}
  
  
  \begin{proposition}[滤子收敛的邻域基刻画]\label{prop:filter-conv-nbhd-base}
  设 $(X,\mathcal T)$ 为拓扑空间,$x\in X$,令 $\mathcal B_x$ 为 $x$ 的邻域基。对任意滤子 $\mathcal F$,有
  \[
  \mathcal F\to x \Longleftrightarrow \forall B\in\mathcal B_x,\ \exists F\in\mathcal F\text{ 使 }F\subseteq B.
  \]
  \end{proposition}
  \begin{proof}
  ($\Rightarrow$) 设 $\mathcal F\to x$。任取 $B\in\mathcal B_x$,则 $B$ 为 $x$ 的邻域,故 $B\in\mathcal F$,取 $F=B$ 即得 $F\subseteq B$。
  
  ($\Leftarrow$) 设对任意 $B\in\mathcal B_x$,存在 $F\in\mathcal F$ 使得 $F\subseteq B$。任取 $x$ 的邻域 $U$,由邻域基性质,取 $B\in\mathcal B_x$ 使 $B\subseteq U$。于是存在 $F\in\mathcal F$ 满足 $F\subseteq B\subseteq U$。由滤子向上封闭性,$U\in\mathcal F$。故 $\mathcal F\to x$。
  \end{proof}
  
  \begin{proposition}[滤子收敛的邻域子基刻画]\label{prop:filter-conv-subnbhd}
  设 $(X,\mathcal T)$ 为拓扑空间,$x\in X$。设 $\mathcal S$ 为 $x$ 的邻域子基,即令
  \[
  \mathcal S_x=\{U\in\mathcal S:\ x\in U\},\qquad
  \mathcal B_x=\Big\{\bigcap_{i=1}^n U_i:\ n\in\mathbb N^*,\ U_i\in\mathcal S_x\Big\}\cup\{X\},
  \]
  则 $\mathcal B_x$ 为 $x$ 的邻域基。对任意滤子 $\mathcal F$,有
  \[
  \mathcal F\to x\ \Longleftrightarrow\ \forall U\in\mathcal S\ (x\in U\Rightarrow \exists F\in\mathcal F\text{ 使 }F\subseteq U).
  \]
  \end{proposition}
  \begin{proof}
  ($\Rightarrow$) 若 $\mathcal F\to x$,则每个 $x$ 的邻域 $U$ 都在 $\mathcal F$ 中,尤其对任意 $U\in\mathcal S_x$,有 $U\in\mathcal F$,取 $F=U$ 即得 $F\subseteq U$。
  
  ($\Leftarrow$) 设对任意 $U\in\mathcal S_x$,存在 $F\in\mathcal F$ 使 $F\subseteq U$。取 $\mathcal B_x$ 为由 $\mathcal S_x$ 的有限交与 $\{X\}$ 生成的邻域基。任取 $B\in\mathcal B_x$:
  \begin{itemize}
    \item 若 $B=X$,任取 $F\in\mathcal F$,则 $F\subseteq X=B$;
    \item 若 $B\ne X$,则 $B=U_1\cap\cdots\cap U_n$($n\ge1$,$U_i\in\mathcal S_x$)。对每个 $i$ 取 $F_i\in\mathcal F$ 使 $F_i\subseteq U_i$;
    \item 由滤子对有限交封闭,$F=\bigcap_{i=1}^n F_i\in\mathcal F$ 且 $F\subseteq\bigcap_{i=1}^n U_i=B$。
  \end{itemize}
  于是对任意 $B\in\mathcal B_x$,存在 $F\in\mathcal F$ 使 $F\subseteq B$。由命题 \ref{prop:filter-conv-nbhd-base} 知 $\mathcal F\to x$。
  \end{proof}
  
  \begin{proposition}[紧的子基刻画(Alexander 子基定理)]\label{prop:alexander-subbase}
  设 $\mathcal S$ 为 $X$ 的子基。则 $X$ 紧 当且仅当:对任意 $\mathcal U\subseteq\mathcal S$,若 $\mathcal U$ 覆盖 $X$,则存在有限子集 $\{U_i\}_{i=1}^n\subseteq\mathcal U$ 亦覆盖 $X$。
  \end{proposition}
  \begin{proof}
  ($\Rightarrow$) 显然:紧空间的任意开覆盖(特别是由子基元素组成的覆盖)都有有限子覆盖。
  
  ($\Leftarrow$) 假设对任意子基覆盖都有有限子覆盖。取任意超滤子 $\mathcal U$ 于 $X$。若 $\mathcal U$ 不收敛,则对每个 $x\in X$,由命题 \ref{prop:filter-conv-subnbhd},可取 $U_x\in\mathcal S$ 且 $x\in U_x$ 使 $U_x\notin\mathcal U$。由超滤子性质(命题 \ref{prop:ultrafilter-properties} 的(2))知 $U_x^c\in\mathcal U$。于是 $\{U_x:x\in X\}\subseteq\mathcal S$ 构成 $X$ 的子基覆盖。由假设取有限多项 $U_{x_1},\dots,U_{x_n}$ 覆盖 $X$,则
  \[
  \bigcap_{i=1}^n U_{x_i}^c=\varnothing.
  \]
  然而每个 $U_{x_i}^c\in\mathcal U$,且超滤子对有限交封闭,故 $\bigcap_{i=1}^n U_{x_i}^c\in\mathcal U$,与滤子不含空集矛盾。故任意超滤子必收敛。由命题 \ref{prop:compact-filter-cluster1} 中 (5) 与 (1) 的等价性,得 $X$ 紧。
  \end{proof}
  
 \begin{definition}[网]\label{def:net-variant}
  设 $\Lambda$ 为定向集,$x:\Lambda\to X$。称 $(x_\alpha)_{\alpha\in\Lambda}$ 为 $X$ 中的网。
  \end{definition}
  
  \begin{definition}[子网]\label{def:subnet}
  设 $(x_\alpha)_{\alpha\in\Lambda}$ 为 $X$ 中的网。若存在定向集 $\Gamma$ 与映射 $\theta: \Gamma\to\Lambda$,满足
  \begin{enumerate}
    \item $\theta$ 保序;
    \item $\theta(\Gamma)$ 在 $\Lambda$ 中无界,即对任意 $\alpha\in\Lambda$,存在 $\beta_0\in\Gamma$,使得 $\beta\ge\beta_0\Rightarrow\theta(\beta)\ge\alpha$,
  \end{enumerate}
  则称 $(x_{\theta(\beta)})_{\beta\in\Gamma}$ 为 $(x_\alpha)_{\alpha\in\Lambda}$ 的子网。等价地,可记为 $x\circ\theta: \Gamma\to X$。
  \end{definition}
  
 \chapter{乘积与和}
 \section{乘积拓扑与其子基}
 
 \begin{proposition}[乘积拓扑的刻画]\label{prop:product-topology-initial}
  设 $\{X_\alpha\}_{\alpha\in\Lambda}$ 为拓扑空间族。记
  \[
   \prod_{\alpha\in\Lambda} X_\alpha
   =\{(x_\alpha)_{\alpha\in\Lambda} : \forall\alpha,\ x_\alpha\in X_\alpha\}.
  \]
  对每个 $\alpha$ 定义投影
  \[
   p_\alpha: \prod_{\gamma\in\Lambda} X_\gamma\to X_\alpha,\qquad (x_\gamma)_{\gamma\in\Lambda}\mapsto x_\alpha.
  \]
  令
  \[
   \mathcal S\equiv\bigcup_{\alpha\in\Lambda}\{\,p_\alpha^{-1}(U): U\subseteq X_\alpha\text{ 为开集}\,\}.
  \]
  则由子基 $\mathcal S$ 生成的拓扑 $\tau$ 使得每个投影 $p_\alpha$ 连续;且若 $\tau'$ 是 $\prod_{\alpha}X_\alpha$ 上的拓扑并使所有 $p_\alpha$ 连续,则 $\tau\subseteq\tau'$。因此 $\tau$ 是使所有投影连续的最小拓扑(即乘积拓扑)。
 \end{proposition}
 
 \begin{proof}
  对任意 $\alpha$ 与 $X_\alpha$ 中开集 $U$,有 $p_\alpha^{-1}(U)\in\mathcal S\subseteq\tau$,故 $p_\alpha$ 连续。若 $\tau'$ 使所有 $p_\alpha$ 连续,则对每个开集 $U\subseteq X_\alpha$ 均有 $p_\alpha^{-1}(U)\in\tau'$,于是 $\mathcal S\subseteq\tau'$。由“由子基生成的最小拓扑”的定义知 $\tau\subseteq\tau'$。证毕。
 \end{proof}
 
 \begin{remark}
  对命题 \ref{prop:product-topology-initial} 中的子基 $\mathcal S$,其有限交族 $\mathcal S_{f\cap}$ 构成乘积拓扑的一个基。具体地,若取有限多指标 $\alpha_1,\dots,\alpha_n$ 及开集 $U_i\subseteq X_{\alpha_i}$,则
  \[
   p_{\alpha_1}^{-1}(U_1)\cap\cdots\cap p_{\alpha_n}^{-1}(U_n)
   = \prod_{\gamma\in\Lambda} V_\gamma,
  \]
  其中 $V_{\alpha_i}=U_i$,而当 $\gamma\notin\{\alpha_1,\dots,\alpha_n\}$ 时取 $V_\gamma=X_\gamma$。因此“盒子”形集合
  \[
   \Big\{\prod_{\gamma\in\Lambda} V_\gamma:\ V_{\alpha}\text{ 为开集仅在有限多 }\alpha,\ \text{其余 }V_\gamma=X_\gamma\Big\}
  \]
  是乘积拓扑的一个基。特别地,当 $\Lambda$ 有限(如 $\{1,\dots,n\}$)时,基可写为
  \[
   \{\,U_1\times\cdots\times U_n:\ U_i\subseteq X_i\text{ 为开集}\,\}.
  \]
  \end{remark}
 
 \begin{proposition}[推论:乘积拓扑的子基]\label{prop:product-subbasis}
  设 $\{X_\alpha\}_{\alpha\in\Lambda}$ 为拓扑空间族。则在 $\prod_{\alpha\in\Lambda} X_\alpha$ 上,集合族
  \[
   \mathcal S=\bigcup_{\alpha\in\Lambda}\{\,p_\alpha^{-1}(U): U\subseteq X_\alpha\ \text{开}\,\}
  \]
  是乘积拓扑的一个子基。
 \end{proposition}
 \begin{proof}
  由命题 \ref{prop:product-topology-initial},乘积拓扑恰为包含上述集合族的最小拓扑,故该集合族为其子基。
 \end{proof}
 
 \begin{proposition}[推论:有限乘积的基]\label{prop:finite-product-basis}
  设 $X_1,\dots,X_n$ 为拓扑空间,$B_i$ 为 $X_i$ 的一组基。则
  \[
   \mathcal{B}=\{\,U_1\times\cdots\times U_n:\ U_i\in B_i\ (i=1,\dots,n)\,\}
  \]
  为 $X_1\times\cdots\times X_n$ 的乘积拓扑的一组基。
 \end{proposition}
 \begin{proof}
  任取开盒 $V_1\times\cdots\times V_n$ 与点 $(x_1,\dots,x_n)$。由 $B_i$ 为基,存在 $U_i\in B_i$ 使 $x_i\in U_i\subseteq V_i$。则
  $(x_1,\dots,x_n)\in U_1\times\cdots\times U_n\subseteq V_1\times\cdots\times V_n$。配合上文的“盒子基”描述可知该族满足基的判别条件,因而为一组基。
  \end{proof}
  
 \begin{proposition}[连续满射下 Lindelöff 性保持]\label{prop:lindeloff-image}
  设 $f:X\to Y$ 连续且满射。若 $X$ 为 Lindelöff 空间,则 $Y$ 为 Lindelöff 空间。
 \end{proposition}
 \begin{proof}
  任取 $Y$ 的开覆盖 $\mathcal U$。则其拉回族
  \[
   f^{-1}(\mathcal U)=\{\,f^{-1}(U):\ U\in\mathcal U\,\}
  \]
  为 $X$ 的开覆盖(连续性保证开性,满射保证覆盖)。由 $X$ 的 Lindelöff 性,存在可数子族 $\{U_n\}_{n\in\mathbb N}\subset\mathcal U$ 使 $\{f^{-1}(U_n)\}$ 覆盖 $X$。于是
  \[
   Y=f(X)=\bigcup_{n\in\mathbb N} f\big(f^{-1}(U_n)\big)\subseteq\bigcup_{n\in\mathbb N} U_n,
  \]
  即 $\{U_n\}$ 为 $\mathcal U$ 的可数子覆盖,$Y$ 为 Lindelöff。
 \end{proof}
  
  \begin{theorem}[Tychonoff 定理]\label{thm:tychonoff}
  若每个 $X_\alpha$ 均为紧空间,则赋以乘积拓扑的笛卡尔积
  \[
   \prod_{\alpha\in\Lambda} X_\alpha
  \]
  亦为紧空间。
 \end{theorem}
 
 \begin{proof}
  取任意超滤子 $\mathcal F$ 于 $\prod_{\alpha}X_\alpha$。对每个 $\alpha$,考虑投影 $p_\alpha: \prod_{\gamma}X_\gamma\to X_\alpha$。则 $p_\alpha(\mathcal F)$ 为 $X_\alpha$ 上的超滤子;由 $X_\alpha$ 的紧性(见命题 \ref{prop:compact-filter-cluster1} 的等价刻画),存在点 $x_\alpha\in X_\alpha$ 使得
  \[
   p_\alpha(\mathcal F)\to x_\alpha.
  \]
  令 $x=(x_\gamma)_{\gamma\in\Lambda}\in\prod_{\gamma}X_\gamma$。
 
  下证 $\mathcal F\to x$。由上文关于乘积拓扑“盒子基”的描述,只需证:对任意指标 $\alpha$ 与含 $x_\alpha$ 的开集 $U\subseteq X_\alpha$,有 $p_\alpha^{-1}(U)\in\mathcal F$。事实上,由 $p_\alpha(\mathcal F)\to x_\alpha$ 与滤子收敛的定义,存在 $E\in p_\alpha(\mathcal F)$ 使 $E\subseteq U$;按像滤子的定义,$p_\alpha^{-1}(E)\in\mathcal F$ 且 $p_\alpha^{-1}(E)\subseteq p_\alpha^{-1}(U)$,由滤子的向上封闭性得 $p_\alpha^{-1}(U)\in\mathcal F$。进而对任意有限族 $\{(\alpha_i,U_i)\}_{i=1}^n$(每个 $U_i$ 为 $x_{\alpha_i}$ 的开邻域),由滤子对有限交封闭性,
  \[
   \bigcap_{i=1}^n p_{\alpha_i}^{-1}(U_i)\in\mathcal F.
  \]
  这些有限交构成 $x$ 的一组基邻域,故 $\mathcal F$ 包含 $x$ 的全部基邻域,从而 $\mathcal F\to x$。
 
  因任意超滤子于 $\prod_{\alpha}X_\alpha$ 上皆收敛,依命题 \ref{prop:compact-filter-cluster1} 的等价性知 $\prod_{\alpha}X_\alpha$ 紧。
 \end{proof}
 
 \begin{definition}[子空间拓扑]\label{def:subspace-topology}
  设 $(X,\mathcal T)$ 为拓扑空间,$Y\subseteq X$。定义
  \[
   \mathcal T|_Y\equiv\{\,U\cap Y:\ U\in\mathcal T\,\}
  \]
  为 $Y$ 上的子空间拓扑。称 $(Y,\mathcal T|_Y)$ 为 $(X,\mathcal T)$ 的子空间。
 \end{definition}
 
 \begin{proposition}[度量子空间]\label{prop:metric-subspace}
  设 $(X,d)$ 为度量空间,$Y\subseteq X$。记 $d_Y:Y\times Y\to\mathbb R$ 为限制度量 $d|_{Y\times Y}$,则由 $d$ 在 $X$ 上诱导的拓扑 $\mathcal T_d$ 限制到 $Y$ 上,与由限制度量 $d_Y$ 在 $Y$ 上诱导的拓扑 $\mathcal T_{d_Y}$ 相同,即
  \[
   \mathcal T_d|_Y=\mathcal T_{d_Y}.
  \]
 \end{proposition}

 \begin{proposition}[Lindelöff 子空间的刻画]\label{prop:lindeloff-subspace}
  设 $X$ 为拓扑空间,$\Omega\subseteq X$。则下列条件等价:
  \begin{enumerate}
    \item $\Omega$ 在子空间拓扑 $\mathcal T|_\Omega$ 下为 Lindelöff 空间;
    \item 对任意由 $X$ 中开集组成的族 $\mathcal U$ 且 $\bigcup \mathcal U\supseteq\Omega$,存在可数子族 $\{U_n\}_{n\in\mathbb N}\subseteq\mathcal U$ 使得 $\bigcup_{n\in\mathbb N} U_n\supseteq\Omega$。
  \end{enumerate}
 \end{proposition}
 \begin{proof}
  $(1)\Rightarrow(2)$:设 $\Omega$ 为 Lindelöff 子空间,取 $X$ 中开集族 $\mathcal U$ 覆盖 $\Omega$。令
  \[
   \mathcal V\equiv\{\,U\cap\Omega:\ U\in\mathcal U\,\}\subseteq\mathcal T|_\Omega.
  \]
  则 $\mathcal V$ 为 $\Omega$ 的开覆盖(子空间拓扑意义下)。由 $\Omega$ 的 Lindelöff 性,存在可数子族 $\{U_n\cap\Omega\}_{n\in\mathbb N}$ 覆盖 $\Omega$,从而 $\{U_n\}_{n\in\mathbb N}\subseteq\mathcal U$ 为所需可数子覆盖。
  
  $(2)\Rightarrow(1)$:设条件 (2) 成立,取 $\Omega$ 的任意开覆盖 $\mathcal V\subseteq\mathcal T|_\Omega$。对每个 $V\in\mathcal V$,由子空间拓扑定义,存在 $X$ 中开集 $U_V$ 使 $V=U_V\cap\Omega$。令 $\mathcal U\equiv\{U_V:\ V\in\mathcal V\}$,则 $\mathcal U$ 为 $X$ 中开集族且覆盖 $\Omega$。由条件 (2),存在可数子族 $\{U_{V_n}\}_{n\in\mathbb N}$ 覆盖 $\Omega$,从而 $\{V_n\}_{n\in\mathbb N}=\{U_{V_n}\cap\Omega\}$ 为 $\mathcal V$ 的可数子覆盖。故 $\Omega$ 为 Lindelöff 子空间。
 \end{proof}
 
 \begin{lemma}[管状邻域引理]\label{lem:tube-lemma}
  设 $X,Y$ 为拓扑空间,$\alpha\in X$,$K\subseteq Y$ 紧。若 $\Omega\subseteq X\times Y$ 为开集且 $\{\alpha\}\times K\subseteq\Omega$,则存在 $\alpha$ 的开邻域 $B\subseteq X$ 与 $K$ 的开邻域 $U\subseteq Y$ 使得
  \[
   \{\alpha\}\times K\subseteq B\times U\subseteq\Omega.
  \]
 \end{lemma}
 \begin{proof}
  对每个 $k\in K$,点 $(\alpha,k)\in\Omega$。由 $\Omega$ 的开性与乘积拓扑的基,存在 $X$ 中开集 $B_k\ni\alpha$ 与 $Y$ 中开集 $U_k\ni k$ 使 $B_k\times U_k\subseteq\Omega$。则 $\{U_k:k\in K\}$ 为 $K$ 的开覆盖;由 $K$ 的紧性,存在有限子覆盖 $\{U_{k_1},\dots,U_{k_n}\}$ 覆盖 $K$。令
  \[
   B\equiv\bigcap_{i=1}^n B_{k_i},\qquad U\equiv\bigcup_{i=1}^n U_{k_i}.
  \]
  则 $B$ 为 $\alpha$ 的开邻域(有限交),$U$ 为 $K$ 的开邻域(包含 $K$)。对任意 $(\beta,y)\in B\times U$,存在 $i$ 使 $y\in U_{k_i}$,且 $\beta\in B\subseteq B_{k_i}$,从而 $(\beta,y)\in B_{k_i}\times U_{k_i}\subseteq\Omega$。故 $\{\alpha\}\times K\subseteq B\times U\subseteq\Omega$。
  \end{proof}
  
 \begin{proposition}[基在子空间中的限制仍为基]\label{prop:subspace-basis-from-basis}
  设 $(X,\mathcal T)$ 为拓扑空间,$\mathcal B$ 为 $X$ 的一组基,$Y\subseteq X$。则
  \[
   \mathcal B|_Y\equiv\{\,B\cap Y:\ B\in\mathcal B\,\}
  \]
  为 $(Y,\mathcal T|_Y)$ 的一组基。
 \end{proposition}
 \begin{proof}
  任取 $\Omega\in\mathcal T|_Y$ 与 $y\in\Omega$。存在 $U\in\mathcal T$ 使 $\Omega=U\cap Y$ 且 $y\in U$。由 $\mathcal B$ 为 $X$ 的基,存在 $B\in\mathcal B$ 使 $y\in B\subseteq U$。则 $y\in B\cap Y\subseteq U\cap Y=\Omega$,且 $B\cap Y\in\mathcal B|_Y$。由基判别定理知 $\mathcal B|_Y$ 为 $(Y,\mathcal T|_Y)$ 的基。
 \end{proof}
 
 \begin{proposition}[子空间的乘积等于乘积的子空间]\label{prop:product-subspace}
  设 $(X,\mathcal T)$ 与 $(Y,\mathcal S)$ 为拓扑空间,$A\subseteq X$,$B\subseteq Y$。则在 $A\times B$ 上有
  \[
   (\mathcal T|_A)\otimes(\mathcal S|_B)\ =\ (\mathcal T\otimes\mathcal S)\big|_{A\times B}.
  \]
 \end{proposition}
 \begin{proof}
  记两边拓扑分别为 $\tau_1$ 与 $\tau_2$。考察恒等映射 $\operatorname{id}_{A\times B}: (A\times B,\tau_1)\to(A\times B,\tau_2)$。
  
  为证 $\operatorname{id}$ 连续,只需验证其对一个乘积基闭合:对任意 $U\in\mathcal T,\ V\in\mathcal S$,有
  \[
   (U\cap A)\times(V\cap B)=(U\times V)\cap(A\times B),
  \]
  右式为 $\tau_2$ 的基元(矩形与 $A\times B$ 的交),故 $\operatorname{id}$ 连续。
  
  反向连续性同理:对 $\tau_2$ 的基元 $(U\times V)\cap(A\times B)$,其像即 $(U\cap A)\times(V\cap B)$,属 $\tau_1$ 的基元,故 $\operatorname{id}^{-1}$ 连续。
  由此 $\tau_1=\tau_2$。
 \end{proof}
  
 \begin{lemma}[乘积中的常值截面连续]\label{lem:product-constant-sections}
  设拓扑空间 $X,Y$,取 $x_0\in X,\ y_0\in Y$。定义
  \[
   \eta_{x_0}:Y\to X\times Y,\ y\mapsto(x_0,y),\qquad
   \xi_{y_0}:X\to X\times Y,\ x\mapsto(x,y_0).
  \]
  则 $\eta_{x_0}$ 与 $\xi_{y_0}$ 均连续。
 \end{lemma}
 \begin{proof}
  对乘积基元 $U\times V$($U\subset X,\ V\subset Y$ 开),有
  $\eta_{x_0}^{-1}(U\times V)=\begin{cases} V,& x_0\in U,\\ \varnothing,& x_0\notin U,\end{cases}$
  与 $\xi_{y_0}^{-1}(U\times V)=\begin{cases} U,& y_0\in V,\\ \varnothing,& y_0\notin V.\end{cases}$
  皆为开集,故两映射连续。
 \end{proof}
 
 \begin{corollary}[联合连续蕴含偏连续]\label{cor:partial-continuity}
  若 $f:X\times Y\to Z$ 在 $(x_0,y_0)$ 处连续,则
  $y\mapsto f(x_0,y):Y\to Z$ 在 $y_0$ 连续,且 $x\mapsto f(x,y_0):X\to Z$ 在 $x_0$ 连续。
 \end{corollary}
 \begin{proof}
  由引理 \ref{lem:product-constant-sections},$\eta_{x_0},\xi_{y_0}$ 连续,故复合 $f\circ\eta_{x_0}$ 与 $f\circ\xi_{y_0}$ 连续,即得结论。
 \end{proof}
 
 \begin{proposition}[乘积的泛性质(坐标映射判别连续)]\label{prop:product-universal}
  设空间族 $\{X_\alpha\}_{\alpha\in\Lambda}$ 与空间 $Z$。给定映射 $f_\alpha:Z\to X_\alpha$($\forall\alpha$),则存在唯一映射
  \[
   f: Z\to\prod_{\alpha\in\Lambda} X_\alpha,\qquad z\mapsto (f_\alpha(z))_{\alpha\in\Lambda},
  \]
  满足 $p_\alpha\circ f=f_\alpha$。且 $f$ 连续当且仅当每个 $f_\alpha$ 连续。
 \end{proposition}
 \begin{proof}
  若 $g$ 亦满足 $p_\alpha\circ g=f_\alpha$,则 $p_\alpha(g(z))=p_\alpha(f(z))$ 对所有 $\alpha$,故 $g(z)=f(z)$。
  
  若 $f$ 连续,则 $f_\alpha=p_\alpha\circ f$ 为连续。
  
  反之,若每个 $f_\alpha$ 连续,对任意 $\alpha$ 与 $X_\alpha$ 中开集 $U$,有
  \[
   f^{-1}\big(p_\alpha^{-1}(U)\big)=(p_\alpha\circ f)^{-1}(U)=f_\alpha^{-1}(U)
  \]
  为 $Z$ 中开集。乘积拓扑由子基 $\{p_\alpha^{-1}(U)\}$ 生成,故 $f$ 连续。
 \end{proof}
 
 \begin{remark}
  上述命题中的坐标映射满足 $\forall\alpha:\ f_\alpha=p_\alpha\circ f$。
 \end{remark}
 
 \begin{proposition}[坐标连续推出乘积映射连续]\label{prop:product-map-continuous}
  设空间族 $\{X_\alpha\}_{\alpha\in\Lambda}$ 与 $\{Y_\alpha\}_{\alpha\in\Lambda}$。若对每个 $\alpha$,映射 $f_\alpha:X_\alpha\to Y_\alpha$ 连续,定义
  \[
   \prod_{\alpha} f_\alpha: \prod_{\alpha\in\Lambda} X_\alpha\longrightarrow \prod_{\alpha\in\Lambda} Y_\alpha,\qquad (x_\alpha)_{\alpha}\mapsto (f_\alpha(x_\alpha))_{\alpha},
  \]
  则 $\prod_{\alpha} f_\alpha$ 连续。
 \end{proposition}
 \begin{proof}
  对任意 $\beta\in\Lambda$,有交换恒等式
  \[
   p_\beta\circ\Big(\prod_{\alpha} f_\alpha\Big)= f_\beta\circ p_\beta.
  \]
  右端为连续映射复合,故左端连续。由命题 \ref{prop:product-universal}(坐标判别:$g$ 连续 $\iff$ $p_\alpha\circ g$ 连续 $\forall\alpha$),得 $\prod_{\alpha} f_\alpha$ 连续。
 \end{proof}
 
 \begin{definition}[拓扑和(不交并)]\label{def:topological-sum}
  设一族两两不交的拓扑空间 $(X_\alpha,\,\mathcal T_\alpha)_{\alpha\in\Lambda}$。记集合不交并
  \[
   \bigsqcup_{\alpha\in\Lambda} X_\alpha=\coprod_{\alpha\in\Lambda} X_\alpha.
  \]
  在该并集上定义拓扑 $\tau$,使每个包含映射
  \[
   i_\alpha:X_\alpha\hookrightarrow \bigsqcup_{\gamma\in\Lambda} X_\gamma
  \]
  连续且为嵌入(即 $i_\alpha:X_\alpha\to i_\alpha(X_\alpha)$ 为同胚)的最大拓扑。约定记号
  \[
   \coprod_{\alpha\in\Lambda} X_\alpha\equiv\big(\bigsqcup_{\alpha\in\Lambda} X_\alpha,\,\tau\big),
  \]
  并在需要时亦记 $\tau=\mathcal T_{\coprod}$。称 $(\bigsqcup_{\alpha}X_\alpha,\,\tau)$ 为拓扑和。
 \end{definition}
  
  \begin{proposition}[拓扑和的显式刻画]\label{prop:sum-topology-explicit}
  有
  \[
   \mathcal T_{\coprod}
   =\Big\{\,\bigsqcup_{\alpha\in\Lambda} U_\alpha:\ U_\alpha\in\mathcal T_\alpha\ (\forall\alpha)\,\Big\}.
  \]
 \end{proposition}
 \begin{proof}
  记右侧集合族为 $\mathcal J$。先证 $\mathcal J$ 为拓扑并使每个 $i_\alpha$ 连续:对任意 $\Omega=\bigsqcup_\alpha U_\alpha\in\mathcal J$,有 $i_\alpha^{-1}(\Omega)=U_\alpha\in\mathcal T_\alpha$,故 $i_\alpha$ 连续。显然 $\mathcal J$ 对任意并封闭;且
  \[
   \Big(\bigsqcup_\alpha U_\alpha\Big)\cap\Big(\bigsqcup_\alpha V_\alpha\Big)=\bigsqcup_\alpha (U_\alpha\cap V_\alpha)\in\mathcal J,
  \]
  故 $\mathcal J$ 为拓扑。
  
  因 $\mathcal T_{\coprod}$ 是使所有 $i_\alpha$ 连续的最大拓扑,得 $\mathcal J\subseteq\mathcal T_{\coprod}$。
  
  反之,任取 $\Omega\in\mathcal T_{\coprod}$,则 $i_\alpha^{-1}(\Omega)=\Omega\cap X_\alpha\in\mathcal T_\alpha$。令 $U_\alpha=\Omega\cap X_\alpha$,则
  \[
   \Omega=\bigsqcup_{\alpha\in\Lambda}(\Omega\cap X_\alpha)=\bigsqcup_{\alpha\in\Lambda} U_\alpha\in\mathcal J.
  \]
  故 $\mathcal T_{\coprod}\subseteq\mathcal J$,两者相等。
 \end{proof}

 \begin{lemma}[嵌入引理]\label{lem:product-coordinates}
  设 $X$ 为拓扑空间,$(Y_\alpha)_{\alpha\in\Lambda}$ 为拓扑空间族。令
  \[
   f:X\to \prod_{\alpha\in\Lambda} Y_\alpha,\qquad x\mapsto (f_\alpha(x))_{\alpha\in\Lambda},
  \]
  其中 $f_\alpha\equiv p_\alpha\circ f$。则
  \begin{enumerate}
    \item $f$ 连续当且仅当每个 $f_\alpha$ 连续。等价地,对任意 $x_0\in X$,$f$ 在 $x_0$ 连续当且仅当每个 $f_\alpha$ 在 $x_0$ 连续;
    \item 若 $f$ 为单射,则族 $\{f_\alpha\}$ 分离 $X$ 的点:对任意 $x\neq y$,存在 $\alpha$ 使 $f_\alpha(x)\neq f_\alpha(y)$;
    \item 若$\{f_\alpha\}$ 分离$X$的点和闭集,则 $f:X\to f(X)$ 为开映射。
  \end{enumerate}
 \end{lemma}
 
 \begin{proof}
  取任意开集 $U\subseteq X$ 与点 $x\in U$。由假设族 $\{f_\alpha\}$ 分离点与闭集,应用于闭集 $U^c$ 与点 $x$,存在指标 $\alpha$ 与 $Y_\alpha$ 中开邻域 $V_\alpha\ni f_\alpha(x)$ 使得 $f_\alpha(U^c)\subseteq V_\alpha^{\,c}$。
  因而 $f(x)\in p_\alpha^{-1}(V_\alpha)\cap f(X)$。对任意 $x'\in X$,若 $f(x')\in p_\alpha^{-1}(V_\alpha)\cap f(X)$,则 $f_\alpha(x')\in V_\alpha$,从而 $x'\notin U^c$;于是 $x'\in U$,并且 $f(x')\in f(U)$。
  故有
  \[
   p_\alpha^{-1}(V_\alpha)\cap f(X)\subseteq f(U).
  \]
  其中 $p_\alpha^{-1}(V_\alpha)$ 为乘积空间中的开集,故其与 $f(X)$ 的交在子空间 $f(X)$ 中开。于是 $f(U)$ 在 $f(X)$ 中对每点 $f(x)$ 含有开邻域,故 $f(U)$ 为 $f(X)$ 的开集。命题(3)得证。
 \end{proof}

 \begin{proposition}[子空间中的闭包]\label{prop:closure-in-subspace}
  设 $A\subseteq Y\subseteq X$。则在子空间 $Y$ 中 $A$ 的闭包满足
  \[
   \overline{A}^{\,Y}=\overline{A}\cap Y,
  \]
  其中右侧闭包在 $X$ 中取。
 \end{proposition}
 \begin{proof}
  “$\subseteq$”:设 $y\in\overline{A}^{\,Y}$。则对任意 $Y$ 中开邻域 $B\in\mathcal N_Y(y)$,有 $B\cap A\neq\varnothing$。取 $X$ 中开集 $U$ 使 $B=U\cap Y$,得 $U\cap A\neq\varnothing$,故 $y\in\overline{A}$,进而 $y\in\overline{A}\cap Y$。
  “$\supseteq$”:设 $y\in\overline{A}\cap Y$。任取 $B\in\mathcal N_Y(y)$,写作 $B=U\cap Y$ 且 $U\in\mathcal T$ 开。由 $y\in\overline{A}$ 知 $U\cap A\neq\varnothing$,故 $B\cap A\neq\varnothing$,从而 $y\in\overline{A}^{\,Y}$。
 \end{proof}
 
 \begin{corollary}\label{cor:dense-in-closure}
  对任意 $A\subseteq X$,$A$ 在 $\overline{A}$ 中稠密。等价地,
  \[
   \overline{A}^{\,\overline{A}}=\overline{A}\cap\overline{A}=\overline{A}.
  \]
 \end{corollary}
 
 \begin{corollary}\label{cor:dense-in-omega}
  若 $\Omega\subseteq X$ 且 $\Omega\subseteq\overline{A}$,则 $A$ 在 $\Omega$ 中稠密,即
  \[
   \overline{A}^{\,\Omega}=\overline{A}\cap\Omega=\Omega.
  \]
 \end{corollary}

 \begin{definition}[可度量化空间]\label{def:metrizable-space-2}
  设拓扑空间 $(X,\mathcal T)$。若存在度量 $d:X\times X\to[0,+\infty)$ 使由 $d$ 诱导的拓扑 $\mathcal T_d$ 与 $\mathcal T$ 相同,即 $\mathcal T_d=\mathcal T$,则称 $(X,\mathcal T)$ 为可度量化(metrizable)。其中
  \[
   \mathcal T_d \equiv \{\,U\subseteq X:\ \forall x\in U\ \exists \varepsilon>0\ \text{s.t.}\ B_d(x,\varepsilon)\subseteq U\,\}.
  \]
 \end{definition}

 \begin{definition}[完备可度量化空间]\label{def:completely-metrizable}
  设拓扑空间 $(X,\mathcal T)$。若存在度量 $d:X\times X\to[0,+\infty)$ 使 $\mathcal T_d=\mathcal T$ 且度量空间 $(X,d)$ 完备,则称 $(X,\mathcal T)$ 为完备可度量化(completely metrizable)。
 \end{definition}
  
\begin{proposition}[同胚不变性]\label{prop:metrizable-homeo-invariance}
  若 $X$ 与 $Y$ 同胚,则 $X$ 可完备度量化当且仅当 $Y$ 可完备度量化。
 \end{proposition}
 
 \begin{proof}
  设 $h:X\to Y$ 为同胚。先证“$\Rightarrow$”。若 $X$ 可完备度量化,则存在完备度量 $d_X$ 使 $\mathcal T_{d_X}=\mathcal T_X$。定义
  \[
   d_Y(y_1,y_2)\equiv d_X\big(h^{-1}(y_1),h^{-1}(y_2)\big),\qquad y_1,y_2\in Y.
  \]
  则 $h:(X,d_X)\to(Y,d_Y)$ 为等距同胚,因而 $\mathcal T_{d_Y}=\mathcal T_Y$。取任意 $(Y,d_Y)$ 的 Cauchy 列 $(y_n)$,则 $(x_n)=(h^{-1}(y_n))$ 满足
  \[
   d_X(x_n,x_m)=d_Y(y_n,y_m)\xrightarrow[n,m\to\infty]{}0,
  \]
  故 $(x_n)$ 为 $(X,d_X)$ 的 Cauchy 列,完备性给出 $x_n\to x$。由 $h$ 的等距连续性,$y_n=h(x_n)\to h(x)$,从而 $(Y,d_Y)$ 完备。于是 $Y$ 可完备度量化。
  反向“$\Leftarrow$”同理,取同胚 $h^{-1}:Y\to X$ 即得。
 \end{proof}
  
 \begin{definition}[等价度量]\label{def:equivalent-metrics}
  设 $X$ 为集合,$d_1,d_2$ 为 $X$ 上度量。称 $d_1$ 与 $d_2$ 一致等价,若恒等映射
  \[
   \operatorname{id}:(X,d_1)\longrightarrow (X,d_2)
  \]
  为一致同胚。
 \end{definition}
 
 \begin{proposition}[有界化度量与原度量等价]\label{prop:bounded-metric-equivalent}
  设 $d$ 为 $X$ 上度量,定义
  \[
   d_b(x,y)\equiv \frac{d(x,y)}{1+d(x,y)}\in[0,1).
  \]
  则 $d_b$ 为 $X$ 上度量,且 $\mathcal T_{d_b}=\mathcal T_d$,因而 $d$ 与 $d_b$ 等价。
 \end{proposition}
 \begin{proof}
  度量性:显然 $d_b(x,y)=d_b(y,x)$,且 $d_b(x,y)=0\iff d(x,y)=0\iff x=y$。对三角不等式,记 $g(t)=t/(1+t)$,则对任意 $a,b\ge0$,
  \[
   g(a)+g(b)-g(a+b)=\frac{ab}{(1+a)(1+b)(1+a+b)}\ge0,
  \]
  故 $g(a+b)\le g(a)+g(b)$。由 $d(x,z)\le d(x,y)+d(y,z)$ 与 $g$ 单调性得
  \[
   d_b(x,z)=g(d(x,z))\le g(d(x,y)+d(y,z))\le g(d(x,y))+g(d(y,z))=d_b(x,y)+d_b(y,z).
  \]
  拓扑等价:$g$ 严格增且 $g^{-1}(s)=s/(1-s)$($0\le s<1$)。于是对任意 $0<\varepsilon<1$,
  \[
   B_{d_b}(x,\varepsilon)=\{y:\ d_b(x,y)<\varepsilon\}=\{y:\ d(x,y)<\tfrac{\varepsilon}{1-\varepsilon}\}=B_d\!\Big(x,\frac{\varepsilon}{1-\varepsilon}\Big),
  \]
  两拓扑相同。
\end{proof}

 \begin{lemma}[可数积的度量化]\label{lem:countable-product-metrizable}
  设 $X_n$($n\in\mathbb N$)皆为可度量化空间,则其乘积 $\prod_{n=1}^{\infty}X_n$(赋以乘积拓扑)可度量化。
 \end{lemma}
 \begin{proof}
  对每个 $n$ 取与 $X_n$ 拓扑相容的度量 $d_n$。由命题 \ref{prop:bounded-metric-equivalent} 可不失一般性假设 $d_n\le1$。定义
  \[
   d\big((x_n),(y_n)\big)\equiv \sum_{n=1}^{\infty}\frac{1}{2^n}\,d_n(x_n,y_n).
  \]
  则 $d$ 为度量:对任意 $(x_n),(y_n),(z_n)$,由每个 $d_n$ 的三角不等式逐项相加得
  \[
   d\big((x_n),(y_n)\big)\le d\big((x_n),(z_n)\big)+d\big((z_n),(y_n)\big).
  \]

    若每个 $(X_n,d_n)$ 完备,则 $d$ 亦完备。任取 $d$-Cauchy 列 $\big((x_n^k)_n\big)_{k\in\mathbb N}$。由
  \[
   2^{-i}\,d_i(x_i^k,x_i^\ell)\le d\big((x_n^k)_n,(x_n^\ell)_n\big)\xrightarrow[k,\ell\to\infty]{}0,
  \]
  知对每个 $i$,$(x_i^k)_k$ 为 $(X_i,d_i)$ 的 Cauchy 列。由完备性,存在 $x_i\in X_i$ 使 $x_i^k\to x_i$。记 $x=(x_i)_i$。
  
  取任意 $\varepsilon>0$。先选 $N$ 使 $\sum_{n>N}2^{-n}<\varepsilon/2$。再为 $1\le i\le N$ 取 $\delta_i>0$,令 $\sum_{i=1}^N2^{-i}\delta_i<\varepsilon/2$。由 $x_i^k\to x_i$,对每个 $i\le N$ 存在 $K_i$ 使 $k\ge K_i$ 时 $d_i(x_i^k,x_i)<\delta_i$。令 $K=\max\{K_1,\dots,K_N\}$,则 $k\ge K$ 时
  \[
   d\big((x_n^k)_n,x\big)=\sum_{i=1}^{N}2^{-i}d_i(x_i^k,x_i)+\sum_{n>N}2^{-n}d_n(x_n^k,x_n)<\frac{\varepsilon}{2}+\frac{\varepsilon}{2}=\varepsilon.
  \]
  故 $((x_n^k)_n)_k$ 于 $d$ 收敛到 $x$,$d$ 完备。
  记 $p_i:\prod X_n\to X_i$ 为坐标投影。注意
  \[
   d\big((x_n),(y_n)\big)=\sum_{n=1}^{\infty}\frac{1}{2^n}d_n(x_n,y_n)\ge \frac{1}{2^i}d_i(x_i,y_i)\qquad(\forall i),
  \]
  故 $p_i:\big(\prod X_n,d\big)\to (X_i,d_i)$ 连续。从而
  \[
   \operatorname{id}:\big(\prod\nolimits_{n}X_n,d\big)\longrightarrow \big(\prod\nolimits_{n}X_n,\mathcal T_{\Pi}\big)
  \]
  连续(乘积拓扑由各 $p_i$ 生成)。
  
  取任意 $(x_n)$ 与 $\varepsilon>0$。选 $N$ 使 $\sum_{n>N}2^{-n}<\varepsilon/2$。由 $d_n\le1$ 得
  \[
   \sum_{n>N}\frac{1}{2^n}d_n(x_n,y_n)<\frac{\varepsilon}{2}\qquad(\forall (y_n)).
  \]
  为 $1\le i\le N$ 取 $\delta_i>0$,使 $\sum_{i=1}^{N}\tfrac{1}{2^i}\delta_i<\varepsilon/2$。令
  \[
   U\equiv B_{d_1}(x_1,\delta_1)\times\cdots\times B_{d_N}(x_N,\delta_N)\times\prod_{n>N}X_n,
  \]
  则 $U$ 是乘积拓扑的基本开邻域。对任意 $(y_n)\in U$,
  \[
   d\big((x_n),(y_n)\big)=\sum_{i=1}^{N}\frac{1}{2^i}d_i(x_i,y_i)+\sum_{n>N}\frac{1}{2^n}d_n(x_n,y_n)<\frac{\varepsilon}{2}+\frac{\varepsilon}{2}=\varepsilon,
  \]
  故 $U\subseteq B_d\big((x_n),\varepsilon\big)$。因而
  \[
   \operatorname{id}:\big(\prod\nolimits_{n}X_n,\mathcal T_{\Pi}\big)\longrightarrow \big(\prod\nolimits_{n}X_n,d\big)
  \]
  于 $(x_n)$ 连续,任意点同理。
  
  综上,$\mathcal T_d=\mathcal T_{\Pi}$,故 $\prod X_n$ 可度量化。
  
  
 \end{proof}
 
 \begin{theorem}[乘积中的网收敛判别]\label{thm:product-net-convergence}
  设 $\{X_i\}_{i\in I}$ 为一族拓扑空间。令 $\big((x_i^{\alpha})_{i\in I}\big)_{\alpha}$ 为 $\prod_{i\in I}X_i$ 中的一网,且 $(x_i)_{i\in I}\in\prod_{i\in I}X_i$。则
  \[
   (x_i^{\alpha})_{i\in I} \longrightarrow (x_i)_{i\in I}\iff \forall i\in I,\ x_i^{\alpha}\longrightarrow x_i\text{ 于 }X_i.
  \]
 \end{theorem}
 \begin{proof}
  “$\Rightarrow$”:各坐标投影 $p_i:\prod X_i\to X_i$ 连续,故由连续映像保持极限,$p_i\big((x_i^{\alpha})_i\big)=x_i^{\alpha}\to p_i\big((x_i)_i\big)=x_i$。
  
  “$\Leftarrow$”:设对每个 $i\in I$ 有 $x_i^{\alpha}\to x_i$。为证 $\big((x_i^{\alpha})_i\big)$ 在乘积空间中收敛,只需验证它对乘积拓扑的任一子基开集最终落入。取任意 $i_0\in I$ 与 $x_{i_0}$ 的开邻域 $U\subset X_{i_0}$,则由 $x_{i_0}^{\alpha}\to x_{i_0}$,存在 $\beta$,使得对所有 $\alpha\succeq\beta$,$x_{i_0}^{\alpha}\in U$。这当且仅当
  \[
   (x_i^{\alpha})_{i\in I}\in p_{i_0}^{-1}(U)\qquad(\forall\alpha\succeq\beta),
  \]
  即该网最终落入子基开集 $p_{i_0}^{-1}(U)$。由子基生成开集,遂该网在乘积拓扑下收敛到 $(x_i)_i$。
 \end{proof}
 
 
 \begin{definition}[波兰空间]\label{def:polish-space}
 称拓扑空间 $X$ 为波兰空间(Polish space),若 $X$ 可分且完全可度量化,即存在完备度量 $d$ 使由 $d$ 诱导的拓扑与 $X$ 的原拓扑一致(记作 $\mathcal T_X=\mathcal T_d$)。亦即 $X$ 为可分完全可度量化空间。
 \end{definition}
 
 \begin{theorem}[Urysohn 嵌入定理的等价刻画]\label{thm:urysohn-embedding}
  设拓扑空间 $X$,记 $I=[0,1]$ 与 $I^{\omega}=[0,1]^{\mathbb N}$。以下命题等价:
  \begin{enumerate}
    \item $X$ 为第二可数的 $T_3$ 空间;
    \item $X$ 为可分度量空间;
    \item $X$ 同胚于某个可分度量空间的子空间;
    \item $X$ 同胚于某个波兰空间的子空间;
    \item $X$ 可嵌入 $I^{\omega}$。
  \end{enumerate}
  此外,$I^{\omega}$ 为波兰空间。
  \end{theorem}
  \begin{proof}

    1 $\Rightarrow$ 5
    
  第二可数 $\Rightarrow$ Lindel\"{o}f;而 $T_3+$Lindel\"{o}f $\Rightarrow T_4$(正规),故可用 Urysohn 引理。
  取 $X$ 的可数基 $\mathcal B$。令
  \[
    \Lambda\equiv\{(U,V)\in\mathcal B\times\mathcal B:\ \overline{U}\subset V\}.
  \]
  对每个 $(U,V)\in\Lambda$,由 Urysohn 引理取连续映射
  \[
    f_{U,V}:X\to[0,1]\quad\text{满足}\quad f_{U,V}(\overline U)\subset\{0\},\ \ f_{U,V}(V^{c})\subset\{1\}.
  \]
  定义
  \[
    \Theta:X\longrightarrow [0,1]^{\Lambda},\qquad \Theta(x)\equiv\big(f_{U,V}(x)\big)_{(U,V)\in\Lambda}.
  \]
  各 $f_{U,V}$ 连续,故 $\Theta$ 连续;且 $\Lambda$ 可数,$[0,1]^{\Lambda}\cong I^{\omega}$。
  
  分离性质:任意 $x\in X$ 与闭集 $F\subset X$ 且 $x\notin F$,由正则性与基可选 $U,V\in\mathcal B$ 使 $x\in U$, $\overline U\subset V$ 且 $F\subset V^{c}$,于是
  \[
    f_{U,V}(x)=0,\qquad f_{U,V}(F)\subset\{1\}.
  \]
  因而 $\Theta(x)\notin \Theta(F)$,即该函数族将点与闭集分离,故 $\Theta$ 为嵌入,$X$ 与 $\Theta(X)$ 同胚,遂 $X$ 可嵌入 $I^{\omega}$。
  
  其余方向:$5\Rightarrow4$ 因 $I^{\omega}$ 为波兰空间;$4\Rightarrow2$ 因波兰空间为可分度量,子空间仍为可分度量;$2\Rightarrow3\Rightarrow2$ 显然;$2\Rightarrow1$ 因度量空间为 $T_3$ 且可分 $\Rightarrow$ 第二可数。
  \end{proof}
  

  
  \section{子网与聚点}
 

 
 \begin{lemma}\label{lem:subnet-limit}
 若 $x_{\alpha}\to x$,则任意子网 $(x_{\theta(\gamma)})_{\gamma\in\Gamma}$ 亦收敛到 $x$。
 \end{lemma}
 \begin{proof}
 任取 $U\in\mathcal{N}(x)$。由 $x_{\alpha}\to x$,存在 $\alpha_0\in\Lambda$ 使得 $\alpha\ge \alpha_0\Rightarrow x_{\alpha}\in U$。
 由 $\theta$ 余最终,存在 $\gamma_0\in\Gamma$ 使得 $\theta(\gamma_0)\ge \alpha_0$。于是当 $\gamma\succeq\gamma_0$ 时有 $\theta(\gamma)\ge \theta(\gamma_0)\ge \alpha_0$,从而 $x_{\theta(\gamma)}\in U$。
 故 $(x_{\theta(\gamma)})_{\gamma\in\Gamma}\to x$。
 \end{proof}
 
 \begin{theorem}\label{thm:cluster-subnet}
 设 $(x_{\alpha})_{\alpha\in\Lambda}$ 为拓扑空间 $X$ 中的网,$x\in X$。下列命题等价:
 \begin{enumerate}
 \item 存在 $(x_{\alpha})$ 的子网收敛到 $x$;
 \item $x\in\displaystyle\bigcap_{\alpha\in\Lambda}\overline{\{x_{\beta}:\beta\ge\alpha\}}$。
 \end{enumerate}
 \end{theorem}
 \begin{proof}
 (1)$\Rightarrow$(2) 设存在子网 $(x_{\alpha_i})_{i\in\Gamma}$ 满足 $x_{\alpha_i}\to x$。
 任取 $\alpha\in\Lambda$ 与邻域 $U\in\mathcal{N}(x)$。由子网余最终性,存在 $i_0\in\Gamma$ 使得 $\alpha_{i_0}\ge\alpha$;由收敛性,存在 $i_1\in\Gamma$ 使得 $i\succeq i_1\Rightarrow x_{\alpha_i}\in U$。
 取 $i\succeq i_0,i_1$,则 $x_{\alpha_i}\in U\cap\{x_{\beta}:\beta\ge\alpha\}$,故 $U\cap\{x_{\beta}:\beta\ge\alpha\}\neq\varnothing$,从而 $x\in\overline{\{x_{\beta}:\beta\ge\alpha\}}$。$\alpha$ 任意,得 (2)。
 
 (2)$\Rightarrow$(1) 令
 \[
 \Gamma=\{(U,\alpha)\in\mathcal{N}(x)\times\Lambda:\ x_{\alpha}\in U\}.
 \]
 在 $\Gamma$ 上定义偏序:对 $(U,\alpha),(V,\beta)\in\Gamma$,令
 \[
 (U,\alpha)\preceq (V,\beta)\iff U\supseteq V\ \text{且}\ \alpha\le\beta.
 \]
 下证 $(\Gamma,\preceq)$ 为定向集:任取 $(U,\alpha),(V,\beta)\in\Gamma$,令 $W=U\cap V\in\mathcal{N}(x)$,并取 $\gamma\in\Lambda$ 使得 $\gamma\ge\alpha,\beta$。
 由 (2) 得 $x\in\overline{\{x_{\delta}:\delta\ge\gamma\}}$,故 $W\cap\{x_{\delta}:\delta\ge\gamma\}\neq\varnothing$,于是存在 $\delta\ge\gamma$ 使得 $x_{\delta}\in W$。
 因而 $(W,\delta)\in\Gamma$ 且 $(U,\alpha)\preceq (W,\delta)$ 与 $(V,\beta)\preceq (W,\delta)$,从而 $(\Gamma,\preceq)$ 为定向集。
 
 定义 $\theta:\Gamma\to\Lambda$ 为 $\theta(U,\alpha)=\alpha$,则 $\theta$ 保序,且对任意 $\alpha_0\in\Lambda$,有 $(X,\alpha_0)\in\Gamma$ 且 $\theta(X,\alpha_0)=\alpha_0$,因此 $\theta$ 余最终。
 于是 $(x_{\theta(U,\alpha)})_{(U,\alpha)\in\Gamma}$ 为 $(x_{\alpha})_{\alpha\in\Lambda}$ 的子网。
 
 最后证该子网收敛到 $x$:任取 $U_0\in\mathcal{N}(x)$,并取任意 $\alpha_*\in\Lambda$。由 (2) 得 $U_0\cap\{x_{\beta}:\beta\ge\alpha_*\}\neq\varnothing$,故存在 $\beta\ge\alpha_*$ 使得 $x_{\beta}\in U_0$。
 令 $\gamma_0=(U_0,\beta)\in\Gamma$。若 $\gamma=(V,\alpha)\succeq\gamma_0$,则 $V\subseteq U_0$ 且 $x_{\theta(\gamma)}=x_{\alpha}\in V\subseteq U_0$。
 因此该子网最终落在 $U_0$ 中,即 $x_{\theta(\gamma)}\to x$。
 \end{proof}

\subsection{紧性与收敛子网}

\begin{theorem}[紧性的网刻画]\label{thm:compact-net}
  设 $X$ 为拓扑空间,则下列命题等价:
  \begin{enumerate}
    \item $X$ 紧;
    \item $X$ 中任意网都存在一个收敛子网。
  \end{enumerate}
\end{theorem}
\begin{proof}
  (1)$\Rightarrow$(2) 设 $(x_{\alpha})_{\alpha\in\Lambda}$ 为 $X$ 中任意网。对每个 $\alpha\in\Lambda$ 令
  \[
    F_{\alpha}=\overline{\{x_{\beta}:\beta\ge\alpha\}}.
  \]
  则 $\{F_{\alpha}\}_{\alpha\in\Lambda}$ 为闭集族,并具有有限交性质:任取 $\alpha_1,\dots,\alpha_n$,由 $\Lambda$ 定向性可取 $\gamma\in\Lambda$ 使 $\gamma\ge\alpha_i$($\forall i$),于是
  \[
    F_{\gamma}\subseteq \bigcap_{i=1}^n F_{\alpha_i},
  \]
  而 $F_{\gamma}\neq\varnothing$。由紧性知 $\bigcap_{\alpha\in\Lambda}F_{\alpha}\neq\varnothing$,取 $x\in\bigcap_{\alpha}F_{\alpha}$。由定理 \ref{thm:cluster-subnet},存在 $(x_{\alpha})$ 的子网收敛到 $x$。

  (2)$\Rightarrow$(1) 反证。若 $X$ 非紧,则存在开覆盖 $\mathcal U$ 使其无有限子覆盖。令
  \[
    \mathcal B=\{\,\mathcal F\subseteq\mathcal U: \mathcal F\ \text{有限}\,\},
  \]
  以包含关系 $\subseteq$ 作偏序,则 $(\mathcal B,\subseteq)$ 为定向集。对每个 $\mathcal F\in\mathcal B$,因 $\mathcal F$ 不能覆盖 $X$,可取
  \[
    x_{\mathcal F}\in X\setminus\bigcup\mathcal F.
  \]
  从而得到网 $(x_{\mathcal F})_{\mathcal F\in\mathcal B}$。任取其子网 $(x_{\theta(\gamma)})_{\gamma\in\Gamma}$。若该子网收敛到某点 $x\in X$,取 $U\in\mathcal U$ 使 $x\in U$。
  令 $\mathcal F_0=\{U\}\in\mathcal B$。由 $\theta$ 余最终性可取 $\gamma_0\in\Gamma$ 使 $\theta(\gamma_0)\supseteq\mathcal F_0$,于是对任意 $\gamma\succeq\gamma_0$ 都有 $\theta(\gamma)\supseteq\mathcal F_0$,从而
  \[
    x_{\theta(\gamma)}\in X\setminus\bigcup\theta(\gamma)\subseteq X\setminus U.
  \]
  因此子网最终不在邻域 $U$ 中,与 $x_{\theta(\gamma)}\to x$ 矛盾。故该网不存在收敛子网,矛盾。于是 $X$ 必紧。
\end{proof}

\subsection{拓扑和的连续性与可度量性}

\begin{proposition}\label{prop:coprod-continuity}
  设 $(X_\alpha,\mathcal T_\alpha)_{\alpha\in\Lambda}$ 为两两不交的拓扑空间族,其拓扑和记为 $\coprod_{\alpha\in\Lambda}X_\alpha$(见定义 \ref{def:topological-sum}),包含映射记为 $i_\alpha:X_\alpha\hookrightarrow\coprod_{\gamma}X_\gamma$。
  对任意拓扑空间 $Y$ 与映射 $f:\coprod_{\alpha\in\Lambda}X_\alpha\to Y$,下列命题等价:
  \begin{enumerate}
    \item $f$ 连续;
    \item 对每个 $\alpha\in\Lambda$,复合 $f\circ i_\alpha:X_\alpha\to Y$ 连续。
  \end{enumerate}
\end{proposition}
\begin{proof}
  (1)$\Rightarrow$(2) 因 $i_\alpha$ 连续,故 $f\circ i_\alpha$ 连续。

  (2)$\Rightarrow$(1) 任取 $Y$ 中开集 $V$。对每个 $\alpha$,令 $U_\alpha=(f\circ i_\alpha)^{-1}(V)$,则 $U_\alpha\in\mathcal T_\alpha$。
  由拓扑和的显式刻画(命题 \ref{prop:sum-topology-explicit})知
  \[
    \bigsqcup_{\alpha\in\Lambda} i_\alpha(U_\alpha)
  \]
  为 $\coprod X_\alpha$ 的开集。另一方面,对任意 $x\in X_\alpha$,有 $i_\alpha(x)\in f^{-1}(V)\iff x\in U_\alpha$,故
  \[
    f^{-1}(V)=\bigsqcup_{\alpha\in\Lambda} i_\alpha(U_\alpha)
  \]
  为开集,从而 $f$ 连续。
\end{proof}

\begin{proposition}\label{prop:coprod-metrizable}
  若对每个 $\alpha\in\Lambda$,空间 $X_\alpha$ 可度量化,则拓扑和 $\coprod_{\alpha\in\Lambda}X_\alpha$ 亦可度量化。
\end{proposition}
\begin{proof}
  对每个 $\alpha$ 取与 $X_\alpha$ 拓扑相容的度量 $d_\alpha$。由命题 \ref{prop:bounded-metric-equivalent} 可不失一般性设 $d_\alpha\le 1$。
  在集合不交并 $\bigsqcup_{\alpha\in\Lambda}X_\alpha$ 上定义函数 $d$:若 $x,y\in X_\alpha$,则令 $d(x,y)=d_\alpha(x,y)$;若 $x\in X_\alpha$ 与 $y\in X_\beta$ 且 $\alpha\neq\beta$,则令 $d(x,y)=1$。
  易验证 $d$ 为度量。记由 $d$ 诱导的拓扑为 $\mathcal T_d$。

  先证 $\mathcal T_d\subseteq\mathcal T_{\coprod}$:任取 $\mathcal T_d$ 中开集 $O$,对每个 $\alpha$,子空间 $O\cap X_\alpha$ 在 $(X_\alpha,d_\alpha)$ 中开,于是
  \[
    O=\bigsqcup_{\alpha\in\Lambda}(O\cap X_\alpha)
  \]
  由命题 \ref{prop:sum-topology-explicit} 知 $O\in\mathcal T_{\coprod}$。

  再证 $\mathcal T_{\coprod}\subseteq\mathcal T_d$:任取 $\mathcal T_{\coprod}$ 中开集 $O=\bigsqcup_{\alpha}U_\alpha$($U_\alpha\in\mathcal T_\alpha$)。任取 $x\in O$,设 $x\in U_\alpha$。
  由 $U_\alpha$ 在 $(X_\alpha,d_\alpha)$ 中开,可取 $\varepsilon\in(0,1)$ 使 $B_{d_\alpha}(x,\varepsilon)\subseteq U_\alpha$。此时 $B_d(x,\varepsilon)=B_{d_\alpha}(x,\varepsilon)\subseteq O$,故 $O$ 在 $\mathcal T_d$ 中开。

  综上 $\mathcal T_d=\mathcal T_{\coprod}$,从而 $\coprod X_\alpha$ 可度量化。
\end{proof}

\begin{lemma}[粘贴引理]\label{lem:gluing}
  设 $f:X\to Y$ 为映射,$\{U_i\}_{i\in I}$ 为 $X$ 的开覆盖。若对每个 $i\in I$,限制 $f|_{U_i}:U_i\to Y$ 连续,则 $f$ 连续。
\end{lemma}
\begin{proof}
  任取 $Y$ 中开集 $V$。对每个 $i$,有
  \[
    (f|_{U_i})^{-1}(V)=U_i\cap f^{-1}(V)
  \]
  在 $X$ 中开。于是
  \[
    f^{-1}(V)=\bigcup_{i\in I}\big(U_i\cap f^{-1}(V)\big)=\bigcup_{i\in I}(f|_{U_i})^{-1}(V)
  \]
  为开集,故 $f$ 连续。
\end{proof}

\subsection{Polish 空间的封闭性}

\begin{theorem}\label{thm:polish-product-sum}
  设 $\{X_n\}_{n\in\mathbb N}$ 为 Polish 空间列,则
  \begin{enumerate}
    \item $\prod_{n\in\mathbb N}X_n$ 为 Polish 空间;
    \item $\coprod_{n\in\mathbb N}X_n$ 为 Polish 空间。
  \end{enumerate}
\end{theorem}
\begin{proof}
  由定义 \ref{def:polish-space},需证可分与完全可度量化。

  (1) 对每个 $n$ 取与 $X_n$ 拓扑相容且完备的度量 $d_n$,并由命题 \ref{prop:bounded-metric-equivalent} 可设 $d_n\le 1$。在 $\prod_n X_n$ 上定义
  \[
    d\big((x_n),(y_n)\big)=\sum_{n=1}^{\infty}2^{-n}d_n(x_n,y_n).
  \]
  则 $d$ 为度量,且与乘积拓扑相容(与引理 \ref{lem:countable-product-metrizable} 的构造一致)。若 $(x^{(k)})$ 在 $d$ 下为 Cauchy,则对每个 $n$,坐标列 $(x^{(k)}_n)$ 在 $d_n$ 下为 Cauchy,因 $d_n$ 完备可得 $x^{(k)}_n\to x_n$。令 $x=(x_n)$,可验证 $x^{(k)}\to x$,故 $d$ 完备,乘积空间完全可度量化。

  再证可分:对每个 $n$ 取可数稠密集 $D_n\subseteq X_n$,并固定一点 $x_n^0\in D_n$。令
  \[
    D=\bigcup_{m\in\mathbb N}\Big(\prod_{n=1}^{m}D_n\times\prod_{n>m}\{x_n^0\}\Big)\subseteq\prod_{n\in\mathbb N}X_n.
  \]
  由引理 \ref{lem:countable-set-closure} 知对每个 $m$,$\prod_{n=1}^{m}D_n$ 可数,且上式为可数并,故 $D$ 可数。
  另一方面,乘积拓扑的基开集由有限多个坐标决定:任取非空基开集
  \[
    O=\bigcap_{k=1}^{m}p_{n_k}^{-1}(U_{n_k}),\qquad U_{n_k}\subseteq X_{n_k}\ \text{开},
  \]
  由 $D_{n_k}$ 在 $X_{n_k}$ 中稠密可取 $y_{n_k}\in D_{n_k}\cap U_{n_k}$。定义点 $y=(y_n)$:当 $n\in\{n_1,\dots,n_m\}$ 时取 $y_n=y_{n}$,其余坐标取 $x_n^0$。则 $y\in D\cap O$,故 $D$ 在 $\prod_n X_n$ 中稠密,从而 $\prod_n X_n$ 可分。

  (2) 对每个 $n$ 取与 $X_n$ 拓扑相容且完备的度量 $d_n$,并可设 $d_n\le 1$。在不交并 $\bigsqcup_n X_n$ 上定义度量
  \[
    d(x,y)=\begin{cases}
      d_n(x,y), & x,y\in X_n,\\
      1, & x\in X_m,\ y\in X_n,\ m\ne n.
    \end{cases}
  \]
  则由命题 \ref{prop:coprod-metrizable} 的论证知该度量诱导拓扑和拓扑,且因每个 $d_n$ 完备可得 $d$ 完备,从而 $\coprod_n X_n$ 完全可度量化。

  取每个 $n$ 的可数稠密集 $D_n\subseteq X_n$,则 $\bigsqcup_n D_n$ 为可数集且在 $\coprod_n X_n$ 中稠密,故 $\coprod_n X_n$ 可分。
\end{proof}

\subsection{可数性引理与可数乘积的第二可数性}

\begin{lemma}\label{lem:countable-set-closure}
  \begin{enumerate}
    \item 可数集的有限直积仍可数;
    \item 可数多个可数集的并仍可数。
  \end{enumerate}
\end{lemma}
\begin{proof}
  (1) 设 $A,B$ 可数,则存在单射 $A\hookrightarrow\mathbb N$ 与 $B\hookrightarrow\mathbb N$,故 $A\times B\hookrightarrow\mathbb N\times\mathbb N$。
  而 $\mathbb N\times\mathbb N$ 可数,从而 $A\times B$ 可数。归纳即得有限直积情形。

  (2) 设 $\{A_n\}_{n\in\mathbb N}$ 皆可数,则存在单射 $A_n\hookrightarrow\mathbb N$,故
  \[
    \bigcup_{n\in\mathbb N}A_n\hookrightarrow\mathbb N\times\mathbb N,
  \]
  从而该并为可数。
\end{proof}

\begin{proposition}\label{prop:countable-product-ca2}
  设 $\{X_n\}_{n\in\mathbb N}$ 为拓扑空间列。若每个 $X_n$ 第二可数(即 $CA_2$),则乘积空间 $\prod_{n\in\mathbb N}X_n$ 第二可数。
\end{proposition}
\begin{proof}
  对每个 $n$ 取 $X_n$ 的可数基 $\mathcal B_n$。乘积拓扑的一个基可取为所有形如
  \[
    \bigcap_{k=1}^m p_{n_k}^{-1}(B_{n_k}),\qquad B_{n_k}\in\mathcal B_{n_k},\ n_1,\dots,n_m\in\mathbb N
  \]
  的集合(其中 $m\ge 1$,$p_n$ 为投影)。对固定的 $(n_1,\dots,n_m)$,由引理 \ref{lem:countable-set-closure}(1) 知选择 $B_{n_k}$ 的方式可数;再对所有有限序列 $(n_1,\dots,n_m)$ 取并,由引理 \ref{lem:countable-set-closure}(2) 知总的基仍可数。
  故 $\prod_n X_n$ 第二可数。
\end{proof}

\subsection{连通性与连续像}

\begin{definition}[连通与不连通]\label{def:connected}
  设 $X$ 为拓扑空间。
  若存在非空开集 $U,V\subseteq X$ 使得
  \[
    U\cap V=\varnothing,\qquad X=U\cup V,
  \]
  则称 $X$ \emph{不连通};否则称 $X$ \emph{连通}。
\end{definition}

\begin{proposition}\label{prop:connected-image}
  设 $f:X\to Y$ 连续,$\Omega\subseteq X$ 连通,则 $f(\Omega)\subseteq Y$ 连通。
\end{proposition}
\begin{proof}
  反证。若 $f(\Omega)$ 不连通,则存在 $Y$ 中互不相交的非空开集 $U,V$ 使得
  \[
    f(\Omega)\subseteq U\cup V,\qquad f(\Omega)\cap U\ne\varnothing,\ \ f(\Omega)\cap V\ne\varnothing.
  \]
  因 $f$ 连续,$f^{-1}(U),f^{-1}(V)$ 为 $X$ 中开集,于是 $\Omega\cap f^{-1}(U)$ 与 $\Omega\cap f^{-1}(V)$ 在子空间 $\Omega$ 中开,且非空、互不相交,并满足
  \[
    \Omega\subseteq\big(\Omega\cap f^{-1}(U)\big)\cup\big(\Omega\cap f^{-1}(V)\big),
  \]
  与 $\Omega$ 连通矛盾。
\end{proof}
\begin{corollary}\label{cor:connected-surjective-image}
  若 $X$ 连通且 $f:X\to Y$ 连续满射,则 $Y$ 连通。
\end{corollary}

\begin{proposition}\label{prop:disconnected-clopen}
  设 $X$ 为拓扑空间,则以下等价:
  \begin{enumerate}
    \item $X$ 不连通;
    \item 存在子集 $\Omega\subseteq X$ 满足 $\varnothing\neq\Omega\neq X$ 且 $\Omega$ 在 $X$ 中既开又闭。
  \end{enumerate}
\end{proposition}
\begin{proof}
  (1)$\Rightarrow$(2) 若 $X$ 不连通,则存在非空开集 $U,V\subseteq X$ 使得
  \[
    U\cap V=\varnothing,\qquad X=U\cup V.
  \]
  则 $U$ 既开又闭,且 $\varnothing\neq U\neq X$。

  (2)$\Rightarrow$(1) 若 $\Omega$ 非空真子集且既开又闭,则 $X=\Omega\cup\Omega^{c}$ 且二者为非空不交开集,故 $X$ 不连通。
\end{proof}

\begin{lemma}\label{lem:connected-meets-clopen}
  设 $A\subseteq X$ 在子空间拓扑下连通,且 $\Omega\subseteq X$ 在 $X$ 中既开又闭。
  则要么 $A\cap\Omega=\varnothing$,要么 $A\subseteq\Omega$。
\end{lemma}
\begin{proof}
  注意到 $A\cap\Omega$ 与 $A\cap\Omega^{c}$ 在子空间 $A$ 中既开又闭,且
  \[
    A=(A\cap\Omega)\cup(A\cap\Omega^{c}),\qquad (A\cap\Omega)\cap(A\cap\Omega^{c})=\varnothing.
  \]
  由命题 \ref{prop:disconnected-clopen}(应用于空间 $A$)知 $A\cap\Omega=\varnothing$ 或 $A\cap\Omega^{c}=\varnothing$,后一种情形等价于 $A\subseteq\Omega$。
\end{proof}

\begin{lemma}\label{lem:union-connected-with-common-point}
  设 $\{A_\alpha\}_{\alpha\in\Lambda}$ 为 $X$ 中一族连通子集,且 $\bigcap_{\alpha\in\Lambda}A_\alpha\neq\varnothing$。
  则 $\bigcup_{\alpha\in\Lambda}A_\alpha$ 连通。
\end{lemma}
\begin{proof}
  反证。若 $\bigcup_{\alpha}A_\alpha$ 不连通,则存在其子空间中的不交非空开集 $U,V$ 使
  \[
    \bigcup_{\alpha\in\Lambda}A_\alpha=U\cup V.
  \]
  取 $x_0\in\bigcap_{\alpha}A_\alpha$,不妨设 $x_0\in U$。对任意 $\alpha$,集合 $A_\alpha\cap U$ 与 $A_\alpha\cap V$ 在 $A_\alpha$ 中开且不交,并满足
  \[
    A_\alpha=(A_\alpha\cap U)\cup(A_\alpha\cap V).
  \]
  由 $A_\alpha$ 连通及 $x_0\in A_\alpha\cap U$ 知 $A_\alpha\cap V=\varnothing$,即 $A_\alpha\subseteq U$。
  于是 $\bigcup_{\alpha}A_\alpha\subseteq U$,与 $V$ 非空矛盾。
\end{proof}

\begin{definition}[连通分支]\label{def:connected-component}
  设 $X$ 为拓扑空间。对 $x,y\in X$,定义关系
  \[
    x\sim y\iff \exists\ \text{连通子集 }A\subseteq X\ \text{使}\ x,y\in A.
  \]
  称等价类 $[x]=\{y\in X:y\sim x\}$ 为点 $x$ 所在的\emph{连通分支}(亦称连通分量)。
\end{definition}

\begin{proposition}\label{prop:connected-component-eqrel}
  上述关系 $\sim$ 为等价关系。
\end{proposition}
\begin{proof}
  自反性:取 $A=\{x\}$。

  对称性:由定义显然。

  传递性:若 $x\sim y$ 与 $y\sim z$,则存在连通集 $A,B$ 使 $x,y\in A$ 且 $y,z\in B$。
  此时 $A\cap B\ni y$,由引理 \ref{lem:union-connected-with-common-point} 知 $A\cup B$ 连通,且 $x,z\in A\cup B$,故 $x\sim z$。
\end{proof}

\begin{proposition}\label{prop:connected-component-maximal}
  对任意 $x\in X$,连通分支 $[x]$ 连通,且是包含 $x$ 的极大连通子集:若 $C\subseteq X$ 连通且 $x\in C$,则 $C\subseteq[x]$。
\end{proposition}
\begin{proof}
  对每个 $y\in[x]$,取连通子集 $A_y\subseteq X$ 使 $x,y\in A_y$。则 $x\in\bigcap_{y\in[x]}A_y$,且
  \[
    [x]=\bigcup_{y\in[x]}A_y.
  \]
  由引理 \ref{lem:union-connected-with-common-point} 得 $[x]$ 连通。

  若 $C$ 连通且 $x\in C$,则对任意 $y\in C$,取 $A=C$ 即有 $x\sim y$,故 $y\in[x]$,从而 $C\subseteq[x]$。
\end{proof}

\begin{proposition}\label{prop:connected-between-closure}
  若 $A\subseteq X$ 连通,且 $A\subseteq \Omega\subseteq \overline{A}$,则 $\Omega$ 连通。
\end{proposition}
\begin{proof}
  反证。若 $\Omega$ 不连通,则存在 $\Omega$ 的不交非空开集 $U,V$ 使 $\Omega=U\cup V$。
  因 $A$ 连通,必有 $A\subseteq U$ 或 $A\subseteq V$;不妨设 $A\subseteq U$。
  注意到 $U$ 在子空间 $\Omega$ 中既开又闭,故 $\overline{A}^{\,\Omega}\subseteq U$。
  但由命题 \ref{prop:closure-in-subspace} 有
  \[
    \overline{A}^{\,\Omega}=\overline{A}\cap\Omega=\Omega,
  \]
  从而 $\Omega\subseteq U$,与 $V$ 非空矛盾。
\end{proof}

\begin{corollary}\label{cor:closure-of-connected}
  若 $A\subseteq X$ 连通,则 $\overline{A}$ 连通。
\end{corollary}

\begin{proposition}\label{prop:connected-components-closed}
  $X$ 的每个连通分支都是闭集。
\end{proposition}
\begin{proof}
  设 $C=[x]$ 为连通分支。由推论 \ref{cor:closure-of-connected} 知 $\overline{C}$ 连通,且 $C\subseteq\overline{C}$。
  由命题 \ref{prop:connected-component-maximal} 的极大性,必有 $\overline{C}\subseteq C$,故 $\overline{C}=C$,即 $C$ 闭。
\end{proof}

\begin{proposition}\label{prop:product-connected-2}
  若 $X$ 与 $Y$ 连通,则 $X\times Y$ 连通。
\end{proposition}
\begin{proof}
  固定 $(x_0,y_0)\in X\times Y$。对任意 $(x,y)\in X\times Y$,令
  \[
    A_{(x,y)}=(X\times\{y_0\})\cup(\{x\}\times Y).
  \]
  其中 $X\times\{y_0\}$ 与 $\{x\}\times Y$ 分别同胚于 $X$ 与 $Y$,故皆连通;且二者交于 $(x,y_0)$,由引理 \ref{lem:union-connected-with-common-point} 得 $A_{(x,y)}$ 连通。
  又 $(x_0,y_0)\in A_{(x,y)}$,并且
  \[
    X\times Y=\bigcup_{(x,y)\in X\times Y}A_{(x,y)}.
  \]
  再次应用引理 \ref{lem:union-connected-with-common-point}(公共点为 $(x_0,y_0)$)得 $X\times Y$ 连通。
\end{proof}

\begin{theorem}\label{thm:arbitrary-product-connected}
  设 $\{X_\alpha\}_{\alpha\in\Lambda}$ 为一族连通拓扑空间,则其乘积 $\prod_{\alpha\in\Lambda}X_\alpha$(赋以乘积拓扑)连通。
\end{theorem}
\begin{proof}
  在每个 $X_\alpha$ 中固定一点 $x_\alpha^0\in X_\alpha$,记 $x^0=(x_\alpha^0)_\alpha\in\prod_\alpha X_\alpha$。

  定义
  \[
    \Omega=\Big\{(y_\alpha)_\alpha\in\prod_{\alpha\in\Lambda}X_\alpha:\ \{\alpha\in\Lambda: y_\alpha\ne x_\alpha^0\}\ \text{为有限集}\Big\}.
  \]
  对任意有限子集 $F\subseteq\Lambda$,令
  \[
    \Omega_F=\{(y_\alpha)_\alpha:\ y_\alpha=x_\alpha^0\ \text{对}\ \alpha\notin F\}.
  \]
  则 $\Omega=\bigcup_{F\subseteq\Lambda\ \text{有限}}\Omega_F$,且每个 $\Omega_F$ 同胚于有限乘积 $\prod_{\alpha\in F}X_\alpha$,故由命题 \ref{prop:product-connected-2}(归纳)知 $\Omega_F$ 连通。
  又 $x^0\in\bigcap_F\Omega_F$,由引理 \ref{lem:union-connected-with-common-point} 得 $\Omega$ 连通。

  下面证 $\Omega$ 在 $\prod_{\alpha}X_\alpha$ 中稠密。任取非空基开集
  \[
    B=\bigcap_{k=1}^{n}p_{\alpha_k}^{-1}(U_k),\qquad U_k\subseteq X_{\alpha_k}\ \text{开且非空},
  \]
  取 $y_{\alpha_k}\in U_k$,并令 $y_\alpha=x_\alpha^0$($\alpha\notin\{\alpha_1,\dots,\alpha_n\}$),则 $y\in\Omega\cap B$,故 $\Omega$ 稠密。

  由推论 \ref{cor:closure-of-connected},$\overline{\Omega}$ 连通;而 $\Omega$ 稠密意味着 $\overline{\Omega}=\prod_{\alpha}X_\alpha$,从而乘积空间连通。
\end{proof}

\subsection{$\mathbb R$ 的连通性、介值性与道路连通}

\begin{proposition}\label{prop:R-connected}
  赋以通常拓扑的实直线 $\mathbb R$ 连通。
\end{proposition}
\begin{proof}
  反证。若 $\mathbb R$ 不连通,则存在非空不交开集 $U,V\subseteq\mathbb R$ 使 $\mathbb R=U\cup V$。
  取 $x\in U$ 与 $y\in V$,不妨设 $x<y$。
  令
  \[
    a=\sup\{z\in U:\ z<y\}.
  \]
  则 $x\le a\le y$。若 $a\in U$,由 $U$ 开可取 $\varepsilon>0$ 使 $(a-\varepsilon,a+\varepsilon)\subseteq U$,从而可取 $z\in U$ 满足 $a<z<y$,与 $a$ 为上确界矛盾。
  因此 $a\notin U$,由 $\mathbb R=U\cup V$ 得 $a\in V$。由 $V$ 开可取 $\delta>0$ 使 $(a-\delta,a+\delta)\subseteq V$。
  但由上确界定义可取 $z\in U$ 使 $a-\delta<z\le a$,于是 $z\in U\cap(a-\delta,a+\delta)\subseteq U\cap V$,矛盾。
\end{proof}

\begin{proposition}\label{prop:interval-connected}
  $\mathbb R$ 中任意区间(如 $(a,b),\ [a,b],\ (a,b],\ [a,b),\ (a,+\infty),\ [a,+\infty),\ (-\infty,a),\ (-\infty,a]$ 等)在子空间拓扑下连通。
\end{proposition}
\begin{proof}
  先证开区间:由连续双射
  \[
    \varphi:(-\tfrac\pi2,\tfrac\pi2)\to\mathbb R,\qquad t\mapsto\tan t
  \]
  为同胚,知 $(-\tfrac\pi2,\tfrac\pi2)$ 与 $\mathbb R$ 同胚。又仿射映射将 $(-\tfrac\pi2,\tfrac\pi2)$ 同胚到任意 $(a,b)$($a<b$),故 $(a,b)$ 与 $\mathbb R$ 同胚,从而由命题 \ref{prop:R-connected} 知 $(a,b)$ 连通。

  其余类型区间可由开区间的闭包与并得到,例如当 $a<b$ 时有 $[a,b]=\overline{(a,b)}$,由推论 \ref{cor:closure-of-connected} 得 $[a,b]$ 连通;同理 $[a,b)=(a,b)\cup\{a\}$、$(a,b]=(a,b)\cup\{b\}$ 亦连通。
  半无穷区间可写作可数并,例如 $(a,+\infty)=\bigcup_{n\ge 1}(a,a+n)$,且这些区间均含公共点 $a+\tfrac12$,由引理 \ref{lem:union-connected-with-common-point} 得其连通;其他半无穷区间同理。
  空集与单点集显然连通。
\end{proof}

\begin{proposition}\label{prop:connected-subset-of-R-interval}
  设 $\Omega\subseteq\mathbb R$。
  则 $\Omega$ 连通当且仅当对任意 $x,y\in\Omega$ 且 $x<y$,有 $(x,y)\subseteq\Omega$。
\end{proposition}
\begin{proof}
  “$\Rightarrow$”:若存在 $x<y$ 属于 $\Omega$ 但某 $z\in(x,y)$ 不在 $\Omega$,则
  \[
    \Omega=\big(\Omega\cap(-\infty,z)\big)\cup\big(\Omega\cap(z,+\infty)\big)
  \]
  为 $\Omega$ 的分离(两部分在 $\Omega$ 中开、非空且不交),矛盾。

  “$\Leftarrow$”:若对任意 $x<y$ 均有 $(x,y)\subseteq\Omega$,则 $\Omega$ 为 $\mathbb R$ 的区间(或空集/单点集),由命题 \ref{prop:interval-connected} 知其连通。
\end{proof}

\begin{theorem}\label{thm:connected-in-R-iff-interval}
  对任意 $\Omega\subseteq\mathbb R$,$\Omega$ 连通当且仅当 $\Omega$ 是区间(允许退化情形:空集或单点集)。
\end{theorem}
\begin{proof}
  由命题 \ref{prop:connected-subset-of-R-interval} 即得。
\end{proof}

\begin{theorem}[介值定理]\label{thm:ivt}
  设 $f:[a,b]\to\mathbb R$ 连续且 $a<b$。若 $f(a)\le 0\le f(b)$ 或 $f(b)\le 0\le f(a)$,则存在 $z\in[a,b]$ 使 $f(z)=0$。
\end{theorem}
\begin{proof}
  由命题 \ref{prop:interval-connected} 知 $[a,b]$ 连通,故由命题 \ref{prop:connected-image} 知 $f([a,b])$ 连通。
  又 $f(a)$ 与 $f(b)$ 分别在 $0$ 的两侧(或一侧含等号),由定理 \ref{thm:connected-in-R-iff-interval} 知连通集 $f([a,b])$ 必包含 $0$。
  因此存在 $z\in[a,b]$ 使 $f(z)=0$。
\end{proof}

\begin{definition}[道路与道路连通]\label{def:path-path-connected}
  记 $I=[0,1]$。
  \begin{enumerate}
    \item 称连续映射 $\alpha:I\to X$ 为 $X$ 中一条\emph{道路}(或路径)。点 $\alpha(0)$ 与 $\alpha(1)$ 分别称为道路的起点与终点。
    \item 若对任意 $x,y\in X$,存在道路 $\alpha:I\to X$ 使 $\alpha(0)=x$ 且 $\alpha(1)=y$,则称 $X$ \emph{道路连通}(或路径连通)。
  \end{enumerate}
\end{definition}

\begin{proposition}\label{prop:path-connected-implies-connected}
  若 $X$ 道路连通,则 $X$ 连通。
\end{proposition}
\begin{proof}
  固定 $x_0\in X$。对任意 $x\in X$,取连接 $x_0$ 与 $x$ 的道路 $\alpha_x:I\to X$。
  因 $I$ 连通且 $\alpha_x$ 连续,$\alpha_x(I)$ 连通。
  又 $x_0\in\alpha_x(I)$ 对所有 $x$ 都成立,且
  \[
    X=\bigcup_{x\in X}\alpha_x(I).
  \]
  由引理 \ref{lem:union-connected-with-common-point} 知 $X$ 连通。
\end{proof}

\begin{lemma}[粘贴引理(有限闭覆盖版)]\label{lem:gluing-closed}
  设 $f:X\to Y$ 为映射,且 $X$ 可写为有限个闭集的并:$X=\bigcup_{k=1}^{n}F_k$,其中每个 $F_k$ 在 $X$ 中闭。
  若对每个 $k$,限制 $f|_{F_k}$ 连续,且对任意 $i,j$ 有 $f|_{F_i\cap F_j}$ 一致,则 $f$ 连续。
\end{lemma}
\begin{proof}
  任取 $Y$ 中闭集 $C$。对每个 $k$,因 $f|_{F_k}$ 连续,集合 $(f|_{F_k})^{-1}(C)$ 在子空间 $F_k$ 中闭,故存在 $X$ 中闭集 $K_k$ 使
  \[
    (f|_{F_k})^{-1}(C)=F_k\cap K_k.
  \]
  又
  \[
    f^{-1}(C)=\bigcup_{k=1}^{n}(f|_{F_k})^{-1}(C)=\bigcup_{k=1}^{n}(F_k\cap K_k).
  \]
  右端为有限个闭集的并,仍为闭集。故 $f^{-1}(C)$ 闭,对任意闭集 $C$ 成立,从而 $f$ 连续。
\end{proof}

\begin{definition}[道路连通分支]\label{def:path-component}
  在 $X$ 上定义关系:对 $x,y\in X$,令
  \[
    x\approx y\iff \exists\ \text{道路}\ \alpha:I\to X\ \text{s.t.}\ \alpha(0)=x,\ \alpha(1)=y.
  \]
  该等价类称为 $X$ 的\emph{道路连通分支}(道路连通分量)。
\end{definition}

\begin{proposition}\label{prop:path-component-eqrel}
  关系 $\approx$ 为等价关系。
\end{proposition}
\begin{proof}
  自反性:取常值道路 $\alpha(t)\equiv x$。

  对称性:若 $\alpha$ 连接 $x$ 与 $y$,则 $\overline\alpha(t)=\alpha(1-t)$ 连接 $y$ 与 $x$。

  传递性:若 $\alpha$ 连接 $x$ 与 $y$,$\beta$ 连接 $y$ 与 $z$,定义
  \[
    \gamma(t)=\begin{cases}
      \alpha(2t), & 0\le t\le\tfrac12,\\
      \beta(2t-1), & \tfrac12\le t\le 1.
    \end{cases}
  \]
  注意 $[0,\tfrac12]\cup[\tfrac12,1]=I$ 为闭覆盖,且两段在交点 $\tfrac12$ 处取值同为 $y$,由引理 \ref{lem:gluing-closed} 得 $\gamma$ 连续,故 $x\approx z$。
\end{proof}

\begin{proposition}\label{prop:path-components-maximal}
  对任意 $x\in X$,道路连通分支 $[x]_{\approx}$ 道路连通,且是包含 $x$ 的极大道路连通子集。
\end{proposition}
\begin{proof}
  道路连通性由定义直接得到。
  若 $C\subseteq X$ 道路连通且 $x\in C$,则任取 $y\in C$,存在道路连接 $x$ 与 $y$,从而 $y\in[x]_{\approx}$,故 $C\subseteq[x]_{\approx}$。
\end{proof}


\begin{proposition}\label{prop:product-path-connected-2}
  若 $X$ 与 $Y$ 道路连通,则 $X\times Y$ 道路连通。
\end{proposition}
\begin{proof}
  任取 $(x,y),(x',y')\in X\times Y$。
  由 $X$ 道路连通,存在道路 $\alpha:I\to X$ 使 $\alpha(0)=x$、$\alpha(1)=x'$;由 $Y$ 道路连通,存在道路 $\beta:I\to Y$ 使 $\beta(0)=y$、$\beta(1)=y'$。
  定义 $\gamma:I\to X\times Y$:$\gamma(t)=(\alpha(t),\beta(t))$。
  则 $p_X\circ\gamma=\alpha$、$p_Y\circ\gamma=\beta$ 连续,故 $\gamma$ 连续,且 $\gamma(0)=(x,y)$、$\gamma(1)=(x',y')$。
\end{proof}

\begin{proposition}\label{prop:product-path-connected}
  设 $\{X_\alpha\}_{\alpha\in\Lambda}$ 为一族道路连通拓扑空间,则其乘积 $\prod_{\alpha\in\Lambda}X_\alpha$ 道路连通。
\end{proposition}
\begin{proof}
  任取 $x=(x_\alpha)_\alpha,\ y=(y_\alpha)_\alpha\in\prod_\alpha X_\alpha$。
  对每个 $\alpha$,取道路 $\gamma_\alpha:I\to X_\alpha$ 使 $\gamma_\alpha(0)=x_\alpha$ 且 $\gamma_\alpha(1)=y_\alpha$。
  定义
  \[
    \gamma:I\to\prod_{\alpha\in\Lambda}X_\alpha,\qquad \gamma(t)=(\gamma_\alpha(t))_{\alpha\in\Lambda}.
  \]
  对任意 $\alpha$ 有 $p_\alpha\circ\gamma=\gamma_\alpha$ 连续,由乘积拓扑的泛性质得 $\gamma$ 连续,且 $\gamma(0)=x$、$\gamma(1)=y$。
\end{proof}

\begin{example}\label{ex:topologist-sine-curve}
  记
  \[
    S=\{(x,\sin\tfrac1x):0<x\le 1\}\subseteq\mathbb R^2,\qquad \Omega=\overline{S}=S\cup\big(\{0\}\times[-1,1]\big).
  \]
  则 $\Omega$ 连通但不道路连通。
\end{example}
\begin{proof}
  令 $\varphi:(0,1]\to\mathbb R^2$,$\varphi(t)=(t,\sin\tfrac1t)$,则 $S=\varphi((0,1])$。
  因 $(0,1]$ 道路连通,故 $S$ 道路连通,从而连通。由推论 \ref{cor:closure-of-connected} 得其闭包 $\Omega$ 连通。

  下面证 $\Omega$ 不道路连通。取 $A=(0,0)\in\{0\}\times[-1,1]$。
  若存在道路 $\alpha:I\to\Omega$ 连接 $A$ 与某点 $B\in S$,设 $f=\pi_1\circ\alpha:I\to\mathbb R$ 为第一坐标。
  则 $f$ 连续且 $f(0)=0$、$f(1)=B_1>0$。
  令 $t_0=\inf\{t\in I:f(t)>0\}$,则 $f(t_0)=0$。
  取数列 $a_n,b_n\downarrow 0$ 使 $\sin\tfrac1{a_n}=1$ 且 $\sin\tfrac1{b_n}=-1$。
  因 $f(t)>0$ 对所有 $t>t_0$ 成立,由介值定理可取 $t_n,s_n\downarrow t_0$ 使 $f(t_n)=a_n$、$f(s_n)=b_n$。
  由于 $a_n,b_n>0$,点 $\alpha(t_n),\alpha(s_n)$ 只能落在 $S$ 上,故其第二坐标分别为 $1$ 与 $-1$。
  令 $g=\pi_2\circ\alpha$,则 $g$ 连续且 $g(t_n)=1$、$g(s_n)=-1$,并且 $t_n,s_n\to t_0$,这与 $g$ 在 $t_0$ 处连续矛盾。
\end{proof}

\begin{definition}[局部连通与局部道路连通]\label{def:local-connected}
  设 $X$ 为拓扑空间。
  \begin{enumerate}
    \item 称 $X$ \emph{局部连通},若对任意 $x\in X$ 及任意 $x$ 的邻域 $U$,存在开集 $V$ 使 $x\in V\subseteq U$ 且 $V$ 连通。
    \item 称 $X$ \emph{局部道路连通},若对任意 $x\in X$ 及任意 $x$ 的邻域 $U$,存在开集 $V$ 使 $x\in V\subseteq U$ 且 $V$ 道路连通。
  \end{enumerate}
\end{definition}

\begin{proposition}\label{prop:components-open-locally-connected}
  若 $X$ 局部连通,则 $X$ 的每个连通分支都是开集。
\end{proposition}
\begin{proof}
  设 $C=[x]$ 为 $x$ 的连通分支。
  由局部连通性,存在连通开集 $V$ 使 $x\in V$。
  因 $V$ 连通且含 $x$,由命题 \ref{prop:connected-component-maximal} 知 $V\subseteq C$。
  故 $x$ 在 $C$ 中有开邻域。任取 $x\in C$ 同理,得 $C$ 开。
\end{proof}

\begin{proposition}\label{prop:path-components-open-locally-path-connected}
  若 $X$ 局部道路连通,则 $X$ 的每个道路连通分支都是开集。
\end{proposition}
\begin{proof}
  设 $P=[x]_{\approx}$ 为 $x$ 的道路连通分支。
  由局部道路连通性,存在道路连通开集 $V$ 使 $x\in V$。
  因 $V$ 道路连通且含 $x$,由命题 \ref{prop:path-components-maximal} 知 $V\subseteq P$。
  故 $P$ 为开集。
\end{proof}

\begin{proposition}\label{prop:components-equal-path-components}
  若 $X$ 局部道路连通,则 $X$ 的连通分支与道路连通分支一致。
\end{proposition}
\begin{proof}
  任一道路连通分支必包含于某个连通分支。
  反过来设 $C$ 为连通分支。由命题 \ref{prop:path-components-open-locally-path-connected},道路连通分支在 $X$ 中开,故 $C$ 可写为若干两两不交开集(道路连通分支)的并。
  由于 $C$ 连通,只能包含其中一个道路连通分支,从而 $C$ 本身即为道路连通分支。
\end{proof}

\begin{corollary}\label{cor:open-connected-implies-path-connected}
  若 $X$ 局部道路连通,且 $U\subseteq X$ 为开且连通的子集,则 $U$ 道路连通。
\end{corollary}
\begin{proof}
  由命题 \ref{prop:components-equal-path-components},$U$ 的连通分支等于道路连通分支。
  但 $U$ 连通意味着它只有一个连通分支,即 $U$ 本身,故 $U$ 道路连通。
\end{proof}

\subsection{同伦与基本群}

\begin{definition}[映射同伦]\label{def:homotopy}
  设 $f,g:X\to Y$ 连续。
  若存在连续映射 $H:X\times I\to Y$ 满足
  \[
    H(x,0)=f(x),\qquad H(x,1)=g(x)\quad(\forall x\in X),
  \]
  则称 $f$ 与 $g$ \emph{同伦},记为 $f\simeq g$,并称 $H$ 为从 $f$ 到 $g$ 的同伦。
\end{definition}

\begin{definition}[道路同伦(端点固定)]\label{def:path-homotopy}
  设 $\alpha,\beta:I\to X$ 为从 $x$ 到 $y$ 的道路(即 $\alpha(0)=\beta(0)=x$ 且 $\alpha(1)=\beta(1)=y$)。
  若存在连续映射 $H:I\times I\to X$ 满足
  \[
    H(s,0)=\alpha(s),\quad H(s,1)=\beta(s),\quad H(0,t)=x,\quad H(1,t)=y,
  \]
  则称 $\alpha$ 与 $\beta$ \emph{端点固定同伦},记为 $\alpha\simeq_{\partial}\beta$。
\end{definition}

\begin{proposition}\label{prop:path-homotopy-eqrel}
  关系 $\simeq_{\partial}$ 是从 $x$ 到 $y$ 的道路集合上的等价关系。
\end{proposition}
\begin{proof}
  自反性:取 $H(s,t)=\alpha(s)$。

  对称性:若 $H$ 给出 $\alpha\simeq_{\partial}\beta$,则 $\overline H(s,t)=H(s,1-t)$ 给出 $\beta\simeq_{\partial}\alpha$。

  传递性:若 $H$ 给出 $\alpha\simeq_{\partial}\beta$,$F$ 给出 $\beta\simeq_{\partial}\gamma$,定义
  \[
    (H\ast F)(s,t)=\begin{cases}
      H(s,2t), & 0\le t\le\tfrac12,\\
      F(s,2t-1), & \tfrac12\le t\le 1.
    \end{cases}
  \]
  注意 $I\times[0,\tfrac12]\cup I\times[\tfrac12,1]=I\times I$ 为闭覆盖,且两段在 $t=\tfrac12$ 处一致,
  由引理 \ref{lem:gluing-closed} 得 $H\ast F$ 连续。
  又 $(H\ast F)(s,0)=\alpha(s)$、$(H\ast F)(s,1)=\gamma(s)$,并且在 $s=0,1$ 处端点固定,故 $\alpha\simeq_{\partial}\gamma$。
\end{proof}

\begin{definition}[道路的拼接与逆道路]\label{def:path-concat-inverse}
  设 $\alpha:I\to X$ 为从 $x$ 到 $y$ 的道路,$\beta:I\to X$ 为从 $y$ 到 $z$ 的道路。
  定义它们的\emph{拼接} $\alpha\ast\beta:I\to X$ 为
  \[
    (\alpha\ast\beta)(t)=\begin{cases}
      \alpha(2t), & 0\le t\le\tfrac12,\\
      \beta(2t-1), & \tfrac12\le t\le 1.
    \end{cases}
  \]
  并定义 $\alpha$ 的\emph{逆道路} $\overline\alpha:I\to X$ 为 $\overline\alpha(t)=\alpha(1-t)$。
\end{definition}

\begin{definition}\label{def:pi1-xy}
  设 $x,y\in X$。
  记
  \[
    \Omega(x,y)=\{\alpha:I\to X:\ \alpha\ \text{连续},\ \alpha(0)=x,\ \alpha(1)=y\}.
  \]
  定义
  \[
    \pi_1(X;x,y)=\Omega(x,y)/\simeq_{\partial},
  \]
  即端点固定同伦类的集合,并用 $[\alpha]$ 表示 $\alpha$ 的等价类。
\end{definition}

\begin{proposition}\label{prop:pi1-compose-well-defined}
  对任意 $x,y,z\in X$,公式
  \[
    \pi_1(X;x,y)\times\pi_1(X;y,z)\to\pi_1(X;x,z),\qquad ([\alpha],[\beta])\mapsto[\alpha\ast\beta]
  \]
  给出良定义的映射。
\end{proposition}
\begin{proof}
  设 $\alpha\simeq_{\partial}\alpha'$ 与 $\beta\simeq_{\partial}\beta'$,分别由同伦 $H,F:I\times I\to X$ 给出。
  定义 $K:I\times I\to X$:
  \[
    K(s,t)=\begin{cases}
      H(2s,t), & 0\le s\le\tfrac12,\\
      F(2s-1,t), & \tfrac12\le s\le 1.
    \end{cases}
  \]
  因 $H(1,t)=y=F(0,t)$,两段在 $s=\tfrac12$ 处一致,且 $[0,\tfrac12]\times I\cup [\tfrac12,1]\times I=I\times I$ 为闭覆盖,
  由引理 \ref{lem:gluing-closed} 得 $K$ 连续。
  又 $K(s,0)=(\alpha\ast\beta)(s)$、$K(s,1)=(\alpha'\ast\beta')(s)$,并且端点固定,故 $\alpha\ast\beta\simeq_{\partial}\alpha'\ast\beta'$。
\end{proof}

\begin{definition}[基本群]\label{def:fundamental-group}
  设 $x_0\in X$。称 $x_0$ 为\emph{基点}。
  定义 $X$ 在基点 $x_0$ 处的\emph{基本群}为
  \[
    \pi_1(X,x_0)=\pi_1(X;x_0,x_0).
  \]
  其中的元素称为以 $x_0$ 为端点的\emph{回路}的同伦类。
\end{definition}

\begin{definition}\label{def:fundamental-group-operation}
  记常值回路 $e_{x_0}:I\to X$,$e_{x_0}(t)=x_0$。
  由命题 \ref{prop:pi1-compose-well-defined},在 $\pi_1(X,x_0)$ 上可定义二元运算
  \[
    [\alpha]\cdot[\beta]=[\alpha\ast\beta].
  \]
\end{definition}

\begin{lemma}\label{lem:path-reparam-homotopy}
  设 $\alpha:I\to X$ 为从 $x$ 到 $y$ 的道路,且 $\varphi:I\to I$ 连续并满足 $\varphi(0)=0,\ \varphi(1)=1$。
  则 $\alpha\circ\varphi\simeq_{\partial}\alpha$。
\end{lemma}
\begin{proof}
  定义 $H:I\times I\to X$:
  \[
    H(s,t)=\alpha\big((1-t)s+t\varphi(s)\big).
  \]
  因 $(s,t)\mapsto (1-t)s+t\varphi(s)$ 连续且值域在 $I$,故 $H$ 连续。
  且 $H(s,0)=\alpha(s)$、$H(s,1)=\alpha(\varphi(s))$,并且 $H(0,t)=x$、$H(1,t)=y$。
\end{proof}

\begin{proposition}\label{prop:pi1-associative}
  运算 $\cdot$ 满足结合律:对任意 $[\alpha],[\beta],[\gamma]\in\pi_1(X,x_0)$,有
  \[
    ([\alpha]\cdot[\beta])\cdot[\gamma]=[\alpha]\cdot([\beta]\cdot[\gamma]).
  \]
\end{proposition}
\begin{proof}
  设 $\delta:I\to X$ 为三段拼接回路
  \[
    \delta(t)=\begin{cases}
      \alpha(3t), & 0\le t\le\tfrac13,\\
      \beta(3t-1), & \tfrac13\le t\le\tfrac23,\\
      \gamma(3t-2), & \tfrac23\le t\le 1.
    \end{cases}
  \]
  由引理 \ref{lem:gluing-closed} 知 $\delta$ 连续。

  设 $\varphi_1,\varphi_2:I\to I$ 为分段线性函数:
  \[
    \varphi_1(t)=\begin{cases}
      \tfrac43 t, & 0\le t\le\tfrac12,\\
      \tfrac23 t+\tfrac13, & \tfrac12\le t\le 1.
    \end{cases}
    \qquad
    \varphi_2(t)=\begin{cases}
      \tfrac23 t, & 0\le t\le\tfrac12,\\
      \tfrac43 t-\tfrac13, & \tfrac12\le t\le 1.
    \end{cases}
  \]
  则可直接验证
  \[
    (\alpha\ast\beta)\ast\gamma=\delta\circ\varphi_1,\qquad \alpha\ast(\beta\ast\gamma)=\delta\circ\varphi_2.
  \]
  由引理 \ref{lem:path-reparam-homotopy} 得 $[\delta\circ\varphi_1]=[\delta]=[\delta\circ\varphi_2]$,从而结论成立。
\end{proof}

\begin{proposition}\label{prop:pi1-identity}
  $[e_{x_0}]$ 为 $\pi_1(X,x_0)$ 中的单位元:对任意 $[\alpha]\in\pi_1(X,x_0)$,有
  \[
    [e_{x_0}]\cdot[\alpha]=[\alpha],\qquad [\alpha]\cdot[e_{x_0}]=[\alpha].
  \]
\end{proposition}
\begin{proof}
  设 $\alpha$ 为以 $x_0$ 为端点的回路。
  定义连续函数 $\rho,\sigma:I\to I$:
  \[
    \rho(t)=\begin{cases}
      2t, & 0\le t\le\tfrac12,\\
      1, & \tfrac12\le t\le 1,
    \end{cases}
    \qquad
    \sigma(t)=\begin{cases}
      0, & 0\le t\le\tfrac12,\\
      2t-1, & \tfrac12\le t\le 1.
    \end{cases}
  \]
  则 $\alpha\circ\rho=\alpha\ast e_{x_0}$,$\alpha\circ\sigma=e_{x_0}\ast\alpha$。
  由引理 \ref{lem:path-reparam-homotopy} 得 $[\alpha\ast e_{x_0}]=[\alpha]=[e_{x_0}\ast\alpha]$。
\end{proof}

\begin{proposition}\label{prop:pi1-inverse}
  对任意 $[\alpha]\in\pi_1(X,x_0)$,其逆元为 $[\overline\alpha]$:
  \[
    [\alpha]\cdot[\overline\alpha]=[e_{x_0}],\qquad [\overline\alpha]\cdot[\alpha]=[e_{x_0}].
  \]
\end{proposition}
\begin{proof}
  设 $\alpha$ 为以 $x_0$ 为端点的回路。
  定义 $H:I\times I\to X$:
  \[
    H(s,t)=\begin{cases}
      \alpha\big(2s(1-t)\big), & 0\le s\le\tfrac12,\\
      \alpha\big(2(1-s)(1-t)\big), & \tfrac12\le s\le 1.
    \end{cases}
  \]
  因 $s=\tfrac12$ 时两段同为 $\alpha(1-t)$,且 $[0,\tfrac12]\times I\cup [\tfrac12,1]\times I=I\times I$ 为闭覆盖,
  由引理 \ref{lem:gluing-closed} 得 $H$ 连续。
  又 $H(s,0)=(\alpha\ast\overline\alpha)(s)$,$H(s,1)=x_0=e_{x_0}(s)$,并且 $H(0,t)=H(1,t)=x_0$,
  故 $\alpha\ast\overline\alpha\simeq_{\partial} e_{x_0}$。
  将 $\alpha$ 替换为 $\overline\alpha$ 即得 $\overline\alpha\ast\alpha\simeq_{\partial} e_{x_0}$。
\end{proof}

\begin{theorem}\label{thm:fundamental-group-is-group}
  配备运算 $\cdot$ 后,$\pi_1(X,x_0)$ 构成群,其单位元为 $[e_{x_0}]$,逆元由 $[\alpha]^{-1}=[\overline\alpha]$ 给出。
\end{theorem}

\begin{definition}[带基点空间与带基点映射]
  \begin{enumerate}[(1)]
    \item \emph{带基点空间}是指一对 $(X,x_0)$,其中 $X$ 为拓扑空间且 $x_0\in X$。
    \item 设 $(X,x_0)$、$(Y,y_0)$ 为带基点空间。若连续映射 $f:X\to Y$ 满足 $f(x_0)=y_0$,则称 $f$ 为\emph{带基点映射},记作
    \[
      f:(X,x_0)\to(Y,y_0).
    \]
  \end{enumerate}
\end{definition}

\begin{definition}
  以带基点空间为对象、带基点映射为态射所成的范畴记为 $\mathbf{Top}_*$。
\end{definition}

\begin{proposition}\label{prop:pi1-induced-hom}
  设 $f:(X,x_0)\to (Y,y_0)$ 为带基点映射,则 $f$ 诱导群同态
  \[
    f_*:\pi_1(X,x_0)\to\pi_1(Y,y_0),\qquad [\alpha]\mapsto [f\circ\alpha].
  \]
\end{proposition}
\begin{proof}
  由带基点映射的定义,若 $\alpha$ 为以 $x_0$ 为端点的回路,则 $f\circ\alpha$ 为以 $y_0$ 为端点的回路。
  若 $\alpha\simeq_{\partial}\beta$,由同伦 $H:I\times I\to X$ 给出,则 $f\circ H$ 给出 $f\circ\alpha\simeq_{\partial} f\circ\beta$,故 $f_*$ 良定义。
  又
  \[
    f_*([\alpha]\cdot[\beta])=f_*([\alpha\ast\beta])=[f\circ(\alpha\ast\beta)]=[(f\circ\alpha)\ast(f\circ\beta)]=f_*([\alpha])\cdot f_*([\beta]),
  \]
  因而 $f_*$ 为群同态。
\end{proof}

\begin{proposition}\label{prop:pi1-functor}
  记 $\mathbf{Grp}$ 为群与群同态所成的范畴。则赋值
  \[
    (X,x_0)\longmapsto \pi_1(X,x_0),\qquad \bigl(f:(X,x_0)\to(Y,y_0)\bigr)\longmapsto f_*
  \]
  给出一个函子 $\pi_1:\mathbf{Top}_*\to\mathbf{Grp}$,即
  \[
    (\mathrm{id}_{(X,x_0)})_* = \mathrm{id}_{\pi_1(X,x_0)},\qquad (g\circ f)_* = g_*\circ f_*.
  \]
\end{proposition}
\begin{proof}
  两个等式均由定义直接验证。
\end{proof}

\begin{corollary}\label{cor:pointed-homeo-implies-pi1-iso}
  若 $(X,x_0)\cong(Y,y_0)$(即存在带基点同胚),则 $\pi_1(X,x_0)\cong\pi_1(Y,y_0)$。
\end{corollary}
\begin{proof}
  设 $f:(X,x_0)\to(Y,y_0)$ 与 $g:(Y,y_0)\to(X,x_0)$ 互逆。
  由函子性,$g_*\circ f_*=(g\circ f)_*=(\mathrm{id})_*=\mathrm{id}$,同理 $f_*\circ g_*=\mathrm{id}$,故 $f_*$ 为同构。
\end{proof}

\begin{theorem}\label{thm:pi1-circle}
  $\pi_1(S^1,1)\cong(\mathbb Z,+)$。
\end{theorem}

\begin{theorem}[Brouwer 不动点定理,二维情形]\label{thm:brouwer-2d}
  设 $f:D^2\to D^2$ 连续,则存在 $x\in D^2$ 使 $f(x)=x$。
\end{theorem}
\begin{proof}
  反证。若 $\forall x\in D^2$ 有 $f(x)\neq x$,则可定义连续映射 $\varphi:D^2\to S^1$:
  对每个 $x$,过 $f(x)$ 与 $x$ 的射线交 $S^1$ 于唯一点 $\varphi(x)$(位于 $x$ 一侧)。
  由几何构造可证 $\varphi$ 连续,且 $\varphi|_{S^1}=\mathrm{id}_{S^1}$。

  设 $i:S^1\hookrightarrow D^2$ 为包含映射,则 $\varphi\circ i=\mathrm{id}_{S^1}$。
  应用函子 $\pi_1$ 得
  \[
    \varphi_*\circ i_*=\mathrm{id}_{\pi_1(S^1,1)}.
  \]
  然而 $\pi_1(D^2,1)=0$($D^2$ 可缩),故 $i_*$ 是零映射,矛盾于 $\pi_1(S^1,1)\cong\mathbb Z\neq 0$。
\end{proof}

\subsection{局部紧空间}

\begin{definition}[局部紧空间]\label{def:locally-compact}
  称拓扑空间 $X$ 为\emph{局部紧}的,若每点都有一个紧邻域。
\end{definition}

\begin{proposition}\label{prop:locally-compact-T2-equiv}
  设 $X$ 为 Hausdorff 空间,则以下条件等价:
  \begin{enumerate}[(1)]
    \item $X$ 局部紧(即每点有紧邻域);
    \item 每点有开邻域 $U$,使得 $\overline U$ 紧。
  \end{enumerate}
\end{proposition}
\begin{proof}
  (2)$\Rightarrow$(1) 显然,取 $A=\overline U$ 即为紧邻域。

  (1)$\Rightarrow$(2) 设 $x\in X$ 有紧邻域 $K$。
  因 $X$ 为 Hausdorff,紧子集 $K$ 是闭的,且是 Hausdorff 空间。
  紧 Hausdorff 空间是正规的(从而正则),故在子空间 $K$ 中,对 $x$ 及开邻域 $K^\circ$($x$ 在 $X$ 中的邻域,故 $x\in K^\circ\subseteq K$),存在 $X$ 中的开集 $U$ 使得 $x\in U\subseteq \overline U \subseteq K^\circ \subseteq K$。
  因 $\overline U$ 是紧集 $K$ 的闭子集,故 $\overline U$ 紧。
\end{proof}

\begin{proposition}\label{prop:locally-compact-T2-regular}
  局部紧 Hausdorff 空间是完全正则空间(从而正则)。
\end{proposition}
\begin{proof}
  设 $X$ 局部紧 Hausdorff。由命题 \ref{prop:locally-compact-T2-equiv},对任意 $x$,存在开邻域 $U$ 使 $\overline U$ 紧。
  $\overline U$ 作为紧 Hausdorff 空间是正规的,故完全正则。
  完全正则性质具有局部性,故 $X$ 完全正则。
  (注:若仅需证明正则,利用 $\overline U$ 的正则性即可:对 $x$ 及闭集 $F\not\ni x$,取 $\overline U$ 中分离它们的开集即可,注意需处理边界情况,或直接引用上述等价命题证明中的构造。)
\end{proof}

\begin{proposition}\label{prop:T2-compact-normal}
  Hausdorff 紧空间为正规空间。
\end{proposition}

\subsection{Baire 空间与纲}

\begin{definition}[Baire 空间]\label{def:baire-space}
  称拓扑空间 $X$ 为 \emph{Baire 空间},若满足:任取 $X$ 中一列稠密开集 $(U_n)_{n\in\mathbb N}$,有 $\bigcap_{n}U_n$ 稠密。
\end{definition}

\begin{definition}[稀疏集与第一纲集]\label{def:meager-nowhere-dense}
  设 $X$ 为拓扑空间,$A\subseteq X$。
  \begin{enumerate}[(1)]
    \item 称 $A$ 为\emph{稀疏集}(或\emph{无处稠密集}),若 $\overline A^\circ=\varnothing$,即 $\overline A$ 的内部为空。
    \item 称 $A$ 为\emph{第一纲集}(meager set),若 $A$ 可写成可数个稀疏集的并。
  \end{enumerate}
  例:$\mathbb R$ 中的有理数集为第一纲集;Cantor 集为稀疏集。
\end{definition}

\begin{proposition}\label{prop:nowhere-dense-equiv}
  $A$ 稀疏(无处稠密)$\Leftrightarrow$ 对任意非空开集 $U$,$A\cap U$ 在 $U$ 中不稠密。
\end{proposition}
\begin{proof}
  ($\Rightarrow$) 设 $A$ 稀疏,即 $\overline A$ 内部为空。对任意非空开集 $U$,若 $A\cap U$ 在 $U$ 中稠密,则 $U\subseteq\overline{A\cap U}\subseteq\overline A$,这蕴含 $\overline A$ 有非空内部 $U$,矛盾。
  
  ($\Leftarrow$) 若 $A$ 不稀疏,则 $\overline A$ 有非空内部 $V$。取 $U=V$,则 $A$ 在 $U$ 中稠密(因 $U\subseteq\overline A$),矛盾。
\end{proof}

\begin{definition}[第一纲与第二纲]\label{def:first-second-category}
  设 $X$ 为拓扑空间,$\Omega\subseteq X$。
  \begin{enumerate}[(I)]
    \item 若存在一列稀疏集 $(E_n)_n$ 使 $\Omega=\bigcup_n E_n$,则称 $\Omega$ 为\emph{第一纲集}(meager set)。
    \item 若 $\Omega$ 不是第一纲集,则称 $\Omega$ 为\emph{第二纲集}。
  \end{enumerate}
  注:第一纲集未必稀疏(例如 $\mathbb Q$ 在 $\mathbb R$ 中)。
\end{definition}

\begin{proposition}\label{prop:baire-equiv-second-category}
  $X$ 为 Baire 空间 $\Leftrightarrow$ $X$ 的每个非空开集为第二纲集。
\end{proposition}
\begin{proof}
  由定义直接可得。
\end{proof}

\begin{proposition}\label{prop:first-category-properties}
  第一纲集对可数并封闭,且向下封闭:若 $A$ 为第一纲集且 $B\subseteq A$,则 $B$ 为第一纲集。
\end{proposition}

\begin{proposition}\label{prop:nowhere-dense-subset}
  稀疏集(无处稠密集)的子集仍为稀疏集。
\end{proposition}

\begin{proposition}\label{prop:nowhere-dense-impl-first-category}
  稀疏集必为第一纲集,反之不然。
\end{proposition}

\begin{proposition}\label{prop:locally-compact-sandwich}
  设 $X$ 为局部紧 Hausdorff 空间,$K$ 紧,$U$ 开且 $K\subseteq U$。
  则存在开集 $V$ 与紧集 $C$ 使得
  \[
    K\subseteq V\subseteq C\subseteq U.
  \]
\end{proposition}
\begin{proof}
  对每个 $x\in K$,取紧邻域 $C_x\ni x$ 使 $C_x\subseteq U$(利用局部紧 Hausdorff 空间的正则性)。
  则 $\{C_x^\circ:x\in K\}$ 覆盖 $K$,由紧性取有限子覆盖
  \[
    K\subseteq C_{x_1}^\circ\cup\cdots\cup C_{x_n}^\circ.
  \]
  令 $V=C_{x_1}^\circ\cup\cdots\cup C_{x_n}^\circ$(开),$C=C_{x_1}\cup\cdots\cup C_{x_n}$(紧),则 $K\subseteq V\subseteq C\subseteq U$。
\end{proof}

\begin{proposition}\label{prop:baire-nowhere-dense-closure}
  设 $X$ 为拓扑空间,$\Omega\subseteq X$。则
  \[
    \Omega\text{ 稀疏}\Leftrightarrow\overline\Omega\text{ 稀疏}\Leftrightarrow\overline\Omega^\circ=\varnothing.
  \]
\end{proposition}
\begin{proof}
  由定义即得。
\end{proof}

\begin{proposition}\label{prop:baire-nowhere-dense-open-equiv}
  设 $X$ 为拓扑空间,$\Omega\subseteq X$。则
  \[
    \Omega\text{ 稀疏}\Leftrightarrow\forall U\text{ 非空开集},\ \exists V\text{ 非空开集使 }V\subseteq U\text{ 且 }V\cap\Omega=\varnothing.
  \]
\end{proposition}
\begin{proof}
  见命题 \ref{prop:nowhere-dense-equiv} 证明。
\end{proof}

\begin{proposition}\label{prop:dense-in-open-subspace}
  设 $X$ 为拓扑空间,$U$ 开集,$\Omega\subseteq X$。则
  \[
    \overline{\Omega\cap U}^U=\overline\Omega\cap U.
  \]
\end{proposition}
\begin{proof}
  ($\subseteq$) $\Omega\cap U\subseteq\overline\Omega\cap U$,后者在 $U$ 中闭,故 $\overline{\Omega\cap U}^U\subseteq\overline\Omega\cap U$。

  ($\supseteq$) 设 $x\in\overline\Omega\cap U$。对任意 $B$ 开于 $U$ 且 $x\in B$,因 $B$ 也开于 $X$,故 $B\cap\Omega\neq\varnothing$。
  而 $B\subseteq U$,故 $B\cap(\Omega\cap U)\neq\varnothing$,即 $x\in\overline{\Omega\cap U}^U$。
\end{proof}

\begin{lemma}\label{lem:category-in-open-subspace}
  设 $X$ 为拓扑空间,$Y$ 为 $X$ 的开子集,$\Omega\subseteq Y$。则:
  \begin{enumerate}[(I)]
    \item $\Omega$ 在 $Y$ 中稀疏 $\Leftrightarrow$ $\Omega$ 在 $X$ 中稀疏;
    \item $\Omega$ 在 $Y$ 中第一纲 $\Leftrightarrow$ $\Omega$ 在 $X$ 中第一纲。
  \end{enumerate}
\end{lemma}
\begin{proof}
  (I) 用命题 \ref{prop:baire-nowhere-dense-open-equiv}。
  
  若 $\Omega$ 在 $X$ 中稀疏,则对任意 $Y$ 中非空开集 $U$,因 $Y$ 开,$U$ 亦为 $X$ 中开集,故存在非空开集 $V\subseteq U$ 使 $V\cap\Omega=\varnothing$,从而 $\Omega$ 在 $Y$ 中稀疏。

  反之,设 $\Omega$ 在 $Y$ 中稀疏。对任意 $X$ 中非空开集 $U$:
  若 $U\cap Y=\varnothing$,取 $V=U$,则 $V\cap\Omega=\varnothing$;
  若 $U\cap Y\neq\varnothing$,则 $U\cap Y$ 为 $Y$ 中非空开集,存在非空开集 $V\subseteq U\cap Y$ 使 $V\cap\Omega=\varnothing$。
  故 $\Omega$ 在 $X$ 中稀疏。

  (II) 若 $\Omega$ 在 $X$ 中第一纲,则 $\Omega=\bigcup_n E_n$,其中每个 $E_n$ 在 $X$ 中稀疏。
  由 (I) 知 $E_n$ 在 $Y$ 中亦稀疏,故 $\Omega$ 在 $Y$ 中第一纲。
  反之,若 $\Omega$ 在 $Y$ 中第一纲,则 $\Omega=\bigcup_n F_n$,其中每个 $F_n$ 在 $Y$ 中稀疏。
  由 (I) 知 $F_n$ 在 $X$ 中亦稀疏,故 $\Omega$ 在 $X$ 中第一纲。
\end{proof}

\begin{proposition}\label{prop:baire-equiv-conditions}
  设 $X$ 为拓扑空间,则以下条件等价:
  \begin{enumerate}[(1)]
    \item $X$ 是 Baire 空间(即可数个稠密开集的交仍稠密);
    \item 每个非空开集为第二纲集;
    \item 每个第一纲集有空内部。
  \end{enumerate}
\end{proposition}
\begin{proof}
  (1)$\Rightarrow$(2):设 $U$ 非空开。若 $U$ 第一纲,则 $U=\bigcup E_n$,$E_n$ 稀疏。
  则 $U_n = X\setminus \overline{E_n}$ 为稠密开集。
  由 Baire 性,$\bigcap U_n$ 稠密,故 $(\bigcap U_n)\cap U \neq \varnothing$。
  但 $(\bigcap U_n)\cap U = U \setminus \bigcup \overline{E_n} \subseteq U \setminus \bigcup E_n = \varnothing$,矛盾。

  (2)$\Rightarrow$(3):若 $A$ 为第一纲集且有非空内部 $U \subseteq A$,则 $U$ 为第一纲集,与 (2) 矛盾。

  (3)$\Rightarrow$(1):设 $U_n$ 稠密开,令 $D = \bigcap U_n$。
  则 $D^c = \bigcup U_n^c$。因 $U_n$ 稠密开,故 $U_n^c$ 稀疏。
  于是 $D^c$ 为第一纲集。由 (3),$(D^c)^\circ = \varnothing$,即 $\overline D = X$,故 $D$ 稠密。
\end{proof}

\begin{proposition}\label{prop:open-iff-complement-closed}
  设 $X$ 为拓扑空间,$\Omega\subseteq X$。则
  \[
    \Omega\text{ 闭且第一纲}\Rightarrow\Omega\text{ 稀疏}.
  \]
  (注意:第一纲闭集的补集未必稠密,但稀疏闭集的补集稠密开。)
\end{proposition}

\begin{definition}[剩余集]\label{def:residual-set}
  设 $X$ 为拓扑空间,$\Omega\subseteq X$。称 $\Omega$ 为\emph{剩余集}(或\emph{余稀疏集},comeager set),若 $\Omega^c$ 为第一纲集。
\end{definition}

\begin{proposition}\label{prop:residual-equiv}
  设 $X$ 为拓扑空间,$\Omega\subseteq X$。则
  \[
    \Omega\text{ 剩余}\Leftrightarrow\exists (U_n)_n\text{ 稠密开集列使 }\Omega\supseteq\bigcap_n U_n.
  \]
\end{proposition}
\begin{proof}
  由定义即得。
\end{proof}

\begin{definition}[局部性质]\label{def:hereditary-property}
  设拓扑空间 $X$ 的性质 $P$ 称为\emph{局部性质},若满足:
  对任意 $X$ 的开覆盖 $\mathcal U$,子集 $\Omega\subseteq X$ 具有性质 $P$ 当且仅当对所有 $U\in\mathcal U$,$\Omega\cap U$ 在 $U$ 中具有性质 $P$。

  例:开集、闭集均具有局部性质。
\end{definition}

\begin{example}\label{ex:open-closed-hereditary}
  开集、闭集均为局部性质。

  设 $X=\bigcup\mathcal U$,$\mathcal U$ 为开覆盖,$\Omega\subseteq X$。

  $\Omega$ 开 $\Rightarrow$ $\forall U\in\mathcal U$,$\Omega\cap U$ 开于 $U$(因 $\Omega\cap U$ 开于 $X$,故开于 $U$)。

  反之,设 $\forall U\in\mathcal U$,$\Omega\cap U$ 开于 $U$。
  则 $\Omega=\bigcup_{U\in\mathcal U}(\Omega\cap U)$,而每个 $\Omega\cap U$ 开于 $U$(也开于 $X$),故 $\Omega$ 开于 $X$。

  闭集类似(或取补)。
\end{example}

\begin{theorem}[Banach 纲定理]\label{thm:banach-category}
  设 $X$ 为拓扑空间。
  \begin{enumerate}[(I)]
    \item 第一纲集具有局部性质;
    \item 稀疏集(即无处稠密集)具有局部性质。
  \end{enumerate}
\end{theorem}
\begin{proof}
  设 $X=\bigcup\mathcal U$,$\mathcal U$ 为开覆盖,$\Omega\subseteq X$。

  (I) 先证 ``$\Rightarrow$''。
  若 $\Omega$ 为第一纲集,则存在稀疏集列 $(A_n)_{n\in\mathbb N}$ 使 $\Omega=\bigcup_n A_n$。
  对任意 $U\in\mathcal U$,有
  \[
    \Omega\cap U=\bigcup_n (A_n\cap U).
  \]
  由命题 \ref{prop:baire-nowhere-dense-open-equiv} 可知 $A_n\cap U$ 在 $U$ 中仍为稀疏集,故 $\Omega\cap U$ 在 $U$ 中为第一纲集。

  再证 ``$\Leftarrow$''。
  设对所有 $U\in\mathcal U$,$\Omega\cap U$ 在 $U$ 中为第一纲集。
  由上一步,对任意开集 $V\subseteq U$,$\Omega\cap V$ 在 $V$ 中亦为第一纲集。

  令
  \[
    \mathscr G=\{V\subseteq X: V\text{ 开且 }\Omega\cap V\text{ 在 }V\text{ 中为第一纲集}\}.
  \]
  由 Zorn 引理,取 $\mathcal V\subseteq\mathscr G$ 为极大两两不交开族,记 $W=\bigcup_{V\in\mathcal V}V$。
  断言 $W$ 在 $X$ 中稠密。否则存在非空开集 $G\subseteq X$ 使 $G\cap W=\varnothing$。
  取 $U\in\mathcal U$ 使 $G\cap U\neq\varnothing$,则 $G\cap U$ 为 $U$ 中非空开集,且 $G\cap U$ 与每个 $V\in\mathcal V$ 不交。
  由前述性质,$\Omega\cap(G\cap U)$ 在 $G\cap U$ 中为第一纲集,即 $G\cap U\in\mathscr G$,从而可将 $G\cap U$ 加入 $\mathcal V$,与极大性矛盾。
  故 $W$ 稠密。

  对每个 $V\in\mathcal V$,可取稀疏集列 $(E_{V,n})_{n\in\mathbb N}$ 使
  \[
    \Omega\cap V=\bigcup_{n\in\mathbb N}E_{V,n}.
  \]
  对每个 $n$,令 $F_n=\bigcup_{V\in\mathcal V}E_{V,n}$。
  因 $\mathcal V$ 两两不交且每个 $E_{V,n}$ 在 $V$ 中稀疏,可由命题 \ref{prop:baire-nowhere-dense-open-equiv} 直接验证 $F_n$ 在 $W$ 中稀疏。
  故
  \[
    \Omega\cap W=\bigcup_{n\in\mathbb N}F_n
  \]
  在 $W$ 中为第一纲集。
  又因 $W$ 为 $X$ 的开子集,$\Omega\cap W$ 在 $X$ 中亦为第一纲集。

  另一方面,由 $W$ 稠密知 $X\setminus W$ 为稀疏闭集,从而 $\Omega\setminus W\subseteq X\setminus W$ 为稀疏集,特别地为第一纲集。
  因此 $\Omega=(\Omega\cap W)\cup(\Omega\setminus W)$ 为第一纲集。

  (II) 设对所有 $U\in\mathcal U$,$\Omega\cap U$ 在 $U$ 中为稀疏集。
  任取非空开集 $G\subseteq X$。
  取 $U\in\mathcal U$ 使 $G\cap U\neq\varnothing$。
  因 $\Omega\cap U$ 在 $U$ 中稀疏,存在非空开集 $V\subseteq G\cap U$ 使 $V\cap\Omega=\varnothing$。
  由命题 \ref{prop:baire-nowhere-dense-open-equiv} 得 $\Omega$ 在 $X$ 中稀疏。
\end{proof}

\begin{theorem}[Baire 纲定理]\label{thm:baire-category-theorem}
  以下空间均为 Baire 空间:
  \begin{enumerate}[(1)]
    \item 完备伪度量空间;
    \item 局部紧 Hausdorff 空间。
  \end{enumerate}
\end{theorem}
\begin{proof}
  (1) 设 $(U_n)$ 为稠密开集列。需证 $\bigcap U_n$ 稠密。
  对任意非空开集 $V_0$,由 $U_1$ 稠密,存在 $x_1$ 及闭球 $\overline{B(x_1, r_1)} \subseteq U_1 \cap V_0$($r_1 < 1/2$)。
  归纳构造闭球套 $\overline{B(x_n, r_n)} \subseteq U_n \cap B(x_{n-1}, r_{n-1})$($r_n < 2^{-n}$)。
  由完备性,$\bigcap \overline{B(x_n, r_n)} \neq \varnothing$,且交点在 $\bigcap U_n \cap V_0$ 中。
  
  (2) 类似,利用局部紧空间的“开集内含紧闭包开集”性质构造紧集套。
  对非空开集 $V_0$,取非空开集 $V_1$ 使 $\overline{V_1} \subseteq U_1 \cap V_0$ 且 $\overline{V_1}$ 紧。
  归纳构造 $\overline{V_n} \subseteq U_n \cap V_{n-1}$ 且 $\overline{V_n}$ 紧。
  由有限交非空性质(紧集套定理),$\bigcap \overline{V_n} \neq \varnothing$。
\end{proof}

\begin{lemma}\label{lem:dense-in-open-cover}
  设 $X=\bigcup\mathcal U$,$\mathcal U$ 为两两不交的开覆盖,$\Omega\subseteq X$。则
  \[
    \Omega\text{ 稠密}\Leftrightarrow\forall U\in\mathcal U,\ \Omega\cap U\text{ 在 }U\text{ 中稠密}.
  \]
\end{lemma}
\begin{proof}
  ($\Rightarrow$) 显然。

  ($\Leftarrow$) 设 $\Omega$ 无处稠密。对任意非空开集 $V$ 开于 $X$,$\mathcal U$ 覆盖 $X$ $\Rightarrow$ 存在 $U\in\mathcal U$ 使 $U\cap V\neq\varnothing$。
  则 $\Omega\cap U$ 在 $U$ 中稠密,故 $(\Omega\cap U)\cap(U\cap V)=\Omega\cap(U\cap V)\neq\varnothing$。
  于是存在 $B$ 开于 $X$ 使 $B\subseteq U\cap V$ 且 $B\cap\Omega\neq\varnothing$ $\Rightarrow$ $B\cap\Omega\cap U=\varnothing$ 矛盾。
\end{proof}

\begin{lemma}\label{lem:first-category-open-cover}
  设 $X$ 为拓扑空间,$\mathcal U$ 为开覆盖,$\Omega\subseteq X$。则
  \[
    \Omega\text{ 第一纲}\Leftrightarrow\forall U\in\mathcal U,\ \Omega\cap U\text{ 在 }U\text{ 中第一纲}.
  \]
\end{lemma}
\begin{proof}
  ($\Rightarrow$) 显然。

  ($\Leftarrow$) 反设 $\Omega$ 在 $X$ 中无处稠密,则对任意 $U\in\mathcal U$,$\Omega\cap U$ 在 $U$ 中无处稠密。
  由 Zorn 引理,取 $\mathcal V\subseteq\{V\text{ 开}:\Omega\cap V\text{ 在 }V\text{ 中第一纲}\}$ 为极大两两不交开族。

  下证 $\bigcup\mathcal V$ 在 $X$ 中稠密。
  反设 $\bigcup\mathcal V$ 不稠密,则存在 $W$ 非空开使 $W\cap(\bigcup\mathcal V)=\varnothing$。
  存在 $U\in\mathcal U$ 使 $U\cap W\neq\varnothing$。而已知 $\Omega\cap U$ 在 $U$ 中第一纲 $\Rightarrow$ $\Omega\cap U\cap W$ 在 $U\cap W$ 中第一纲 $\Rightarrow$ 存在 $B$ 开于 $X$ 使 $B\cap(\Omega\cap U)=\varnothing$ 且 $B\cap\Omega\cap U=\varnothing$ 矛盾于 $\mathcal V$ 极大。

  故 $\bigcup\mathcal V$ 稠密。对每个 $V\in\mathcal V$,$\Omega\cap V$ 在 $V$ 中第一纲 $\Rightarrow$ 在 $X$ 中第一纲。
  故 $\Omega=\Omega\cap(\bigcup\mathcal V)=\bigcup_{V\in\mathcal V}(\Omega\cap V)$ 为第一纲(若 $\mathcal V$ 可数)。
\end{proof}




\end{document}